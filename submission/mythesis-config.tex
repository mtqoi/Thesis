
\PassOptionsToPackage{utf8}{inputenc}
  \usepackage{inputenc}

\PassOptionsToPackage{T1}{fontenc} % T2A for cyrillics
  \usepackage{fontenc}



\PassOptionsToPackage{
parts=true,
  drafting=false,    % print version information on the bottom of the pages
  tocaligned=false, % the left column of the toc will be aligned (no indentation)
  dottedtoc=false,  % page numbers in ToC flushed right
  eulerchapternumbers=true, % use AMS Euler for chapter font (otherwise Palatino)
  linedheaders=false,       % chaper headers will have line above and beneath
  floatperchapter=true,     % numbering per chapter for all floats (i.e., Figure 1.1)
  eulermath=true,  % use awesome Euler fonts for mathematical formulae (only with pdfLaTeX)
  beramono=true,    % toggle a nice monospaced font (w/ bold)
  palatino=true,    % deactivate standard font for loading another one, see the last section at the end of this file for suggestions
  style=classicthesis % classicthesis, arsclassica
}{classicthesis}


% ****************************************************************************************************
% 2. Personal data and user ad-hoc commands (insert your own data here)
% ****************************************************************************************************
%\newcommand{\Title}{Thesis draft - selected chapters\xspace}
\title{Thesis draft}
\newcommand{\myTitle}{Thesis draft - selected chapters\xspace}
%\newcommand{\mySubtitle}{An Homage to The Elements of Typographic Style\xspace}
%\newcommand{\myDegree}{Doktor-Ingfenieur (Dr.-Ing.)\xspace}
\newcommand{\myName}{Matthew Thornton\xspace}
\newcommand{\myProf}{Natalia Korolkova\xspace}
%\newcommand{\myOtherProf}{Put name here\xspace}
%\newcommand{\mySupervisor}{Put name here\xspace}
\newcommand{\myFaculty}{Put data here\xspace}
\newcommand{\myDepartment}{School of Physics and Astronomy\xspace}
\newcommand{\myUni}{University of St Andrews\xspace}
\newcommand{\myLocation}{St Andrews\xspace}
\newcommand{\myTime}{March 2020\xspace}
\newcommand{\myVersion}{\classicthesis}


\usepackage[utf8]{inputenc}
\usepackage[draft]{graphicx}
%\usepackage{graphicx}
\usepackage{amsmath}
\usepackage{physics}
%\usepackage[noBBpl]{mathpazo}
%\usepackage{geometry} \geometry{ a4paper, total={170mm,257mm}, left=20mm, top=20mm,}
%\usepackage{geometry} \geometry{ a4paper, total={170mm,257mm}, left=30mm, right=30mm,}
\usepackage[space]{grffile}
\usepackage{amssymb}
\usepackage{hyperref}	
\usepackage[stable]{footmisc}
\usepackage[nottoc,numbib]{tocbibind}
\usepackage{dsfont}
\usepackage{framed}
%\usepackage[font=small,labelfont=bf]{caption}
\usepackage[font=small]{caption}
\usepackage{subcaption}
\usepackage{wrapfig}
\usepackage{bm}
\usepackage{siunitx}
\usepackage{floatrow}
\usepackage{multirow}
\usepackage{dcolumn}
\usepackage{siunitx}
\usepackage{braket}
\usepackage[nolist]{acronym}
\usepackage{multirow}
\usepackage{etoolbox}
\usepackage[para]{threeparttable}
\usepackage{placeins}
\usepackage{bm}
\AtBeginDocument{\DeclareUnicodeCharacter{00B4}{'}}
%\usepackage{xcolor}

\newcommand{\highlight}[1]{%
  \colorbox{green!30}{$\displaystyle#1$}}
\usepackage{newfloat}
%\DeclareFloatingEnvironment[placement={!ht},name=List]{mylist}
\DeclareFloatingEnvironment[placement={h!},name=List]{mylist}

\newcommand{\eff}{\text{eff}}
\newcommand*\Diff[1]{\mathop{}\!\mathrm{d^#1}}
\newcommand{\ncl}{\hat{a}\left(\hat{a}^\dagger \hat{a} - 1\right)}
\newcommand{\gncl}{\gamma_{\text{NCL}}}
\newcommand{\hc}{\text{h. c.}}
%\newcommand{\d}{\delta}
\newcommand*\erfc{\text{erfc}}
\newcommand{\perr}{\text{p}_{\text{err}}}
\newcommand{\perrps}{\perr\left(\Delta_r, \Delta_\theta\right)}
\newcommand{\pe}{\text{p}_{\text{e}}}
\newcommand{\given}{\; \middle| \;}
\newcommand{\cond}{\; | \;}
\newcommand{\tmsv}{\rho_{\text{TMSV}}}
\newcommand{\rps}{\mathcal{R}_{\text{PS}}}
\newcommand{{\systemB}}{QDS-$b$-QSS-$b$-CV-QPSK}
\newcommand{{\systemF}}{QDS-$f$-QKD-$f$-CV-QPSK}
\newcommand{\ddt}[1][]{\frac{\mathrm{d}#1}{\mathrm{d}t}}
\newcommand{\ddtau}[1][]{\frac{\mathrm{d}#1}{\mathrm{d}\tau}}
%\newcommand{\tr}[1][]{\text{Tr}\left[#1\right]}
\newcommand{\dims}{\left|\mathcal{H}\right|}
\newcommand{\code}[1]{\texttt{#1}}
\newcommand{\ra}[1]{\renewcommand{\arraystretch}{#1}}	
\newcommand{\head}[1]{\multicolumn{1}{c}{#1}}
\newcommand{\enot}[2]{{#1}\!\times\!10^{#2}}
\newcommand{\qout}{q_{\text{out}}}
\newcommand{\pout}{p_{\text{out}}}


\usepackage[
    backend=bibtex8,
%    style=phys,
    style=phys ,
%    style=alphabetic,
%    citestyle=numeric-comp,
%   citestyle = alphabetic,
%    sortlocale=de_DE,
%    natbib=true,
    url=false, 
    doi=false,
    eprint=false,
    isbn=false
]{biblatex}
%\bibliography{Thesis}
%\usepackage{biblatex}


\DeclareGraphicsExtensions{.png, .pdf,.jpg}




\usepackage{color} 
\def\red#1{\textcolor{red}{#1}}
\def\blue#1{\textcolor{blue}{#1}}
\def\MT#1{\textcolor{magenta}{#1}}
\def\cyan#1{\textcolor{cyan}{#1}}
\def\extrafig#1{\textcolor{RoyalBlue}{#1}}

\graphicspath{ {images/} }

\setcounter{tocdepth}{1} % Show sections
%\setcounter{tocdepth}{2} % + subsections
%\setcounter{tocdepth}{3} % + subsubsections
%\setcounter{tocdepth}{4} % + paragraphs
%\setcounter{tocdepth}{5} % + subparagraphs












% ********************************************************************
% Setup, finetuning, and useful commands
% ********************************************************************
\providecommand{\mLyX}{L\kern-.1667em\lower.25em\hbox{Y}\kern-.125emX\@}
\newcommand{\ie}{i.\,e.}
\newcommand{\Ie}{I.\,e.}
\newcommand{\eg}{e.\,g.}
\newcommand{\Eg}{E.\,g.}
% ****************************************************************************************************


% ****************************************************************************************************
% 3. Loading some handy packages
% ****************************************************************************************************
% ********************************************************************
% Packages with options that might require adjustments
% ********************************************************************
\PassOptionsToPackage{ngerman,american, english}{babel} % change this to your language(s), main language last
% Spanish languages need extra options in order to work with this template
%\PassOptionsToPackage{spanish,es-lcroman}{babel}
    \usepackage{babel}

\usepackage{csquotes}
\PassOptionsToPackage{%
  %backend=biber,bibencoding=utf8, %instead of bibtex
  backend=bibtex8,bibencoding=ascii,%
  language=auto,%
  style=numeric-comp,%
  %style=authoryear-comp, % Author 1999, 2010
  %bibstyle=authoryear,dashed=false, % dashed: substitute rep. author with ---
  sorting=nyt, % name, year, title
  maxbibnames=10, % default: 3, et al.
  %backref=true,%
  natbib=true % natbib compatibility mode (\citep and \citet still work)
}{biblatex}
    \usepackage{biblatex}

\PassOptionsToPackage{fleqn}{amsmath}       % math environments and more by the AMS
  \usepackage{amsmath}

% ********************************************************************
% General useful packages
% ********************************************************************
\usepackage{graphicx} %
\usepackage{scrhack} % fix warnings when using KOMA with listings package
\usepackage{xspace} % to get the spacing after macros right
\PassOptionsToPackage{printonlyused,smaller}{acronym}
  \usepackage{acronym} % nice macros for handling all acronyms in the thesis
  %\renewcommand{\bflabel}[1]{{#1}\hfill} % fix the list of acronyms --> no longer working
  %\renewcommand*{\acsfont}[1]{\textsc{#1}}
  %\renewcommand*{\aclabelfont}[1]{\acsfont{#1}}
  %\def\bflabel#1{{#1\hfill}}
  \def\bflabel#1{{\acsfont{#1}\hfill}}
  \def\aclabelfont#1{\acsfont{#1}}
% ****************************************************************************************************
%\usepackage{pgfplots} % External TikZ/PGF support (thanks to Andreas Nautsch)
%\usetikzlibrary{external}
%\tikzexternalize[mode=list and make, prefix=ext-tikz/]
% ****************************************************************************************************


% ****************************************************************************************************
% 4. Setup floats: tables, (sub)figures, and captions
% ****************************************************************************************************
\usepackage{tabularx} % better tables
  \setlength{\extrarowheight}{3pt} % increase table row height
\newcommand{\tableheadline}[1]{\multicolumn{1}{l}{\spacedlowsmallcaps{#1}}}
\newcommand{\myfloatalign}{\centering} % to be used with each float for alignment
%\usepackage{subfig}
% ****************************************************************************************************


% ****************************************************************************************************
% 5. Setup code listings
% ****************************************************************************************************
\usepackage{listings}
%\lstset{emph={trueIndex,root},emphstyle=\color{BlueViolet}}%\underbar} % for special keywords
\lstset{language=[LaTeX]Tex,%C++,
  morekeywords={PassOptionsToPackage,selectlanguage},
  keywordstyle=\color{RoyalBlue},%\bfseries,
  basicstyle=\small\ttfamily,
  %identifierstyle=\color{NavyBlue},
  commentstyle=\color{Green}\ttfamily,
  stringstyle=\rmfamily,
  numbers=none,%left,%
  numberstyle=\scriptsize,%\tiny
  stepnumber=5,
  numbersep=8pt,
  showstringspaces=false,
  breaklines=true,
  %frameround=ftff,
  %frame=single,
  belowcaptionskip=.75\baselineskip
  %frame=L
}
% ****************************************************************************************************




% ****************************************************************************************************
% 6. Last calls before the bar closes
% ****************************************************************************************************
% ********************************************************************
% Her Majesty herself
% ********************************************************************
\usepackage{classicthesis}


% ********************************************************************
% Fine-tune hyperreferences (hyperref should be called last)
% ********************************************************************
\hypersetup{%
  %draft, % hyperref's draft mode, for printing see below
  colorlinks=true, linktocpage=true, pdfstartpage=3, pdfstartview=FitV,%
  % uncomment the following line if you want to have black links (e.g., for printing)
  %colorlinks=false, linktocpage=false, pdfstartpage=3, pdfstartview=FitV, pdfborder={0 0 0},%
  breaklinks=true, pageanchor=true,%
  pdfpagemode=UseNone, %
  % pdfpagemode=UseOutlines,%
  plainpages=false, bookmarksnumbered, bookmarksopen=true, bookmarksopenlevel=1,%
  hypertexnames=true, pdfhighlight=/O,%nesting=true,%frenchlinks,%
  urlcolor=CTurl, linkcolor=CTlink, citecolor=CTcitation, %pagecolor=RoyalBlue,%
  %urlcolor=Black, linkcolor=Black, citecolor=Black, %pagecolor=Black,%
  pdftitle={\myTitle},%
  pdfauthor={\textcopyright\ \myName, \myUni, \myFaculty},%
  pdfsubject={},%
  pdfkeywords={},%
  pdfcreator={pdfLaTeX},%
  pdfproducer={LaTeX with hyperref and classicthesis}%
}


% ********************************************************************
% Setup autoreferences (hyperref and babel)
% ********************************************************************
% There are some issues regarding autorefnames
% http://www.tex.ac.uk/cgi-bin/texfaq2html?label=latexwords
% you have to redefine the macros for the
% language you use, e.g., american, ngerman
% (as chosen when loading babel/AtBeginDocument)
% ********************************************************************
\makeatletter
\@ifpackageloaded{babel}%
  {%
    \addto\extrasamerican{%
      \renewcommand*{\figureautorefname}{Figure}%
      \renewcommand*{\tableautorefname}{Table}%
      \renewcommand*{\partautorefname}{Part}%
      \renewcommand*{\chapterautorefname}{Chapter}%
      \renewcommand*{\sectionautorefname}{Section}%
      \renewcommand*{\subsectionautorefname}{Section}%
      \renewcommand*{\subsubsectionautorefname}{Section}%
    }%
      % Fix to getting autorefs for subfigures right (thanks to Belinda Vogt for changing the definition)
      \providecommand{\subfigureautorefname}{\figureautorefname}%
    }{\relax}
\makeatother



