\chapter{Introduction to quantum cryptography}\label{chapter:crypto_intro}

\section{Conventional (classical) cryptography}

Cryptography is a field probably as old as civilization itself. For as long as communication has existed, so too has the desire to keep information hidden. Both the Greeks and the Romans are known to have used ciphers to encrypt messages \cite{Singh2000}. A cipher, after applied to a message, allows the encrypted message to be freely transmitted and intercepted without an adverse party interpreting its meaning. The intended recipients, however, can undo the effects of the cipher and read the original message. 

One famous example is the Caesar cipher. In the Caesar cipher, each element of the alphabet which makes up the message (``plain'') is assigned a new symbol (``cipher''). Typically this is done by shifting the alphabet by a known quantity, Fig.~\ref{fig:caesar}. The plaintext message is encoded with the cipher, replacing letters from the plain with letters fromt he cipher. This encoded message is known as ``ciphertext'' and now may be freely distributed. At face value, the ciphertext is unreadable to anyone without access to the cipher.

\begin{figure}[htp]
\centering
\captionsetup{width=\linewidth}
\begin{framed}
\begin{align*}
\text{Plain:} &\text{  \code{ABCDEFGHIJKLMNOPQRSTUVWXYZ}} \\
\text{Cipher:} &\text{  \code{FGHIJKLMNOPQRSTUVWXYZABCDE}} \\
\text{Plaintext:} &\text{  I like physics} \\
\text{Ciphertext:} &\text{  N qnpj umdxnhx}
\end{align*}
\end{framed}
\caption{\label{fig:caesar} The cipher alphabet is formed of the plain alphabet shifted $5$ elements to the left. Knowledge of the cipher allows the plaintext message to be recovered.}
\end{figure} %TODO: change this to Hello to Jason Isaacs, or similar.
%TODO: put Hello to Jason, or similar, in my acknowledgements. It'll be banter.


Cryptanalysis -- the art and science of breaking cryptographic systems -- has existed for as long as cryptography, and history of cryptographic development can be viewed as an arms-race between cryptographers and cryptanalysts. The cryptographers, which we canonically call Alice and Bob, continually invent new schemes to perform their secure communication task. The cryptanalyst, which we canonically call Eve, continually tries to break these schemes in order to interfere in Alice and Bob's communication and obtain their messages. For example, the Caesar cipher can be broken by trying all possible shifts of the alphabet and checking which give a sensible message at the output. Against a more general cipher, Eve can perform a statistical analysis on the ciphertext, provided that she knows the language of the message. In English, for example, Eve knows that ``e'' is the most frequently occurring letter, and so the most common letter in the ciphertext is likely to decode to ``e''.

Many advances on the Ceasar cipher have been developed, in which a key (a shared secret piece of information) is used to encrypt and then decrypt a message. While many schemes are secure against decryption, they meet significant practical issues to actually distribute the key. Indeed, the key distribution problem was one of the longstanding and difficult problems which cryptographers have faced over the millenia. Should the shared keys fall into enemy hands, secret messages may be freely decrypted and their sensitive information made public. 

One practical method to distribute keys requires Alice and Bob to meet face-to-face, in advance of their communication, to share the keys which they will use for the next round of communication. While secure, this is impractical. A third party courier could be used as a go-between, but this places an assumption about the trustworthiness of the messenger, and requires an unweildly overhead for large-scale communications. In the second half of the $20^\text{th}$ Century, the following critical question became the focus of intense study of a small group of cryptographers: ``How can Alice and Bob share a secret key, without ever meeting each other?'' 

To solve this problem, Diffie and Hellman \cite{Diffie1976} required a fundamental paradigm shift to the structure of conventional encryption. Normally, as with the Caesar cipher, Fig.~\ref{fig:caesar}, the same key is used to encrypt and to decrypt the message, Fig.~\ref{fig:pubpriv} (a), a structure known as ``private-key (symmetric) cryptography''. We have seen that sharing this key becomes a weak link in the encryption protocol. Diffie and Hellman realised that it is possible to share the key without Alice and Bob ever meeting\footnote{Of course, at the beginning of the protocol Alice and Bob must be sure that they are actually talking to each other \cite{Shneier1996}.} face-to-face. The central idea behind the new ``public-key (asymmetric) cryptography'' was the existence of so-called \emph{one way} functions, which are easy to perform but difficult to invert. 

Diffie and Hellman's proposal runs as follows. At the start of communication, Alice and Bob publicly agree on a function $Y^x \text{ modulo } P$, with $Y < P$. The $Y$ and $P$ are assumed to be public knowledge. For example, they may choose the function $f\left(x\right) 13^x \text{ modulo } 19$. Now, Alice and Bob each choose a number, labelled $A, B$, and keep it secret, e.g. $A = 2$ and $B = 4$. Each number is fed into $f$: $\alpha :=f\left(A\right) = 17$ and $\beta := f\left(B\right) = 4$. The outputs $\alpha, \beta$ are shared between Alice and Bob; crucially, although calculating $\alpha, \beta$ was simple, it is tricky to find $A$ and $B$ from this public information. Finally, Alice calculates $\beta^A \text{ modulo } 19 = 16$ and Bob calculates $\alpha^B \text{ modulo } 19 = 16$: Alice and Bob reach the same number, $16$, which can then be used as the encryption key. This discovery allows Alice and Bob to establish a key entirely over public and insecure communication channels. 



\begin{figure}[htp]
\centering
\captionsetup{width=\linewidth}
\begin{framed}
\begin{subfigure}{0.4\textwidth}
\begin{align*}
m \mapsto \text{Encrypt}_\mathcal{K}\left(m\right) \\
\text{Decrypt}_\mathcal{K}\left[E_\mathcal{K}\left(m\right)\right] \mapsto m
\end{align*}
\caption{}
\end{subfigure}
\begin{subfigure}{0.4\textwidth}
\begin{align*}
m \mapsto \text{Encrypt}_\mathcal{E}\left(m\right) \\
\text{Decrypt}_\mathcal{D}\left[\text{Encrypt}_\mathcal{E}\left(m\right)\right] \mapsto m
\end{align*}
\caption{}
\end{subfigure}
\caption{(a) Private-key encryption. The same key $\mathcal{K}$ allows Alice to encrypt and Bob to decrypt message $m$. (b) Public-key encryption. Alice and Bob use different keys, $\mathcal{E}$ and $\mathcal{D}$ to encrypt and decrypt $m$. The key $\mathcal{E}$ can be public knowledge without affecting the security of the key $\mathcal{D}$.}
\label{fig:pubpriv}
\end{framed}
\end{figure}


Security of this key exchange system relies on the fact that $A$ and $B$ are kept secret, and it is the distinctions and relationships between public and private information which underpin public key cryptography. The function $f$ is sometimes referred to as a ``trapdoor'' or ``one-way'' function, and is typically based on a mathematical problem which is deemed to be computationally hard: that is, even the most powerful computers cannot hope to solve it in a feasible amount of time. Typically the time taken to solve scales exponentially in the size of the key. Perhaps the most well-known hard problem is that of factoring a large integer into primes, which underlies the commonly used RSA protocol \cite{Rivest1978, Schneier1996}.


This type of security, relying on assumptions about computing power, is known as \emph{computational} security. In principle these cryptosystems could be broken with a sufficiently powerful computer, or with algorithmic advances. It has been shown, however, that while these problems are hard for a classical computer, there exist algorithms for a future quantum computer which can break them. The most well known of these is Shor's algorithm \cite{Shor1997},  which provides an exponential speedup in the ability to split an integer into its prime factors. %nstead of taking an exponential number of steps in the length of the integer, the quantum computer will take a number of steps which is only polynomial in integer length. 
The existence of such algorithms which successfully solve the hard problems poses a threat to many commonly used cryptosystems %such as RSA, DSA and ECDHE 
\cite{Rivest1978, Schneier1996, Amiri2015, Nielsen2010, Shor1997}. One must therefore carefully consider how to respond to this threat posed by quantum computers. 

One solution will be to switch the underlying hard problem to a different class of problems, which even a quantum computer cannot solve. This is the approach adopted by the Post-Quantum Cryptography (PQC) community, whose aim is to design protocols based on problems for which no good quantum algorithm is yet known \cite{Bernstein2017, Chen2016, Gagliardoni2017a, Bernstein2009, Alagic2019, Chrome2016}. However, it is still an open question which problems a quantum computer can hope to solve%\footnote{It is even not yet known whether they can solve a larger class of problems than a classical computer}
, and so a premature implementation of a secure system based on a PQC hard-problem, may still be threatened by a quantum computer as new algorithms are developed. 

%In any case, it is clear that the currently implemented cryptographic systems must either be strengthened or replaced, and this may prove challenging. We will briefly discuss some of the challenges, and a possible solution which has gained traction among the conventional cryptography community in recent years, in Chapter~\ref{chapter:aqc}

The second solution to the threat posed by quantum computers is to begin to adopt cryptosystems which are provably secure against a quantum computer. There exist classical protocols for which this is possible \cite{Shamir1979, Blakley1979}, and we will discuss some of them in Sec.~\ref{sec:qss_lit_review}. However for many applications classical cryptography does not allow for such provably secure systems without an initial face-to-face interaction\footnote{To facilitate, for example, the sharing of large, random, secure keys.}, and so one must move to the quantum realm.

Quantum cryptography bases its security not on the assumption of a mathematical problem's difficulty, but on physical laws. Instead of aiming for computational security (albeit security against a quantum computer), quantum cryptography aims to build the stronger \emph{unconditionally secure} (or \emph{information-theoretically secure}) protocols, which cannot be broken even in principle. By basing security on physical laws quantum cryptography requires the sharing of physical systems between players, and we shall see in the remainder of this Thesis that quantum light is a natural object with which to perform such cryptographic tasks. 

One may think of the advantage provided by quantum cryptography in terms of the one-way functions discussed earlier, Fig.~\ref{fig:qutrapdoor}. While the classical one-way functions are only computationally hard, the quantum analogue of the one-way function is provably impossible to invert. For example, if the unknown quantum states are chosen to be non-orthogonal then it is impossible to perfectly determine the classical information which they encode \cite{Nielsen2010, brendon_book}. Any malevolent party attempting to gain information will not do so perfectly, and will thus leave a detectable trace.

\begin{figure}[h!]
\centering
\captionsetup{width=\linewidth}
\begin{framed}
\begin{subfigure}{0.49\linewidth}
\begin{align*}
x_i &\mapsto f\left(x_i\right) \qq{easy} \\
f\left(x_i\right) &\mapsto x_i \qq{hard}
\end{align*}
\caption{}
\end{subfigure}
\begin{subfigure}{0.49\linewidth}
\begin{align*}
x_i &\mapsto \ket{x_i} \qq{easy}\\
\ket{x_i} &\mapsto x_i \qq{impossible}
\end{align*}
\caption{}
\end{subfigure}
\caption{(a) A classical one-way function $f$ is easy to perform but computationally difficult to invert. $f$ is typically based on a hard problem. (b) A quantum one-way function. If the quantum states $\ket{x_i}$ are chosen to be non-orthogonal then it is impossible to perfectly determine the classical information $x$, given a quantum state $\ket{x}$. This forms the basis for quantum cryptosystems, whose security is guaranteed by the no-cloning theorem \cite{Nielsen2010, brendon_book}}
\label{fig:qutrapdoor}
\end{framed}
\end{figure}

%\clearpage
\section{Quantum digital signatures protocols}
%This section will basically be my "literature review" section.
%I will focus on the main thread of QDS developments initially, but I can supplement it by including some of the asian papers later.

%Note: after I have this section I can compare it to the Amiri2015 review paper and to Collins2018 progress report (and to Callum's thesis)


\subsection*{Quantum one-way function}
%Talk about Gottesman and Chuang.
Gottesman and Chuang \cite{Gottesman2001} generalized Lamport's scheme \MT{cite} in $2001$ to build the first Quantum Digital Signatures protocol. The key contribution of their scheme is to replace the one-way function in \MT{cite} with a so-called \emph{quantum one-way function}, thereby securing the signatures protocol against a quantum adversary.

\MT{TODO: chat more about quantum one-way function. Include the "figure" that I currently have in my historical introduction}

A direct analogue of public-key cryptography, their protocol relies on the difficult task, described in Fig.~\MT{X}, of accurately distinguishing between non-orthogonal quantum states. Their security relies on the fact that performing measurement on a state of $n$~qubits can yield at most $n$~bits of information, and so the protocol in Ref.~\cite{Gottesman2001} is designed such that this is insufficient to distinguish between states.

The key tool in the protocol is a quantum $SWAP$ test, Fig.~\MT{X}, which probabilistically determines whether two states are identical. To perform this test, players prepare $\ket{f_x}, \ket{f_{x^\prime}}$ and an additional ancilla $\left(\ket{0} + \ket{1}\right)/\sqrt{2}$. Players perform a Fredkin gate \MT{cite} using the ancilla as a control, and then perform a Hadamard \MT{cite} on the ancilla. In other words, the $SWAP$ test performs the mapping
\begin{equation}
\ket{f_x}\ket{f_{x^\prime}}\frac{\left(\ket{0} + \ket{1}\right)}{\sqrt{2}} \mapsto \frac{\left(\ket{f_x}\ket{f_{x^\prime}} \pm \ket{f_{x^\prime}}\ket{f_x}\right)\ket{y_{\pm}}}{\sqrt{2}}
\end{equation}
with $y_+=0$ and $y_-=1$. Finally, the ancilla qubit is measured in the $0, 1$ basis, and since $\ket{0}, \ket{1}$ are orthogonal they can be distinguished.  Therefore if $x = x^\prime$ the coefficient of $\ket{1}$ is identically zero, and so the $SWAP$ test always outputs $\ket{0}$. If $x \ne x^\prime$ outputs either $\ket{1}$ or $\ket{0}$. 

The probabilistic nature of this test will cause participants in the protocol to sometimes mistake distinct states for identical ones, but the probability that this occurs may be estimated. Crucially, the protocol may be proven secure if states are chosen such that this probability of honest failure is smaller than the probability to correctly distinguish between large entangled states of non-orthogonal qubits. 

The protocol is a significant attempt to generalise and translate structures from the field of classical cryptography to the quantum realm, and it sets the pattern for all subsequent QDS protocols, and so it is worth examining the protocol in detail. Alice has a $1$~bit message $b$ which she would like to sign, and send to Bob and Charlie. In the Distribution state, for each $b$ Alice creates $M$ classical strings $k_m^i$, length $L$. Each classical string is mapped to a corresponding quantum state $\ket{k_m^i}$ of $n$~qubits which are chosen to be highly non-orthogonal. Two of each of these quantum states are sent to Bob and Charlie. The quantum states, $4M$ in total, are Alice's public keys which may be freely distributed--and they may even be given to a dishonest external party. The corresponding classical strings $k_m^i$ are Alice's private keys.

Bob and Charlie each receive two of the $\ket{k_m^i}$. They each perform a $SWAP$ test between their two copies of the public key, to check whether individual copies are equivalent. Then, they should perform a $SWAP$ test between one of Bob's keys and one of Charlie's keys, to test whether they received identical keys to each other. If all $SWAP$ tests pass then the protocol continues to the next step, otherwise it aborts. Bob and Charlie should now store the quantum public keys which they hold.

Later, in the Messaging stage, Alice sends $\left(m, k_m^i\right)$. For each of the $M$ strings $k_m^i$, Bob creates $\ket{k_m^i}$ and performs a $SWAP$ test with his corresponding stored quantum state. If his test passes most of the time then he accepts the message as genuine and transferable, and passes $\left(m, k_m^i\right)$ to Charlie who performs similar tests. 

Although laying the groundwork for practical QDS protocols, this original proposal cannot be implemented. The most pressing problem is the requirement for long-term quantum memory. State-of-the-art technology can store a quantum state for \MT{X}, and so long-term storage of many copies of quantum states with many qubits will be technologically challenging. Furthermore, the need for every party to be able to create and distribute the states and the multiple required $SWAP$ tests render this protocol impractical for implementation. 

However, as we shall see, the structure of this protocol is very closely aligned to classical signatures protocols. Since the public keys are truly public (all of them can be handed to Eve). Furthermore, every recipient is given identical quantum public keys and so the number of recipients does not need to be fixed before the start of the protocol. These requirements are subtly changed in later--more practical--QDS protocols. \MT{make sure I talk about this later.}

\MT{Perhaps talk about repudiation somewhere in this section?}

%\subsection{Andersson2006 (+ implementation)}
\subsection*{QDS implementation}
%Talk about Andersson2006 and Clarke2012

%\subsection{Dunjko2014 (+ implementation}
\subsection*{Removing quantum memory}
The requirement that recipients possess long-term and efficient quantum memory, needed for the above protocol, makes it impractical for realization. The removal of this requirement by Dunjko \emph{et. al.} \cite{Dunjko2014} was one of the major milestones towards a practical QDS which can be implemented. 

The key insight of Ref.~\cite{Dunjko2014} was to effectively replace the quantum public key by a classical one, albeit one which relies on the distribution and measurement of non-orthogonal quantum states. This physical requirement is a practical one, relying on simply linear optics (beamsplitters) and photodetectors capable of distinguishing just between zero and nonzero photon numbers. The storage of classical public keys is clearly no restriction. 

The main difference then between Refs.~\cite{Dunjko2014} and \cite{Gottesman2001}, is that in Dunjko \emph{et. al.}, recipients Bob and Charlie perform photon-number measurement as they receive the quantum states. Remarkably, despite this fundamental change to the nature of the protocol's one-way function, secure QDS is possible. \MT{do I need to revise this sentence? Is it accurate and fair?}

\MT{Include a figure (minipage thing) comparing the one-way functions used by Gottesman2001 and by Dunjko2014.}

In the Distribution stage of the protocol, Alice generates classical strings $\left\{k_j^m\right\}_{j=0}^L$, length $L$, corresponding to each future one-bit message $m$. The $k_j^m$ are chosen uniformly at random from the BPSK alphabet of coherent state phases $\left\{- \alpha, \alpha\right\}$. Alice then forms sequences of coherent states $\rho = \otimes_{j=0}^L \ket{k_j^m}$ which she then distributes to Bob and to Charlie. 

\begin{figure}[htp]
\centering
\includegraphics[width=0.8\linewidth]{multiport.png}
\caption{\label{fig:dunjko2014_multiport}}
\end{figure}

Bob and Charlie pass their received coherent states through the shared optical multiport, Fig.~\ref{fig:dunjko2014_multiport}, which serves to symmetrize their individual quantum states. That is, after the multiport Bob and Charlie's reduced density matrices are identical, which guards against Alice's repudiation attack. Each recipient has two outputs of the multiport. One output, the so-called "null-port" should be monitored for clicks of the photodiode which imply that $\alpha \ne \beta$ (Bob and Charlie have different coherent states, Fig.~\ref{fig:dunjko2014_multiport}) which may imply the presence of an attack. Bob and Charlie should also perform unambiguous state discrimination (USD) on the outputs of their signal ports, which will accurately distinguish between non-orthogonal states $\ket{\alpha}, \ket{-\alpha}$ at the expense that it will sometimes fail to give an answer. 

During Messaging, Alice will declare $\left(m, k_j^m\right)$ which recipients will compare to their USD outcomes. Provided that there are enough matches between Alice's phase declarations $k_j^m$ and Bob/Charlie's USD outcomes, message $m$ is accepted and the protocol has succeeded.

This first protocol avoiding the requirement for quantum memory shows that QDS may be both practical and secure. Furthermore the limited physical requirements--tensor-products of coherent states, beamsplitters and non-photon-number-resolving detectors--are feasible to work with, unlike the large number of superposition qubits required for Ref.~\cite{Gottesman2001}. \MT{Now talk about the implementation paper}.

Notice though the subtle shift between Refs.~\cite{Gottesman2001} and \cite{Dunjko2014}. While previously the number of recipients did not need to be determined until the Messaging stage, here it must be determined before Distribution. After the coherent states have passed through the multiport the number of recipients cannot be changed. 
\MT{I should note later that removing the multiport removes this restriction.} Because of the physical requirement for the optical multiport, it will also be challenging (though possible) to generalize to more recipients, at the expense of altering the protocol to rely on a measurement scheme other than USD. \MT{why?}. Realistic implementation of the multiport also introduces noise and losses due to misalignment and instability, further reducing the efficiency of the protocol.


The most difficult assumption which Ref.~\cite{Dunjko2014} makes, however, is that there should be no eavesdroppers on the quantum channels. This is a strong and impractical assumption, and one which subsequent papers will endeavour to remove.







\subsection{Wallden2015 (+ implementation)}

\subsection{Tokyo installed fibers scheme}
\MT{Perhaps the rest of the DPS-based protocols here too?}

\subsection{Amiri2016 (+ implementation)}

\subsection{Puthoor2016 (+ implementation)}
\MT{Though first talk about side-channel attacks}

\subsection{The other "almost-agile" ones?}

\subsection{An2019 (+ implementation)}

\subsection{Croal2016}
\MT{Discuss DV vs CV first}

\subsection{Quick chat about my PRA}

\subsection{"Classical" unconditionally secure signatures}

\subsection{Extensions to signature schemes}



\section{How to share a secret}\label{sec:qss_lit_review}
\MT{introductory remarks}
Such schemes are regularly employed in \MT{context}, the canonical example being that of a bank. The head of the bank, Alice, wishes to distribute keys to its vault between several deputies, any unknown subset of whom may be dishonest. If the deputies work together and use their keys simultaneously they are able to access the vault, but any nefarious deputies working along should not be able to gain access.

\MT{Make sure to cite the paper "how to share a secret" and talk about its title}.


\subsection*{Classical secret sharing}
Although many existing secret-sharing schemes are already information-theoretically secure while relying only on classical resources \MT{cite}, they may encounter problems when distribution shares of the secret across insecure channels. \MT{talk in more detail about an information-theoretically secure classical scheme} This is analogous to the classical unconditionally secure signature schemes discussed in Sec.~\MT{X}. Secure implementation of a scheme such as \MT{X} requires shared secret keys which, in reality, requires QKD. Thus we may ask whether it is more or less resource-efficient to first run pairwise QKD between players, or to run a "direct"-QSS scheme without first distilling pairwise secret keys. We should expect interesting parallels between QSS and QKD, since intuitively they are very similar, and rely on encryption of classical messages.

Let us consider some examples. Alice wishes to share a secret $m$ between $n$ players, such that any $k$ of them can access $m$. The general framework for this is an $\left(n, k\right)$-threshold scheme, where there are $n$ players in total and any subset of $k$ players can reconstruct the secret. An example of an information-theoretically secure threshold sharing scheme was designed by Shamir in Ref.~\MT{X}. Shamir's scheme relies on polynomial equations and is provably secure even against an adversary with infinite computing power. 

For example, Alice wishes to distribute a secret $m$ between four players, such that any three of them can access $m$. Alice generates a prime number $p$, and the polynomial \MT{what does it mean to add $m$ to a polynomial? TODO: read Shamir's original paper. why do we need to take it modulo p?}
\begin{equation}
\left( a x^2 + b x + m \right) \text{modulo} p.
\end{equation}
Prime $p$ should be chosen larger than any of the coefficients $a, b$ or $m$. Alice then evaluates this polynomial at $4$ different points $x$, and sends the outcomes to each player. These points will be referred to as ``shares".

The polynomial has three unknown coefficients, $a$, $b$ and $m$, and so any three players can combine their shares to create three equations, which may be solved for each unknown. Any fewer points will yield an underdefined system which cannot be solved. An attempt to guess the final share will show that any such message $m$ can be the secret and so such a guessing attempt is useless.

Another analogous threshold secret sharing scheme was built on similar principles by Blakley \MT{cite}. In Blakley's scheme, the message $m$ is defined as a point in a large $k$ dimensional space. Each share is then a hyperplane in a $k-1$~dimensional space, which includes the point $m$. It therefore requires the intersection of all $k$ hyperplanes to reveal $m$. For example, if Alice again wishes to share a secret between four players, such that three of them are able to access $m$, then each share is a two-dimensional plane. The intersection of any two planes is a one-dimensional line containing $m$, and the third plane is required to reduce this line to the point $m$.

While both of these schemes are information-theoretically secure once the shares have been distributed (assuming that each share is securely stored and cannot be stolen), the main issue arises when considering how the shares can be distributed in the first place. If a malevolent party can access the shares while Alice distributes them, then they can reconstruct the secret. In implementation, Shamir's and Blakley's schemes are therefore only as secure as the underlying encryption used to share the shares.

%\subsection*{Early quantum secret sharing}
%\MT{Gottesman1999, Karlsson1999, Hillery1999, Cleve1999}
%
%\MT{distinguish between secret sharing and state sharing}

\subsection{Quantum secret sharing}
One might therefore wish to investigate whether the task of secret sharing can be made secure using quantum resources. It is important to notice that the translation from classical secret sharing to quantum secret sharing is not straightforward. There are at least three directions which one can pursue in order to perform a secret sharing task using quantum mechanics:

\begin{itemize}
\item quantum-assisted classical secret sharing (qCSS): encrypt a classical secret sharing protocol (e.g. Shamir \MT{cite}) using quantum resources. For example, perform pairwise QKD between Alice and each recipient, then encrypt the shares of the classical secret sharing protocol
\item quantum secret sharing (QSS): use quantum states to securely distribute shares of a classical secret
\item quantum state sharing (QStS): securely distribute shares of a quantum state
\end{itemize}

Quantum state sharing is an important and exciting research direction in its own right and helps to establish the close links between quantum secret sharing, QKD and quantum teleportation \MT{cite some stuff}. Despite the fact that both QSS and QStS are natural extensions of classical secret sharing to the quantum realm, , and despite the fact that early work \MT{cite} proposes related protocols for each task, it should be understood that the two are distinct quantum tasks with different goals and hardware requirements, so for the rest of this Thesis we will restrict ourselves to QSS. In what follows we will only refer to the first two options as quantum secret sharing, while the third option we shall refer to as quantum state sharing.

\subsection{Entanglement-based QSS}

%\MT{I can add a fuller explanation of HBB (and the other protocols), since I go into quite some depth in QDS lit review.}
All three directions are discussed at length in the pioneering work by Hillery \emph{et. al.} \cite{Hillery1999}. They propose the use of a GHZ resource state Eq.~\MT{X} shared between three players, which can be used to distribute shares of a classical secret such that collaborating recipients can recover the secret while a dishonest subset of players cannot. Alternatively, the GHZ resource state may be used to distribute shares of a quantum state, such that collaborating players may reconstruct the original quantum state while a dishonest subset of players can gain no information.

Each player chooses independently and at random to measure their state in either the $x$ or $y$ basis \MT{define these}. If for example \MT{give an example of the type of calculation that lets players recover the state. Just copy it from the HBB paper}.  Crucially, knowledge of the outcomes of two players allows one to infer the outcome of the third player.  \MT{talk about how this actually helps to distribute a secret.}

Despite its high resource requirement, and despite the fact that $50\%$ of the resource states are wasted \MT{why?} the HBB protocol has influenced the direction of all subsequent QSS protocols, and the paper was instrumental in demonstrating that multipartite entanglement may be utilized as an important resource for quantum communication protocols. 
\MT{Should I talk somewhere about qCSS in HBB paper?}

Multipartite entanglement is difficult to create and manipulate, and will degrade quickly as it is distributed over a quantum channel exposed to realistic loss or noise levels. Just as QKD has an equivalence between entanglement-based and prepare-and-measure versions \MT{Talk about ekert and BB92?}, it should be expected that the requirement of large multipartite state in Ref.~\cite{Hillery1999} can likewise be reduced \cite{Karlsson1999, Tittel2001, Zhang2005b, Williams2019}. Karlsson \emph{et. al.} \cite{Karlsson1999} propose an entanglement-based QSS scheme which, rather than relying on creation and distribution of the GHZ state, relies on distribution of \emph{pairs} of entangled qubits in a Bell state. 

This configuration allows for correlations between players to be established identically to HBB with more readily accessible resources. Recipients Bob and Charlie can determine with certainty which Bell state Alice sent, which allows Alice to establish a key with Bob/Charlie, which may subsequently be used to encrypt a message. \MT{demonstrate that it can give the same measurement outcomes as HBB with GHZ.}

This protocol drastically reduces the requirements of practical QSS, but the resulting protocol is still non-trivial to implement. The protocol requires Bell states and superpositions of Bell states which will all be degraded over a realistic channel. 

These protocols also introduces a fundamental asymmetry into QSS at the quantum level. While in the HBB protocol any of the three players can be chosen as dealer \MT{check this}, for Ref.~\cite{Karlsson1999} it is established at the time of quantum state distribution that Alice is dealer, which may make the protocol require bespoke hardware.

Both of these protocols \cite{Hillery1999, Karlsson1999} assume both perfect state creation and noiseless and lossless quantum channels. This is an unrealistic assumption and one which must be relaxed before entanglement-based QSS can be implemented securely. Chen \emph{et. al.} \cite{Chen2005a} modify the HBB protocol to the case when the resource state is a noisy-GHZ state. By proposing a method for distillation of multipartite entangled states which may be used before a communication protocol--such as QSS or quantum conferencing--requiring multiparite entanglement as a resource, their entire protocol allows for successful QSS even when the resource state does not violate a Bell inequality.


An important generalization of the HBB scheme allows for analysis of some of the optimal entangled states required to share a secret between more than three players. While one option would be to simply replace the resource state with the N-partite GHZ state

\MT{insert equation}
\MT{problems with npartite GHZ state}

\noindent another option is to generalize to graph states \cite{Markham2008a, Keet2010} or continuous-variable cluster states \cite{Lau2013, Wu2016} under which the tasks qCSS, QSS, QStS and entanglement-based QKD may be united and described within the same framework. A graph state is \MT{add a description}. One advantage of using such a state is that it can allow fro QSS to be completed without collaboration from all recipients, which may help practical QSS to be robust and prevent against denial-of-service attacks from a dishonest internal player\footnote{Though we note that even QKD is susceptible to denial-of-service attack where Eve simply destroys the quantum (or classical) channels between Alice and Bob.}

There have been several attempts to prove security of entanglement-based QSS. As we have seen, security proofs based on highly-entangled GHZ states or graph states become insecure once realistic channel parameters are considered, even though they offer unconditional security in the ideal limit. One way to tackle this is to borrow tools from entanglement-based QKD. Kogias \emph{et. al.} use similar analysis to so-called one-sided device-independent ($1$sDI) QKD \cite{Armstrong2015} \MT{check it is this paper} in order to prove QSS security while modelling channel effects on their CV resource state.

Key to Kogias' protocol is the assumption that neither the measurement device of Bob nor of Charlie should be trusted. Rather, each player is assumed to possess a black-box which can output one of two measurement outcomes, corresponding in the honest case to homodyne measurement in either $x$ or $p$ quadrature. Protocol security is based on monogamy of entanglement \MT{cite} and employs an entropic uncertainty relation which makes no assumption about the action of a dishonest player, by analogy with $1$sDI QKD. To our knowledge Ref.~\cite{Kogias2017} marked the first full security proof of QSS. It was later shown that the resource required for entanglement-based QSS is two-way steering of the shared state \cite{Xiang2017, Xiang2018}, and the optimal Gaussian resource states for a given energy were also considered. 

The links between QSS and $1$sDI QKD explored in Ref.~\cite{Kogias2017} hint at an interesting direction for exploration: what is the relationship between QSS and other quantum communication protocols? It was already shown in Ref.~\cite{Markham2008a} that qCSS, QSS and QStS may be united under the same framework using graph states, while even in the original HBB work \cite{Hillery1999} the links between qCSS (classical secret sharing + QKD) and QSS were acknowledged. Additionally it can be shown \MT{cite} that a QStS protocol may be readily constructed from a teleportation protocol plus QSS (or qCSS or QKD) scheme if Alice teleports a quantum state to Bob, but sends the classical information required for state reconstruction to Charlie.

There are strong links between QSS and quantum conferencing \cite{Wu2016, Ottaviani2017b} which is a natural multipartite generalization of QKD in which $N$ players receive identical keys. Indeed, as shown in Refs.~\cite{Wu2016, Ottaviani2017b} the same resource states and network configurations may be readily used for both QSS and quantum conferencing. It is an open question however whether these additional tasks have the same requirements \cite{Kogias2017, Xiang2017} on the resource state as QSS, or whether the optimal resource state for one protocol remains optimal for another protocol. 


%\MT{Add some more detail to this section. Add some examples of states and the transformations on them, and how they are used for QSS. Add some pictures too.}

%\MT{Add some chat about experimental implementations of EBQSS.}

%\MT{Still got some papers I need to talk about.}

\subsection{Sequential QSS}
Although the above protocols which implement QSS using entangled resource states offer an advanced level of security and neatly demonstrate the important role of entanglement in quantum communication, it is hard to see how they will be preferable to qCSS which can offer equivalent levels of security but without the problems associated with generation and distribution of large entangled states. An entanglement-based scheme may even be fine if the number of players is small--for example the schemes \cite{Karlsson1999} \MT{and others} only relying on Bell-pairs--they cannot be easily scaled to many parties. We note that qCSS scales much more favourably as the number of required quantum channels is linear in the total number of players.

It should still be explored whether there are any QSS protocols which outperform qCSS. One promising direction is that of sequential QSS\footnote{This is sometimes referred to as entanglement-free QSS} in which the QSS task is fulfilled by sharing of a single quantum system between multiple players.

In the first sequential QSS protocol \cite{Zhang2005}, Zhang \emph{et. al.} propose a system in which Bob prepares a single photon state with his choice of polarization, and sends it to Charlie. Charlie performs a unitary operation \MT{what does he do?} on the photon and sends it to Alice, who stores the photon in a quantum memory. This process is repeated many times. Later, Alice will sample some of her stored photons for errors \MT{how?}, and on the remaining photons she performs her unitaries \MT{what are they?} and sends them back to Charlie. \MT{how does the rest of the protocol run?}

Just as prepare-and-measure QKD allows Alice and Bob to mimic the measurement outcomes of a shared entangled state \MT{is it only when they measure in the same basis?} under \MT{criterion} the scheme \cite{Zhang2005} allows players to receive the same measurement outcomes they would if they had shared a GHZ state. Secret sharing then may proceed in the usual way. \MT{talk about this.}

Sequential protocols have the obvious advantage that large entangled states are not required. Even though Ref.~\cite{Zhang2005} proposes to use a quantum memory it is ultimately not necessary for the protocol \MT{talk some more about this.}

\MT{Add more stuff about sequential QSS. Make sure I have cited everything.}


%
%\subsection*{Recent quantum secret sharing}
%\MT{(I should think up better titles for these)}
%
%\subsubsection*{Using entangled states}
%\subsubsection*{Using sequential measurements}
%\subsubsection*{Using a QKD-like setup}


