\chapter{Introduction to quantum cryptography}

\section{Conventional (classical) cryptography}
%introduction and literature review
Goal of chapter: historical overview of development of quantum cryptography. Lead up to a thorough literature review for QDS, QSS and QKD.

Cryptography is a field probably as old as civilization itself. For as long as communication has existed, so too has the desire to keep information hidden. Both the Greeks and the Romans are known to have used ciphers to encrypt messages \MT{cite}. A cipher, after applied to a message, alows the encrypted message to be freely transmitted and intercepted without an adverse party knowing its meaning. The intended recipients, however, can undo the effects of the cipher and read the original message. \MT{It'd be cool to have an example of e.g. a ceasar cipher}.

Cryptanalysis--the art and science of breaking cryptographic systems--has existed for as long as cryptography. The history of cryptographic development can be viewed as a race between cryptographers and cryptanalysts. The cryptographers, Alice and Bob, continually invent new schemes to perform their secure communication task. The cryptanalyst, Eve, continually tries to break these schemes in order to interfere in Alice and Bob's communication.

\MT{The following segue seems quite abrupt}
There are two main important strands of cryptography: private-key and public-key crytography, Fig.~\ref{fig:pubpriv}. In private-key cryptography (also known as \emph{symmetric} cryptography), Alice and Bob share some secret information which they will use to perform some task, and without which an eavesdropper cannot break the system. An important example of private-key cryptography is encryption with a cipher $\mathcal{K}$ which is used to encrypt and decrypt messages. Any party with $\mathcal{K}$ can freely encypt or decrypt any piece of information, while any party lacking knowledge of $\mathcal{K}$ cannot. A relevant example of this type of cryptosystem is the one-time pad (OTP) which is discussed above.

\MT{Perhaps some discussion about how $\mathcal{K}$ can be distributed and attacked}

\MT{I should have a discussion at some point about how public- and private- key cryptography relate to each other and how they are both used in modern infrastructure}

The second strand is public-key cryptography (also known as \emph{asymmetric} cryptography). Here, there are several pieces of information required to run a protocol. Alice is assumed to hold a key $\mathcal{E}$ (her ``private key'') while Bob holds a key $\mathcal{D}$ (Alice's ``public-key'') which is assumed to be publicly known. As an example, Alice may encrypt a message using $\mathcal{E}$ and sends it to Bob, who may decrypt it using $\mathcal{D}$.  \MT{Include more examples of this thing actually being used in practice}

\begin{figure}[htp]
\centering
\captionsetup{width=0.8\linewidth}
\begin{framed}
\begin{subfigure}{0.4\textwidth}
\begin{align*}
m \mapsto \text{Encrypt}_\mathcal{K}\left(m\right) \\
\text{Decrypt}_\mathcal{K}\left[E_\mathcal{K}\left(m\right)\right] \mapsto m
\end{align*}
\caption{}
\end{subfigure}
\begin{subfigure}{0.4\textwidth}
\begin{align*}
m \mapsto \text{Encrypt}_\mathcal{E}\left(m\right) \\
\text{Decrypt}_\mathcal{D}\left[\text{Encrypt}_\mathcal{E}\left(m\right)\right] \mapsto m
\end{align*}
\caption{}
\end{subfigure}
\caption{(a) Private-key encryption. The same key $\mathcal{K}$ allows Alice to encrypt and Bob to decrypt message $m$. (b) Public-key encryption. Alice and Bob use different keys, $\mathcal{E}$ and $\mathcal{D}$ to encrypt and decrypt $m$. The decyption key $\mathcal{D}$ can be public knowledge without affecting the security of the encryption key $\mathcal{E}$.}
\label{fig:pubpriv}
\end{framed}
\end{figure}

Usually, given knowledge of $\mathcal{E}$ it should be easy to derive $\mathcal{D}$. The converse should not be true, and the publicly available $\mathcal{D}$ should give no indication of $\mathcal{E}$. If $\mathcal{E}$ is unique to Alice, then a successful decryption using $\mathcal{D}$ proves sufficient to prove Alice's identity.\footnote{One may be tempted to build a digital signatures protocol on this exchange, but \MT{explain why it is a bad idea, with reference to Simmons}.}

The function $f: \mathcal{E} \mapsto \mathcal{D}$ is sometimes referred to as a ``trapdoor'' or ``one-way'' function, and $f$ is typically based on a mathematical problem which is deemed to be computationally hard, for example \MT{examples} which underly the commonly used \MT{protocols}. \MT{motivate quantum cryptography via breaking of these trapdoors}


\section{Quantum digital signatures protocols}
%This section will basically be my "literature review" section.
%I will focus on the main thread of QDS developments initially, but I can supplement it by including some of the asian papers later.

%Note: after I have this section I can compare it to the Amiri2015 review paper and to Collins2018 progress report (and to Callum's thesis)


\subsection*{Quantum one-way function}
%Talk about Gottesman and Chuang.
Gottesman and Chuang \cite{Gottesman2001} generalized Lamport's scheme \MT{cite} in $2001$ to build the first Quantum Digital Signatures protocol. The key contribution of their scheme is to replace the one-way function in \MT{cite} with a so-called \emph{quantum one-way function}, thereby securing the signatures protocol against a quantum adversary.

\MT{TODO: chat more about quantum one-way function. Include the "figure" that I currently have in my historical introduction}

A direct analogue of public-key cryptography, their protocol relies on the difficult task, described in Fig.~\MT{X}, of accurately distinguishing between non-orthogonal quantum states. Their security relies on the fact that performing measurement on a state of $n$~qubits can yield at most $n$~bits of information, and so the protocol in Ref.~\cite{Gottesman2001} is designed such that this is insufficient to distinguish between states.

The key tool in the protocol is a quantum $SWAP$ test, Fig.~\MT{X}, which probabilistically determines whether two states are identical. To perform this test, players prepare $\ket{f_x}, \ket{f_{x^\prime}}$ and an additional ancilla $\left(\ket{0} + \ket{1}\right)/\sqrt{2}$. Players perform a Fredkin gate \MT{cite} using the ancilla as a control, and then perform a Hadamard \MT{cite} on the ancilla. In other words, the $SWAP$ test performs the mapping
\begin{equation}
\ket{f_x}\ket{f_{x^\prime}}\frac{\left(\ket{0} + \ket{1}\right)}{\sqrt{2}} \mapsto \frac{\left(\ket{f_x}\ket{f_{x^\prime}} \pm \ket{f_{x^\prime}}\ket{f_x}\right)\ket{y_{\pm}}}{\sqrt{2}}
\end{equation}
with $y_+=0$ and $y_-=1$. Finally, the ancilla qubit is measured in the $0, 1$ basis, and since $\ket{0}, \ket{1}$ are orthogonal they can be distinguished.  Therefore if $x = x^\prime$ the coefficient of $\ket{1}$ is identically zero, and so the $SWAP$ test always outputs $\ket{0}$. If $x \ne x^\prime$ outputs either $\ket{1}$ or $\ket{0}$. 

The probabilistic nature of this test will cause participants in the protocol to sometimes mistake distinct states for identical ones, but the probability that this occurs may be estimated. Crucially, the protocol may be proven secure if states are chosen such that this probability of honest failure is smaller than the probability to correctly distinguish between large entangled states of non-orthogonal qubits. 

The protocol is a significant attempt to generalise and translate structures from the field of classical cryptography to the quantum realm, and it sets the pattern for all subsequent QDS protocols, and so it is worth examining the protocol in detail. Alice has a $1$~bit message $b$ which she would like to sign, and send to Bob and Charlie. In the Distribution state, for each $b$ Alice creates $M$ classical strings $k_m^i$, length $L$. Each classical string is mapped to a corresponding quantum state $\ket{k_m^i}$ of $n$~qubits which are chosen to be highly non-orthogonal. Two of each of these quantum states are sent to Bob and Charlie. The quantum states, $4M$ in total, are Alice's public keys which may be freely distributed--and they may even be given to a dishonest external party. The corresponding classical strings $k_m^i$ are Alice's private keys.

Bob and Charlie each receive two of the $\ket{k_m^i}$. They each perform a $SWAP$ test between their two copies of the public key, to check whether individual copies are equivalent. Then, they should perform a $SWAP$ test between one of Bob's keys and one of Charlie's keys, to test whether they received identical keys to each other. If all $SWAP$ tests pass then the protocol continues to the next step, otherwise it aborts. Bob and Charlie should now store the quantum public keys which they hold.

Later, in the Messaging stage, Alice sends $\left(m, k_m^i\right)$. For each of the $M$ strings $k_m^i$, Bob creates $\ket{k_m^i}$ and performs a $SWAP$ test with his corresponding stored quantum state. If his test passes most of the time then he accepts the message as genuine and transferable, and passes $\left(m, k_m^i\right)$ to Charlie who performs similar tests. 

Although laying the groundwork for practical QDS protocols, this original proposal cannot be implemented. The most pressing problem is the requirement for long-term quantum memory. State-of-the-art technology can store a quantum state for \MT{X}, and so long-term storage of many copies of quantum states with many qubits will be technologically challenging. Furthermore, the need for every party to be able to create and distribute the states and the multiple required $SWAP$ tests render this protocol impractical for implementation. 

However, as we shall see, the structure of this protocol is very closely aligned to classical signatures protocols. Since the public keys are truly public (all of them can be handed to Eve). Furthermore, every recipient is given identical quantum public keys and so the number of recipients does not need to be fixed before the start of the protocol. These requirements are subtly changed in later--more practical--QDS protocols. \MT{make sure I talk about this later.}

\MT{Perhaps talk about repudiation somewhere in this section?}

%\subsection{Andersson2006 (+ implementation)}
\subsection*{QDS implementation}
%Talk about Andersson2006 and Clarke2012
A step forward to implementation of QDS occurs in Ref.~\cite{Andersson2006}, in which Andersson \emph{et. al.} replace the tricky to perform $SWAP$ test from Ref.~\cite{Gottesman2001} with a practical state comparison method. The qubits required previously are also replaced by coherent states (qumodes). Combined with the new state comparison scheme, the requirements for QDS have been reduced to just generation and distribution of coherent states and linear optics components (beamsplitters).

\begin{figure}[htp]
\centering
\includegraphics{andersoon2006_state_comparison.png}
\caption{\label{fig:andersson2006_state_comparison}}
\end{figure}

The key step, the comparison of coherent states, is displayed pictorially in Fig.~\ref{fig:andersson2006_state_comparison}. If the photodetector clicks it is a strong indication (a certain indication, in the ideal limit) that $\alpha \ne \beta$. Furthermore this comparison is non-destructive, and simply by placing another beamsplitter in the path of the upper beam, with vacuum input at the fourth port, one recovers $\ket{\alpha}$. Otherwise, for $\alpha \ne \beta$ the output states of the second beamsplitter are identical. This practical state comparison forms the building-block for their QDS protocol. 

\begin{figure}[htp]
\centering
\includegraphics[width=0.8\linewidth]{multiport.png}
\caption{\label{fig:andersson2006_multiport}}
\end{figure}

To allow both parties to perform comparisons, an optical multiport Fig.~\ref{fig:andersson2006_multiport} is used. This allows Bob and Charlie to compare Alice's state declaration with her previously distributed state. It also has the advantage of symmetrizing Bob and Charlie's output states, thus preventing a repudiating Alice. The null ports of the multiport are monitored, since they will click if either Alice is trying to repudiate, or if a malicious party is interacting with the state distribution.

Alice sends coherent states from her alphabet of possible coherent states, both to Bob and to Charlie, and keeps a record of which states she sent. Bob and Charlie feed their states through the shared multiport, thereby ensuring that Alice has sent them identical states (or symmetrizing them if she hasn't), and store their output states in quantum memory. Later, Alice sends the classical message, plus classical information describing which states she had previously sent. Bob and Charlie create the corresponding coherent states, and compare them via the method in Fig.~\ref{fig:andersson2006_state_comparison} with the states retrieved from quantum memory. If no clicks are recorded at the null ports, it is an indication that the message is genuine and the protocol passes.

This protocol was implemented by Clarke \emph{et. al.} in Ref.~\cite{Clarke2012}, where an alphabet with $8$ phase-encoded coherent states was used, and signature lengths $L \sim $\MT{X} were obtained. \MT{I don't think they actually quote one, but I can estimate it from their $g$ value.} To get around the requirement for quantum memory, in Ref.~\cite{Clarke2012} the Messing and Distribution stages occur at the same time so the coherent state corresponding to the chosen private key may be interfered with the distributed quantum signatures. This prevents their scheme from being used in a realistic setting where the Distribution and Messaging stages can typically occur with a delay of days, weeks or even years.




%\subsection{Dunjko2014 (+ implementation}
\subsection*{Removing quantum memory}
The requirement that recipients possess long-term and efficient quantum memory, needed for the above protocols, makes it impractical for realization. The removal of this requirement by Dunjko \emph{et. al.} \cite{Dunjko2014} was one of the major milestones towards a practical QDS which can be implemented. 

The key insight of Ref.~\cite{Dunjko2014} was to effectively replace the quantum public key by a classical one, albeit one which relies on the distribution and measurement of non-orthogonal quantum states. This physical requirement is a practical one, relying on simply linear optics (beamsplitters) and photodetectors capable of distinguishing just between zero and nonzero photon numbers, such as avalanche photodiodes (APDs). The storage of classical public keys is clearly no restriction. 

The main difference then between Refs.~\cite{Dunjko2014} and \cite{Gottesman2001}, is that in Dunjko \emph{et. al.}, recipients Bob and Charlie perform photon-number measurement as they receive the quantum states. Remarkably, despite this fundamental change to the nature of the protocol's one-way function, secure QDS is possible. \MT{do I need to revise this sentence? Is it accurate and fair?}

\MT{Include a figure (minipage thing) comparing the one-way functions used by Gottesman2001 and by Dunjko2014.}

In the Distribution stage of the protocol, Alice generates classical strings $\left\{k_j^m\right\}_{j=0}^L$, length $L$, corresponding to each future one-bit message $m$. The $k_j^m$ are chosen uniformly at random from the BPSK alphabet of coherent state phases $\left\{- \alpha, \alpha\right\}$. Alice then forms sequences of coherent states $\rho = \otimes_{j=0}^L \ket{k_j^m}$ which she then distributes to Bob and to Charlie. 



Bob and Charlie pass their received coherent states through the shared optical multiport, Fig.~\ref{fig:dunjko2014_multiport}, which serves to symmetrize their individual quantum states. That is, after the multiport Bob and Charlie's reduced density matrices are identical, which guards against Alice's repudiation attack. Each recipient has two outputs of the multiport. One output, the so-called "null-port" should be monitored for clicks of the photodiode which imply that $\alpha \ne \beta$ (Bob and Charlie have different coherent states, Fig.~\ref{fig:dunjko2014_multiport}) which may imply the presence of an attack. Bob and Charlie should also perform unambiguous state discrimination (USD) on the outputs of their signal ports, which will accurately distinguish between non-orthogonal states $\ket{\alpha}, \ket{-\alpha}$ at the expense that it will sometimes fail to give an answer. 

During Messaging, Alice will declare $\left(m, k_j^m\right)$ which recipients will compare to their USD outcomes. Provided that there are enough matches between Alice's phase declarations $k_j^m$ and Bob/Charlie's USD outcomes, message $m$ is accepted and the protocol has succeeded.

This first protocol avoiding the requirement for quantum memory shows that QDS may be both practical and secure. Furthermore the limited physical requirements--tensor-products of coherent states, beamsplitters and non-photon-number-resolving detectors--are feasible to work with, unlike the large number of superposition qubits required for Ref.~\cite{Gottesman2001}. \MT{Now talk about the implementation paper}. 

An implementation of a variation of Dunjko's scheme is described in Ref.~\cite{Collins2014}. Collins \emph{et. al.} modify Dunjko's scheme in two key ways. Firstly, a QPSK alphabet Eq.~\MT{X} is used, rather than BPSK. This is in order to make the second modification: instead of using unambiguous state discrimination (USD) measurement, they perform unambiguous state \emph{elimination} (USE) measurement. If the measurement succeeds, rather than being able to say definitively which state was received, a recipient can say with certainty which state was \emph{not} received. This measurement scheme is described further in Fig.~\ref{fig:USE}. The key advantage of the USE measurement scheme is that the probability that the measurement fails is significantly smaller than for USD, and so the resulting QDS scheme gains a boost in efficiency. Indeed, if USE eliminates $N-1$ of $N$ possible states then one knows with certainty which state was sent, USD may be viewed as a special case of the more general USE measurement. The shift from state discrimination to state elimination allows for much greater efficiency in QDS schemes. \MT{Make sure to talk about it later in the context of our QDS - make a graph showing $g$ (or $L$?) under discrimination vs elimination.}

\begin{figure}[htp]
\centering
\includegraphics[width=0.8\linewidth]{USE.png}
\caption{\label{fig:USE}}
\end{figure}

Collins \emph{et. al.} estimate a signature length $L = 5.1 \times 10^{13}$ in order to sign a message. Notice the subtle shift between Refs.~\cite{Gottesman2001} and \cite{Andersson2006, Clarke2012, Dunjko2014, Collins2014}. While previously the number of recipients did not need to be determined until the Messaging stage, here it must be determined before Distribution. After the coherent states have passed through the multiport the number of recipients cannot be changed. 
\MT{I should note later that removing the multiport removes this restriction.} Because of the physical requirements for the optical multiport, it will also be challenging (though possible) to generalize to more recipients, though note that for more than two recipients in Ref.~\cite{Dunjko2014} one may not use $USD$ measurement. \MT{why?}. Realistic implementation of the multiport also introduces noise and losses due to misalignment and instability, further reducing the efficiency of the protocol, and requires Bob and Charlie to be physically connected. In Ref.~\cite{Clarke2012, Collins2014}, for example Bob and Charlie are separated by $5$~m of optical fibre. 

The most difficult assumption which Refs.~\cite{Clarke2012, Dunjko2014, Collins2014} make, however, is that there should be no eavesdroppers on the quantum channels. This is a strong and impractical assumption, and one which subsequent papers endeavour to remove.

\subsection*{Removing multiport}
%\subsection{Wallden2015 (+ implementation)}

The fact that the QDS schemes discussed above require dedicated hardware at the receivers--the optical multiport--makes implementation in real-world situations difficult. The multiport introduces losses and noise, and requires tricky synchronisation between Bob and Charlie in order to correctly interfere the states. The experiment in Ref.~\cite{Collins2014} therefore has Bob and Charlie only separated by $5$~m optical fibre. 

To combat this, Wallden \emph{et. al.} \cite{Wallden2015} propose two QDS schemes specifically designed to run over the same hardware platform as QKD. In particular, they get rid of the multiport which was previously use to symmetrize Bob and Charlie's reduced output states. Their key insight is that rather than symmetrizing their states, it is sufficient to symmetrize their measurement outcomes. Therefore, a step is added to the distribution stage in which Bob and Charlie randomly swap half of their measurement outcomes over a secure classical channel. If Alice can gain no information about which outcomes were swapped then she cannot repudiate. 

This protocol was implemented by Donaldson \emph{et. al.} in Ref.~\cite{Donaldson2016} in which a message is securely signed over distances $500$~m, $1000$~m, and $2000$~m, with no requirement on the physical separation between Bob and Charlie. The secure classical link may be realised via QKD, and so Refs.~\cite{Wallden2015, Donaldson2016} begin to explore the close connections between these two different quantum communication protocols. We explore this further in Chapter.~\MT{X}. Donaldson \emph{et. al.} achieve signature length $L = 1.93 \times 10^9$ using QPSK coherent states and USE measurement. This is a vast improvement over the $L = 5 \times 10^{13}$ required in Ref.~\cite{Collins2014} and means that secure quantum signatures may actually be both useful and practical. 

\subsection*{Allowing Eve}
%\subsection{Amiri2016 (+ implementation)}
All signature schemes considered so far have made the assumption that the quantum distribution channels are secure, that is, they may not be attacked or monitored by an eavesdropper, Eve. This is clearly an unrealistic and unphysical assumption, but was a sensible one while the pressing impracticalities of early QDS schemes (quantum memory, multiport, tricky state comparison tests) were overcome. The emphasis in earlier papers was on dishonesty internal to the protocol, i.e. which attacks can Bob or Charlie mount when they already hold perfect copies of the quantum public keys. However, in a realistic scenario it is clear that an eavesdropper \emph{could} attack the quantum channels as states are being distributed, and so it is important to consider whether this has any effect on QDS security.

Amiri \emph{et. al.} provide a QDS scheme which allows for an Eve to eavesdrop on the quantum channels. In the worst-case scenario it is assumed that Eve will conspire with a dishonest internal player (Bob or Charlie in the case of a forging attack), and so knowledge which Bob/Charlie hold about their own quantum public key measurements is supplemented by knowledge learned through Eve's attack. For short, we describe Bob or Charlie as performing the eavesdropping attack, so as not to confuse the nomenclature. 

The key modification which Ref.~\cite{Amiri2016} makes is to have Alice use \emph{different} private keys (and so different sequences of quantum coherent states) for each recipient. This means that the dishonest recipient is forced to eavesdrop on the honest recipient's quantum channel if he is to gain any information. This is in contrast to earlier protocols in which the dishonest recipient held a perfect copy of the quantum public key, which was identical to that of the honest recipient. 

Because the dishonest player is forced to eavesdrop, he in fact receives a worst copy of the honest player's public key than in the above protocols, and so somewhat counter-intuitively Ref.~\cite{Amiri2016} requires \emph{shorter} $L$ than previous protocols, despite relaxing a security assumption. The protocol relies on sending weak attenuated coherent states, identically to decoy-state BB$84$ \MT{cite}, and \MT{X} detection at the receiver. Because the security of discrete-variable QKD is so advanced, the QDS protocol proposed in Ref.~\cite{Amiri2016} is secure against coherent eavesdropping attacks.\footnote{We will discuss the hierarchy of eavesdropping attacks in Sec.~\MT{X}}

\subsection{Side-channel attacks}
%\subsection{Puthoor2016 (+ implementation)}
%\MT{Though first talk about side-channel attacks}
It should be noted that "security" of a protocol is a theoretical statement, and not a physical one. A protocol is secure with respect to a model of how it operates in the real world, and whether a so-called unconditionally secure protocol can be broken in practice depends on how realistic or practical its underling modelling assumptions are. For example, although in many QKD protocols Eve is allowed to attack the quantum channels and eavesdrop on all communication, she is prevented from attacking the physical devices which are used to implement the protocol. 

For example, the QDS scheme presented in Ref.~\cite{Amiri2016} relies on distribution and detection of quasi-single photons \MT{make sure I have discussed this earlier} in different polarization bases. Physically, Alice must prepare her states before they are sent. A realistic Eve could attack Alice's device in order to gain information about the polarization of the prepared state and so she might gain enough information to forge without detection, even though the protocol is unconditionally secure against conventional types of eavesdropping attack.

An example of such a side-channel attack \MT{define what is a side-channel attack} is the trojan horse attack presented in Ref.~\cite{Jain2014}. Here, Eve shines a bright laser pulse into Alice's device and measures the few back-reflected photons which are scattered back. These photons have picked up the same polarization as Alice imparted to her prepared state, and so Eve is able to infer the chosen polarization basis choice, which gives her an undetected advantage.

To guard against side-channel attacks, honest parties have several options. One direction is to close known side-channels by additional protocol steps or additional hardware. For example, with the trojan horse attack Alice and Bob could add additional filters to their devices to block out light at Eve's required wavelength. It was shown however that Eve can bypass this by breaking Alice's filters, in a way that is undetectable to honest players \MT{cite the laser damage paper}. Closing side-channels in this way may open up the protocols to additional attack methods, which must then be understood, modelled and reacted-to. This places quantum cryptography into the same "cat-and-mouse" development cycles as conventional cryptography. \MT{expand on this, and make sure I have discussed it earlier in the historical overview.}

To break this cycle, and to provide genuinely unconditional security, guaranteed against all conceivable side-channel attacks, there has been a recent push towards device-independent cryptography. The security of device-independent (DI) protocols makes no trust assumptions about the devices used and it may even be assumed that the  devices are held by the malevolent party. DI cryptography is then based entirely on laws of quantum mechanics, specifically on the violation of a bell inequality. \MT{cite some stuff, and expand this paragraph.}

Full DI cryptography, while secure, is difficult to perform and may offer figures of merit which are too pessimistic for the desired application. One may compromise, then, and instead implement measurement device independent (MDI) cryptographic protocols, in which no trust assumptions are placed on the measurement devices (and they can even be owned by Eve), while the state-preparation and sending devices are held by honest parties and are trusted. \MT{cite some papers}

The first MDI QDS scheme is presented by Puthoor \emph{et. al.} in Ref.~\cite{Puthoor2016}.



\subsection{Tokyo installed fibers scheme}
\MT{Perhaps the rest of the DPS-based protocols, or installed-fiber protocols here too?}





\subsection{The other "almost-agile" ones?}

\subsection{An2019 (+ implementation)}

\subsection{Croal2016}
\MT{Discuss DV vs CV first}

\subsection{Quick chat about my PRA}
\MT{perhaps I just want to mention this in the "abstract"/introduction of this chapter? I don't want to talk about it too much because it is what the rest of the chapter is about.}


\subsection{"Classical" unconditionally secure signatures}
\MT{Talk about protocol P2 from Wallden2015, and also Amiri's scheme}

\subsection{Extensions to signature schemes}

\MT{then I can have a section on QSS lit review and introduction, and one for agility. This way the chapters can be shorter and more readable.}