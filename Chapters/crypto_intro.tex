\chapter{Introduction to quantum cryptography}\label{chapter:crypto_intro}

\section{Conventional (classical) cryptography}

Cryptography is a field probably as old as civilization itself. For as long as communication has existed, so too has the desire to keep information hidden. Both the Greeks and the Romans are known to have used ciphers to encrypt messages \cite{Singh2000}. A cipher, after applied to a message, allows the encrypted message to be freely transmitted and intercepted without an adverse party interpreting its meaning. The intended recipients, however, can undo the effects of the cipher and read the original message. 

One famous example is the Caesar cipher. In the Caesar cipher, each element of the alphabet which makes up the message (``plain'') is assigned a new symbol (``cipher''). Typically this is done by shifting the alphabet by a known quantity, Fig.~\ref{fig:caesar}. The plaintext message is encoded with the cipher, replacing letters from the plain with letters fromt he cipher. This encoded message is known as ``ciphertext'' and now may be freely distributed. At face value, the ciphertext is unreadable to anyone without access to the cipher.

\begin{figure}[htp]
\centering
\captionsetup{width=0.8\linewidth}
\begin{framed}
\begin{align*}
\text{Plain:} &\text{  \code{ABCDEFGHIJKLMNOPQRSTUVWXYZ}} \\
\text{Cipher:} &\text{  \code{FGHIJKLMNOPQRSTUVWXYZABCDE}} \\
\text{Plaintext:} &\text{  I like physics} \\
\text{Ciphertext:} &\text{  N qnpj umdxnhx}
\end{align*}
\end{framed}
\caption{\label{fig:caesar} The cipher alphabet is formed of the plain alphabet shifted $5$ elements to the left. Knowledge of the cipher allows the plaintext message to be recovered.}
\end{figure} %TODO: change this to Hello to Jason Isaacs, or similar.
%TODO: put Hello to Jason, or similar, in my acknowledgements. It'll be banter.


Cryptanalysis -- the art and science of breaking cryptographic systems -- has existed for as long as cryptography, and history of cryptographic development can be viewed as an arms-race between cryptographers and cryptanalysts. The cryptographers, which we canonically call Alice and Bob, continually invent new schemes to perform their secure communication task. The cryptanalyst, which we canonically call Eve, continually tries to break these schemes in order to interfere in Alice and Bob's communication and obtain their messages. For example the Caesar cipher can be broken by trying all possible shifts of the alphabet and checking which give a sensible message at the output. Against a more general cipher, Eve can perform a statistical analysis on the ciphertext, provided that she knows the language of the message. In English, for example, Eve knows that ``e'' is the most frequently occurring letter, and so the most common letter in the ciphertext is likely to decode to ``e''.

There are two important strands of cryptography: private-key and public-key crytography, Fig.~\ref{fig:pubpriv}. In private-key cryptography (also known as \emph{symmetric} cryptography), Alice and Bob share some secret information which they will use to perform a task, and without which an eavesdropper cannot break the system. An important example of private-key cryptography is encryption, with the cipher $\mathcal{K}$ used to encrypt and decrypt messages. Any party with $\mathcal{K}$ can freely encypt or decrypt any piece of information, while any party lacking knowledge of $\mathcal{K}$ cannot. %A relevant example of this type of cryptosystem is the one-time pad (OTP) which is discussed above.

%\MT{Perhaps some discussion about how $\mathcal{K}$ can be distributed and attacked}

%\MT{I should have a discussion at some point about how public- and private- key cryptography relate to each other and how they are both used in modern infrastructure}

The second strand is public-key cryptography (also known as \emph{asymmetric} cryptography) \cite{Diffie1976}. Here, there are several pieces of information required to run a protocol. %Alice is assumed to hold a key $\mathcal{E}$ (her ``private key'') while Bob holds a key $\mathcal{D}$ (Alice's ``public-key''). $\mathcal{D}$ is assumed to be publicly known. Alice may encrypt a message using $\mathcal{E}$ and send it to Bob, who can decrypt it using $\mathcal{D}$.  %\MT{Include more examples of this thing actually being used in practice}
Alice possesses two keys, one public, $\mathcal{E}_A$ and one private, $\mathcal{D}_A$. The public key can be deduced from the private key, but it should not be possible to deduce the private key from the public key. If Alice encrypts a message with her private key, then anyone can decrypt it with her public key\footnote{And thus prove that it was indeed Alice who sent the message. This is an example of message authentication in a public-key cryptosystem.}. A public-key cryptosystem can also be used to send encrypted messages. Alice encrypts her message using Bob's public key, and sends it to Bob, who can then decrypt it using his private key.

\begin{figure}[htp]
\centering
\captionsetup{width=0.8\linewidth}
\begin{framed}
\begin{subfigure}{0.4\textwidth}
\begin{align*}
m \mapsto \text{Encrypt}_\mathcal{K}\left(m\right) \\
\text{Decrypt}_\mathcal{K}\left[E_\mathcal{K}\left(m\right)\right] \mapsto m
\end{align*}
\caption{}
\end{subfigure}
\begin{subfigure}{0.4\textwidth}
\begin{align*}
m \mapsto \text{Encrypt}_\mathcal{E}\left(m\right) \\
\text{Decrypt}_\mathcal{D}\left[\text{Encrypt}_\mathcal{E}\left(m\right)\right] \mapsto m
\end{align*}
\caption{}
\end{subfigure}
\caption{(a) Private-key encryption. The same key $\mathcal{K}$ allows Alice to encrypt and Bob to decrypt message $m$. (b) Public-key encryption. Alice and Bob use different keys, $\mathcal{E}$ and $\mathcal{D}$ to encrypt and decrypt $m$. The key $\mathcal{D}$ can be public knowledge without affecting the security of the key $\mathcal{E}$.}
\label{fig:pubpriv}
\end{framed}
\end{figure}

Crucially, given knowledge of $\mathcal{D}$ it should be easy to derive $\mathcal{E}$. The converse should not be true, and the publicly available $\mathcal{E}$ should give no knowledge of $\mathcal{D}$. %If $\mathcal{E}$ is unique to Alice, then a successful decryption using $\mathcal{D}$ proves sufficient to prove Alice's identity.%\footnote{One may be tempted to build a digital signatures protocol on this exchange, but \MT{explain why it is a bad idea, with reference to Simmons}.}

The function $f: \mathcal{D} \mapsto \mathcal{E}$ is sometimes referred to as a ``trapdoor'' or ``one-way'' function. Function $f$ is typically based on a mathematical problem which is deemed to be computationally hard, that is, even the most powerful computers cannot hope to solve it in a feasible amount of time. Typically the time taken to solve scales exponentially in the size of the key. Examples of hard problems include factoring a large integer into primes or the discrete logarithm problem, which underly the commonly used RSA  and Diffie-Hellman protocols \cite{Rivest1978, Diffie1976, Schneier1996}. This type of security, relying on assumptions about computing power, is typically known as \emph{computational} security. In principle these cryptosystems could be broken with a sufficiently powerful computer, or with algorithmic advances.

It has been shown, however, that while these problems are hard for a classical computer, there exist algorithms for a future quantum computer which can break them. The most well known of these is Shor's algorithm \cite{Shor1997},  which provides an exponential speedup in the ability to split an integer into its prime factors. %nstead of taking an exponential number of steps in the length of the integer, the quantum computer will take a number of steps which is only polynomial in integer length. 
The existence of such algorithms which successfully solve the hard problems poses a threat to many commonly used cryptosystems %such as RSA, DSA and ECDHE 
\cite{Rivest1978, Schneier1996, Amiri2015, Nielsen2010, Shor1997}. One must therefore carefully consider how to respond to this threat posed by quantum computers. 

One solution will be to switch the underlying hard problem to a different class of problems, which even a quantum computer cannot solve. This is the approach adopted by the Post-Quantum Cryptography (PQC) community, whose aim is to design protocols based on problems for which no good quantum algorithm is yet known \cite{Bernstein2017, Chen2016, Gagliardoni2017a, Bernstein2009, Alagic2019, Chrome2016}. However, it is still an open question which problems a quantum computer can hope to solve%\footnote{It is even not yet known whether they can solve a larger class of problems than a classical computer}
, and so a premature implementation of a secure system based on a PQC hard-problem, may still be threatened by a quantum computer as new algorithms are developed. 

%In any case, it is clear that the currently implemented cryptographic systems must either be strengthened or replaced, and this may prove challenging. We will briefly discuss some of the challenges, and a possible solution which has gained traction among the conventional cryptography community in recent years, in Chapter~\ref{chapter:aqc}

The second solution to the threat posed by quantum computers is to begin to adopt cryptosystems which are provably secure against a quantum computer. There exist classical protocols for which this is possible \cite{Shamir1979, Blakley1979}, and we will discuss some of them in Sec.~\ref{sec:qss_lit_review}. However for many applications classical cryptography does not allow for such provably secure systems without an initial face-to-face interaction\footnote{To facilitate, for example, the sharing of large, random, secure keys.}, and so one must move to the quantum realm.

Quantum cryptography bases its security not on the assumption of a mathematical problem's difficulty, but on physical laws. Instead of aiming for computational security (albeit security against a quantum computer), quantum cryptography aims to build the stronger \emph{unconditionally secure} (or \emph{information-theoretically secure}) protocols, which cannot be broken even in principle. By basing security on physical laws, quantum cryptography requires the sharing of physical systems between players, and as we shall see in the remainder of this Thesis, quantum light is a natural object with which to perform such cryptographic tasks. 

One may think of the advantage provided by quantum cryptography in terms of the one-way functions discussed earlier, Fig.~\ref{fig:qutrapdoor}. While the classical one-way functions are only computationally hard, the quantum analogue of the one-way function is provably impossible to invert. For example, if the unknown quantum states are chosen to be non-orthogonal then it is impossible to perfectly determine the classical information which they encode \cite{Nielsen2010, brendon_book}. Any malevolent party attempting to gain information will not do so perfectly, and will thus leave a detectable trace.

\begin{figure}[h!]
\centering
\captionsetup{width=0.8\linewidth}
\begin{framed}
\begin{subfigure}{0.49\linewidth}
\begin{align*}
x_i &\mapsto f\left(x_i\right) \qq{easy} \\
f\left(x_i\right) &\mapsto x_i \qq{hard}
\end{align*}
\caption{}
\end{subfigure}
\begin{subfigure}{0.49\linewidth}
\begin{align*}
x_i &\mapsto \ket{x_i} \qq{easy}\\
\ket{x_i} &\mapsto x_i \qq{impossible}
\end{align*}
\caption{}
\end{subfigure}
\caption{(a) A classical one-way function $f$ is easy to perform but computationally difficult to invert. $f$ is typically based on a hard problem. (b) A quantum one-way function. If the quantum states $\ket{x_i}$ are chosen to be non-orthogonal then it is impossible to perfectly determine the classical information $x$, given a quantum state $\ket{x}$. This forms the basis for quantum cryptosystems, whose security is guaranteed by the no-cloning theorem \cite{Nielsen2010, brendon_book}}
\label{fig:qutrapdoor}
\end{framed}
\end{figure}

%\clearpage
\section{Quantum digital signatures protocols}
%This section will basically be my "literature review" section.
%I will focus on the main thread of QDS developments initially, but I can supplement it by including some of the asian papers later.

%Note: after I have this section I can compare it to the Amiri2015 review paper and to Collins2018 progress report (and to Callum's thesis)


\subsection*{Quantum one-way function}
%Talk about Gottesman and Chuang.
Gottesman and Chuang \cite{Gottesman2001} generalized Lamport's scheme \MT{cite} in $2001$ to build the first Quantum Digital Signatures protocol. The key contribution of their scheme is to replace the one-way function in \MT{cite} with a so-called \emph{quantum one-way function}, thereby securing the signatures protocol against a quantum adversary.

\MT{TODO: chat more about quantum one-way function. Include the "figure" that I currently have in my historical introduction}

A direct analogue of public-key cryptography, their protocol relies on the difficult task, described in Fig.~\MT{X}, of accurately distinguishing between non-orthogonal quantum states. Their security relies on the fact that performing measurement on a state of $n$~qubits can yield at most $n$~bits of information, and so the protocol in Ref.~\cite{Gottesman2001} is designed such that this is insufficient to distinguish between states.

The key tool in the protocol is a quantum $SWAP$ test, Fig.~\MT{X}, which probabilistically determines whether two states are identical. To perform this test, players prepare $\ket{f_x}, \ket{f_{x^\prime}}$ and an additional ancilla $\left(\ket{0} + \ket{1}\right)/\sqrt{2}$. Players perform a Fredkin gate \MT{cite} using the ancilla as a control, and then perform a Hadamard \MT{cite} on the ancilla. In other words, the $SWAP$ test performs the mapping
\begin{equation}
\ket{f_x}\ket{f_{x^\prime}}\frac{\left(\ket{0} + \ket{1}\right)}{\sqrt{2}} \mapsto \frac{\left(\ket{f_x}\ket{f_{x^\prime}} \pm \ket{f_{x^\prime}}\ket{f_x}\right)\ket{y_{\pm}}}{\sqrt{2}}
\end{equation}
with $y_+=0$ and $y_-=1$. Finally, the ancilla qubit is measured in the $0, 1$ basis, and since $\ket{0}, \ket{1}$ are orthogonal they can be distinguished.  Therefore if $x = x^\prime$ the coefficient of $\ket{1}$ is identically zero, and so the $SWAP$ test always outputs $\ket{0}$. If $x \ne x^\prime$ outputs either $\ket{1}$ or $\ket{0}$. 

The probabilistic nature of this test will cause participants in the protocol to sometimes mistake distinct states for identical ones, but the probability that this occurs may be estimated. Crucially, the protocol may be proven secure if states are chosen such that this probability of honest failure is smaller than the probability to correctly distinguish between large entangled states of non-orthogonal qubits. 

The protocol is a significant attempt to generalise and translate structures from the field of classical cryptography to the quantum realm, and it sets the pattern for all subsequent QDS protocols, and so it is worth examining the protocol in detail. Alice has a $1$~bit message $b$ which she would like to sign, and send to Bob and Charlie. In the Distribution state, for each $b$ Alice creates $M$ classical strings $k_m^i$, length $L$. Each classical string is mapped to a corresponding quantum state $\ket{k_m^i}$ of $n$~qubits which are chosen to be highly non-orthogonal. Two of each of these quantum states are sent to Bob and Charlie. The quantum states, $4M$ in total, are Alice's public keys which may be freely distributed--and they may even be given to a dishonest external party. The corresponding classical strings $k_m^i$ are Alice's private keys.

Bob and Charlie each receive two of the $\ket{k_m^i}$. They each perform a $SWAP$ test between their two copies of the public key, to check whether individual copies are equivalent. Then, they should perform a $SWAP$ test between one of Bob's keys and one of Charlie's keys, to test whether they received identical keys to each other. If all $SWAP$ tests pass then the protocol continues to the next step, otherwise it aborts. Bob and Charlie should now store the quantum public keys which they hold.

Later, in the Messaging stage, Alice sends $\left(m, k_m^i\right)$. For each of the $M$ strings $k_m^i$, Bob creates $\ket{k_m^i}$ and performs a $SWAP$ test with his corresponding stored quantum state. If his test passes most of the time then he accepts the message as genuine and transferable, and passes $\left(m, k_m^i\right)$ to Charlie who performs similar tests. 

Although laying the groundwork for practical QDS protocols, this original proposal cannot be implemented. The most pressing problem is the requirement for long-term quantum memory. State-of-the-art technology can store a quantum state for \MT{X}, and so long-term storage of many copies of quantum states with many qubits will be technologically challenging. Furthermore, the need for every party to be able to create and distribute the states and the multiple required $SWAP$ tests render this protocol impractical for implementation. 

However, as we shall see, the structure of this protocol is very closely aligned to classical signatures protocols. Since the public keys are truly public (all of them can be handed to Eve). Furthermore, every recipient is given identical quantum public keys and so the number of recipients does not need to be fixed before the start of the protocol. These requirements are subtly changed in later--more practical--QDS protocols. \MT{make sure I talk about this later.}

\MT{Perhaps talk about repudiation somewhere in this section?}

%\subsection{Andersson2006 (+ implementation)}
\subsection*{QDS implementation}
%Talk about Andersson2006 and Clarke2012
A step forward to implementation of QDS occurs in Ref.~\cite{Andersson2006}, in which Andersson \emph{et. al.} replace the tricky to perform $SWAP$ test from Ref.~\cite{Gottesman2001} with a practical state comparison method. The qubits required previously are also replaced by coherent states (qumodes). Combined with the new state comparison scheme, the requirements for QDS have been reduced to just generation and distribution of coherent states and linear optics components (beamsplitters).

\begin{figure}[htp]
\centering
\includegraphics{andersoon2006_state_comparison.png}
\caption{\label{fig:andersson2006_state_comparison}}
\end{figure}

The key step, the comparison of coherent states, is displayed pictorially in Fig.~\ref{fig:andersson2006_state_comparison}. If the photodetector clicks it is a strong indication (a certain indication, in the ideal limit) that $\alpha \ne \beta$. Furthermore this comparison is non-destructive, and simply by placing another beamsplitter in the path of the upper beam, with vacuum input at the fourth port, one recovers $\ket{\alpha}$. Otherwise, for $\alpha \ne \beta$ the output states of the second beamsplitter are identical. This practical state comparison forms the building-block for their QDS protocol. 

\begin{figure}[htp]
\centering
\includegraphics[width=0.8\linewidth]{multiport.png}
\caption{\label{fig:andersson2006_multiport}}
\end{figure}

To allow both parties to perform comparisons, an optical multiport Fig.~\ref{fig:andersson2006_multiport} is used. This allows Bob and Charlie to compare Alice's state declaration with her previously distributed state. It also has the advantage of symmetrizing Bob and Charlie's output states, thus preventing a repudiating Alice. The null ports of the multiport are monitored, since they will click if either Alice is trying to repudiate, or if a malicious party is interacting with the state distribution.

Alice sends coherent states from her alphabet of possible coherent states, both to Bob and to Charlie, and keeps a record of which states she sent. Bob and Charlie feed their states through the shared multiport, thereby ensuring that Alice has sent them identical states (or symmetrizing them if she hasn't), and store their output states in quantum memory. Later, Alice sends the classical message, plus classical information describing which states she had previously sent. Bob and Charlie create the corresponding coherent states, and compare them via the method in Fig.~\ref{fig:andersson2006_state_comparison} with the states retrieved from quantum memory. If no clicks are recorded at the null ports, it is an indication that the message is genuine and the protocol passes.

This protocol was implemented by Clarke \emph{et. al.} in Ref.~\cite{Clarke2012}, where an alphabet with $8$ phase-encoded coherent states was used, and signature lengths $L \sim $\MT{X} were obtained. \MT{I don't think they actually quote one, but I can estimate it from their $g$ value.} To get around the requirement for quantum memory, in Ref.~\cite{Clarke2012} the Messing and Distribution stages occur at the same time so the coherent state corresponding to the chosen private key may be interfered with the distributed quantum signatures. This prevents their scheme from being used in a realistic setting where the Distribution and Messaging stages can typically occur with a delay of days, weeks or even years.




%\subsection{Dunjko2014 (+ implementation}
\subsection*{Removing quantum memory}
The requirement that recipients possess long-term and efficient quantum memory, needed for the above protocols, makes it impractical for realization. The removal of this requirement by Dunjko \emph{et. al.} \cite{Dunjko2014} was one of the major milestones towards a practical QDS which can be implemented. 

The key insight of Ref.~\cite{Dunjko2014} was to effectively replace the quantum public key by a classical one, albeit one which relies on the distribution and measurement of non-orthogonal quantum states. This physical requirement is a practical one, relying on simply linear optics (beamsplitters) and photodetectors capable of distinguishing just between zero and nonzero photon numbers, such as avalanche photodiodes (APDs). The storage of classical public keys is clearly no restriction. 

The main difference then between Refs.~\cite{Dunjko2014} and \cite{Gottesman2001}, is that in Dunjko \emph{et. al.}, recipients Bob and Charlie perform photon-number measurement as they receive the quantum states. Remarkably, despite this fundamental change to the nature of the protocol's one-way function, secure QDS is possible. \MT{do I need to revise this sentence? Is it accurate and fair?}

\MT{Include a figure (minipage thing) comparing the one-way functions used by Gottesman2001 and by Dunjko2014.}

In the Distribution stage of the protocol, Alice generates classical strings $\left\{k_j^m\right\}_{j=0}^L$, length $L$, corresponding to each future one-bit message $m$. The $k_j^m$ are chosen uniformly at random from the BPSK alphabet of coherent state phases $\left\{- \alpha, \alpha\right\}$. Alice then forms sequences of coherent states $\rho = \otimes_{j=0}^L \ket{k_j^m}$ which she then distributes to Bob and to Charlie. 



Bob and Charlie pass their received coherent states through the shared optical multiport, Fig.~\ref{fig:dunjko2014_multiport}, which serves to symmetrize their individual quantum states. That is, after the multiport Bob and Charlie's reduced density matrices are identical, which guards against Alice's repudiation attack. Each recipient has two outputs of the multiport. One output, the so-called "null-port" should be monitored for clicks of the photodiode which imply that $\alpha \ne \beta$ (Bob and Charlie have different coherent states, Fig.~\ref{fig:dunjko2014_multiport}) which may imply the presence of an attack. Bob and Charlie should also perform unambiguous state discrimination (USD) on the outputs of their signal ports, which will accurately distinguish between non-orthogonal states $\ket{\alpha}, \ket{-\alpha}$ at the expense that it will sometimes fail to give an answer. 

During Messaging, Alice will declare $\left(m, k_j^m\right)$ which recipients will compare to their USD outcomes. Provided that there are enough matches between Alice's phase declarations $k_j^m$ and Bob/Charlie's USD outcomes, message $m$ is accepted and the protocol has succeeded.

This first protocol avoiding the requirement for quantum memory shows that QDS may be both practical and secure. Furthermore the limited physical requirements--tensor-products of coherent states, beamsplitters and non-photon-number-resolving detectors--are feasible to work with, unlike the large number of superposition qubits required for Ref.~\cite{Gottesman2001}. \MT{Now talk about the implementation paper}. 

An implementation of a variation of Dunjko's scheme is described in Ref.~\cite{Collins2014}. Collins \emph{et. al.} modify Dunjko's scheme in two key ways. Firstly, a QPSK alphabet Eq.~\MT{X} is used, rather than BPSK. This is in order to make the second modification: instead of using unambiguous state discrimination (USD) measurement, they perform unambiguous state \emph{elimination} (USE) measurement. If the measurement succeeds, rather than being able to say definitively which state was received, a recipient can say with certainty which state was \emph{not} received. This measurement scheme is described further in Fig.~\ref{fig:USE}. The key advantage of the USE measurement scheme is that the probability that the measurement fails is significantly smaller than for USD, and so the resulting QDS scheme gains a boost in efficiency. Indeed, if USE eliminates $N-1$ of $N$ possible states then one knows with certainty which state was sent, USD may be viewed as a special case of the more general USE measurement. The shift from state discrimination to state elimination allows for much greater efficiency in QDS schemes. \MT{Make sure to talk about it later in the context of our QDS - make a graph showing $g$ (or $L$?) under discrimination vs elimination.}

\begin{figure}[htp]
\centering
\includegraphics[width=0.8\linewidth]{USE.png}
\caption{\label{fig:USE}}
\end{figure}

Collins \emph{et. al.} estimate a signature length $L = 5.1 \times 10^{13}$ in order to sign a message. Notice the subtle shift between Refs.~\cite{Gottesman2001} and \cite{Andersson2006, Clarke2012, Dunjko2014, Collins2014}. While previously the number of recipients did not need to be determined until the Messaging stage, here it must be determined before Distribution. After the coherent states have passed through the multiport the number of recipients cannot be changed. 
\MT{I should note later that removing the multiport removes this restriction.} Because of the physical requirements for the optical multiport, it will also be challenging (though possible) to generalize to more recipients, though note that for more than two recipients in Ref.~\cite{Dunjko2014} one may not use $USD$ measurement. \MT{why?}. Realistic implementation of the multiport also introduces noise and losses due to misalignment and instability, further reducing the efficiency of the protocol, and requires Bob and Charlie to be physically connected. In Ref.~\cite{Clarke2012, Collins2014}, for example Bob and Charlie are separated by $5$~m of optical fibre. 

The most difficult assumption which Refs.~\cite{Clarke2012, Dunjko2014, Collins2014} make, however, is that there should be no eavesdroppers on the quantum channels. This is a strong and impractical assumption, and one which subsequent papers endeavour to remove.

\subsection*{Removing multiport}
%\subsection{Wallden2015 (+ implementation)}

The fact that the QDS schemes discussed above require dedicated hardware at the receivers--the optical multiport--makes implementation in real-world situations difficult. The multiport introduces losses and noise, and requires tricky synchronisation between Bob and Charlie in order to correctly interfere the states. The experiment in Ref.~\cite{Collins2014} therefore has Bob and Charlie only separated by $5$~m optical fibre. 

To combat this, Wallden \emph{et. al.} \cite{Wallden2015} propose two QDS schemes specifically designed to run over the same hardware platform as QKD. In particular, they get rid of the multiport which was previously use to symmetrize Bob and Charlie's reduced output states. Their key insight is that rather than symmetrizing their states, it is sufficient to symmetrize their measurement outcomes. Therefore, a step is added to the distribution stage in which Bob and Charlie randomly swap half of their measurement outcomes over a secure classical channel. If Alice can gain no information about which outcomes were swapped then she cannot repudiate. 

This protocol was implemented by Donaldson \emph{et. al.} in Ref.~\cite{Donaldson2016} in which a message is securely signed over distances $500$~m, $1000$~m, and $2000$~m, with no requirement on the physical separation between Bob and Charlie. The secure classical link may be realised via QKD, and so Refs.~\cite{Wallden2015, Donaldson2016} begin to explore the close connections between these two different quantum communication protocols. We explore this further in Chapter.~\MT{X}. Donaldson \emph{et. al.} achieve signature length $L = 1.93 \times 10^9$ using QPSK coherent states and USE measurement. This is a vast improvement over the $L = 5 \times 10^{13}$ required in Ref.~\cite{Collins2014} and means that secure quantum signatures may actually be both useful and practical. 

\subsection*{Allowing Eve}
%\subsection{Amiri2016 (+ implementation)}
All signature schemes considered so far have made the assumption that the quantum distribution channels are secure, that is, they may not be attacked or monitored by an eavesdropper, Eve. This is clearly an unrealistic and unphysical assumption, but was a sensible one while the pressing impracticalities of early QDS schemes (quantum memory, multiport, tricky state comparison tests) were overcome. The emphasis in earlier papers was on dishonesty internal to the protocol, i.e. which attacks can Bob or Charlie mount when they already hold perfect copies of the quantum public keys. However, in a realistic scenario it is clear that an eavesdropper \emph{could} attack the quantum channels as states are being distributed, and so it is important to consider whether this has any effect on QDS security.

Amiri \emph{et. al.} provide a QDS scheme which allows for an Eve to eavesdrop on the quantum channels. In the worst-case scenario it is assumed that Eve will conspire with a dishonest internal player (Bob or Charlie in the case of a forging attack), and so knowledge which Bob/Charlie hold about their own quantum public key measurements is supplemented by knowledge learned through Eve's attack. For short, we describe Bob or Charlie as performing the eavesdropping attack, so as not to confuse the nomenclature. 

The key modification which Ref.~\cite{Amiri2016} makes is to have Alice use \emph{different} private keys (and so different sequences of quantum coherent states) for each recipient. This means that the dishonest recipient is forced to eavesdrop on the honest recipient's quantum channel if he is to gain any information. This is in contrast to earlier protocols in which the dishonest recipient held a perfect copy of the quantum public key, which was identical to that of the honest recipient. 

Because the dishonest player is forced to eavesdrop, he in fact receives a worst copy of the honest player's public key than in the above protocols, and so somewhat counter-intuitively Ref.~\cite{Amiri2016} requires \emph{shorter} $L$ than previous protocols, despite relaxing a security assumption. The protocol relies on sending weak attenuated coherent states, identically to decoy-state BB$84$ \MT{cite}, and \MT{X} detection at the receiver. Because the security of discrete-variable QKD is so advanced, the QDS protocol proposed in Ref.~\cite{Amiri2016} is secure against coherent eavesdropping attacks.\footnote{We will discuss the hierarchy of eavesdropping attacks in Sec.~\MT{X}}

\subsection{Side-channel attacks}
%\subsection{Puthoor2016 (+ implementation)}
%\MT{Though first talk about side-channel attacks}
It should be noted that "security" of a protocol is a theoretical statement, and not a physical one. A protocol is secure with respect to a model of how it operates in the real world, and whether a so-called unconditionally secure protocol can be broken in practice depends on how realistic or practical its underling modelling assumptions are. For example, although in many QKD protocols Eve is allowed to attack the quantum channels and eavesdrop on all communication, she is prevented from attacking the physical devices which are used to implement the protocol. 

For example, the QDS scheme presented in Ref.~\cite{Amiri2016} relies on distribution and detection of quasi-single photons \MT{make sure I have discussed this earlier} in different polarization bases. Physically, Alice must prepare her states before they are sent. A realistic Eve could attack Alice's device in order to gain information about the polarization of the prepared state and so she might gain enough information to forge without detection, even though the protocol is unconditionally secure against conventional types of eavesdropping attack.

An example of such a side-channel attack \MT{define what is a side-channel attack} is the trojan horse attack presented in Ref.~\cite{Jain2014}. Here, Eve shines a bright laser pulse into Alice's device and measures the few back-reflected photons which are scattered back. These photons have picked up the same polarization as Alice imparted to her prepared state, and so Eve is able to infer the chosen polarization basis choice, which gives her an undetected advantage.

To guard against side-channel attacks, honest parties have several options. One direction is to close known side-channels by additional protocol steps or additional hardware. For example, with the trojan horse attack Alice and Bob could add additional filters to their devices to block out light at Eve's required wavelength. It was shown however that Eve can bypass this by breaking Alice's filters, in a way that is undetectable to honest players \MT{cite the laser damage paper}. Closing side-channels in this way may open up the protocols to additional attack methods, which must then be understood, modelled and reacted-to. This places quantum cryptography into the same "cat-and-mouse" development cycles as conventional cryptography. \MT{expand on this, and make sure I have discussed it earlier in the historical overview.}

To break this cycle, and to provide genuinely unconditional security, guaranteed against all conceivable side-channel attacks, there has been a recent push towards device-independent cryptography. The security of device-independent (DI) protocols makes no trust assumptions about the devices used and it may even be assumed that the  devices are held by the malevolent party. DI cryptography is then based entirely on laws of quantum mechanics, specifically on the violation of a bell inequality. \MT{cite some stuff, and expand this paragraph.}

Full DI cryptography, while secure, is difficult to perform and may offer figures of merit which are too pessimistic for the desired application. One may compromise, then, and instead implement measurement device independent (MDI) cryptographic protocols, in which no trust assumptions are placed on the measurement devices (and they can even be owned by Eve), while the state-preparation and sending devices are held by honest parties and are trusted. \MT{cite some papers}

The first MDI QDS scheme is presented by Puthoor \emph{et. al.} in Ref.~\cite{Puthoor2016}.



\subsection{Tokyo installed fibers scheme}
\MT{Perhaps the rest of the DPS-based protocols, or installed-fiber protocols here too?}





\subsection{The other "almost-agile" ones?}

\subsection{An2019 (+ implementation)}

\subsection{Croal2016}
\MT{Discuss DV vs CV first}

\subsection{Quick chat about my PRA}
\MT{perhaps I just want to mention this in the "abstract"/introduction of this chapter? I don't want to talk about it too much because it is what the rest of the chapter is about.}


\subsection{"Classical" unconditionally secure signatures}
\MT{Talk about protocol P2 from Wallden2015, and also Amiri's scheme}

\subsection{Extensions to signature schemes}



\section{How to share a secret}\label{sec:qss_lit_review}
A secret sharing scheme allows for secure splitting and distribution of classical information among multiple recipients, an unknown subset of whom may be dishonest. The the canonical example of such a scheme is that of a bank. The head of the bank, Alice, wishes to distribute keys to the vault between several potentially untrusted deputies. If the deputies work together and use their keys simultaneously they are able to access the vault, but any nefarious deputies working alone should not be able to gain access.


%\MT{Make sure to cite the paper "how to share a secret" and talk about its title}.


\subsection{Classical secret sharing}\label{sec:qss_qcss}
Although many existing classical secret-sharing schemes are already information-theoretically secure \cite{Shamir1976, Blakley1979}, they may encounter problems when distributing  shares of the secret across insecure channels. This is analogous to the classical unconditionally secure signature schemes \cite{Wallden2015, Amiri2015a} discussed in Sec.~\ref{sec:lit_review_qds}, which implicitly required an underlying QKD encryption. Thus we may ask whether it is more or less resource-efficient to first run pairwise QKD between players, or to run a ``direct''-QSS scheme without first distilling pairwise secret keys. We should expect interesting parallels between QSS and QKD, since intuitively they are very similar, both effectively performing encryption of classical messages.

Let us consider some examples. Alice wishes to share a secret, $m$, between $n$ players, such that any $k \le n$ of them can access $m$. The general framework for this is called an $\left(n, k\right)$-threshold scheme, where of the $n$ players any subset of $k$ players can reconstruct the secret. An information-theoretically secure threshold sharing scheme was designed by Shamir in Ref.~\cite{Shamir1976}. Shamir's scheme relies on polynomial equations over finite fields, and is provably secure even against an adversary with infinite computing power. 

For example, Alice wishes to distribute a secret $m$ between four players, such that any three of them can access $m$. Alice generates a prime number $p$, and the polynomial 
\begin{equation}
\left( a x^2 + b x + m \right) \qq{modulo} p.
\end{equation}
Prime $p$ should be chosen larger than any of the coefficients $a, b$ or $m$. Alice then evaluates this polynomial at $4$ different points $x$, and sends the outcomes to each player. These points will be referred to as ``shares". %https://crypto.stackexchange.com/questions/9295/why-does-shamirs-secret-sharing-scheme-need-a-finite-field explains why a finite field is needed.
% It does need to be a field (integers aren't). Also it being finite means we can actually do the protocol.

The polynomial has three unknown coefficients, $a$, $b$ and $m$, and so any three players can combine their shares to create three equations, which may be solved for each unknown. Any fewer points will yield an underdefined system which cannot be solved. An attempt to guess the final share will show that any message $m$ can be the secret, and so such a guessing attempt is useless.

Another threshold secret sharing scheme was built on similar principles by Blakley \cite{Blakley1979}. In this scheme, the message $m$ is defined as a point in a large $k$ dimensional space. Each share is then a hyperplane in a $k-1$~dimensional space, which includes the point $m$. It therefore requires the intersection of all $k$ hyperplanes to reveal $m$. For example, if Alice again wishes to share a secret between four players, such that three of them are able to access $m$, then each share is a two-dimensional plane. The intersection of any two planes is a one-dimensional line containing $m$, and the third plane is required to reduce this line to the point $m$.

While both of these schemes are information-theoretically secure once the shares have been distributed (assuming that each share is securely stored and cannot be stolen), the main issue arises when considering how the shares can be distributed in the first place. If a malevolent party can access the shares during distribution then they can reconstruct the secret. In implementation, Shamir's and Blakley's schemes are therefore only as secure as the underlying encryption which is used to share the shares.

%\subsection*{Early quantum secret sharing}
%\MT{Gottesman1999, Karlsson1999, Hillery1999, Cleve1999}
%
%\MT{distinguish between secret sharing and state sharing}

\subsection{Quantum secret sharing}
We therefore wish to investigate whether the task of secret sharing can be made secure using quantum resources. It is important to notice that the translation from classical secret sharing to quantum secret sharing is not straightforward, and there are at least three directions which one can pursue:

\begin{itemize}
\item quantum-assisted classical secret sharing (qCSS): encrypt a classical secret sharing protocol \cite{Shamir1976, Blakley1979} using quantum resources. For example, perform pairwise QKD between Alice and each recipient, then encrypt the shares of the classical secret sharing protocol. This is analogous to the classical unconditionally secure schemes discussed earlier.
\item quantum secret sharing (QSS): use quantum states to securely distribute shares of a classical secret.
\item quantum state sharing (QStS): securely distribute shares of a quantum state.
\end{itemize}

Quantum state sharing is an important and exciting research direction in its own right and helps to establish the close links between quantum secret sharing, QKD and quantum teleportation \cite{Braunstein1998, Hillery1999, Markham2008a}. Despite the fact that both QSS and QStS are natural extensions of classical secret sharing to the quantum realm, and despite the fact that early work \cite{Hillery1999} proposes related protocols for each task, it should be understood that they are distinct quantum tasks with different goals and hardware requirements. For the rest of this Thesis we will restrict ourselves to QSS. In what follows we will only refer to the first two options as ``quantum secret sharing'', while the third option we shall refer to as ``quantum state sharing''.

\subsection{Entanglement-based QSS}


All three directions, qCSS, QSS and QStS are discussed at length in the pioneering work by Hillery \emph{et. al.} \cite{Hillery1999}. They propose the use of a GHZ resource state
\begin{equation}
\ket{GHZ} = \frac{1}{\sqrt{2}} \left(\ket{000} + \ket{111} \right)
\end{equation}
shared between three players, which can be used to distribute shares of a classical secret. Collaborating recipients can recover the secret while a dishonest subset of players cannot. Alternatively, the GHZ resource state may be used to distribute shares of a quantum state, such that collaborating players may reconstruct the original quantum state while a dishonest subset of players can gain no information.

For QSS, each player chooses independently and at random to measure their state in either $x$ or $y$ basis:
\begin{align}
&\ket{\pm x} = \frac{1}{\sqrt{2}} \left( \ket{0} \pm \ket{1}\right), \notag \\
%
&\ket{\pm y} = \frac{1}{\sqrt{2}} \left( \ket{0} \pm i \ket{1} \right). \notag
\end{align}
If, for example, all three players measure in the $x$ basis, then Charlie can infer from his measurement outcome whether Alice and Bob's measurements are correlated or anticorrelated. By collaborating, then, Bob and Charlie can accurately and securely infer Alice's bit. In fact, whenever Alice and Bob measure in the same basis as each other, Charlie must measure in $x$ in order to gain information. Conversely, if Alice and Bob measure in opposite bases then Charlie must measure in $y$, otherwise he gains no information. We see, then, that since each player randomly chooses which basis to measure, $50\%$ of the resource GHZ states will yield no information, and are effectively discarded.

Despite its high resource requirement, and despite the fact that $50\%$ of the resource states are wasted, Hillery's protocol has influenced the direction of all subsequent QSS protocols, and the paper was instrumental in demonstrating that multipartite entanglement is an important resource for quantum communication protocols. 
%\MT{Should I talk somewhere about qCSS in HBB paper?}
Multipartite entanglement is difficult to create and manipulate, and will degrade quickly as it is distributed over a quantum channel exposed to realistic loss or noise levels. Just as QKD has an equivalence between entanglement-based and prepare-and-measure versions \cite{Grosshans2003, Laudenbach2017}, it should be expected that the requirement of large multipartite state in Ref.~\cite{Hillery1999} can likewise be reduced \cite{Karlsson1999, Tittel2001, Zhang2005b, Williams2019}. 

To accomplish this, Karlsson \emph{et. al.} \cite{Karlsson1999} propose an entanglement-based QSS scheme which, rather than relying on creation and distribution of the GHZ state, relies on distribution of \emph{pairs} of entangled qubits in a Bell state. This configuration allows for correlations between players to be established identically to Hillery's scheme, but with more readily accessible resources. Recipients Bob and Charlie can determine with certainty which Bell state Alice sent, which allows Alice to establish a key with Bob/Charlie, and which may subsequently be used to encrypt a message. %\MT{demonstrate that it can give the same measurement outcomes as HBB with GHZ.}

This protocol drastically reduces the requirements of practical QSS, but the resulting protocol is still tricky to implement. The protocol requires Bell states and superpositions of Bell states, which will degrade over a realistic channel.  The protocol also introduces a fundamental asymmetry into QSS at the quantum level. While in Hillery's protocol protocol any of the three players can be chosen as dealer, for Ref.~\cite{Karlsson1999} it is established at the time of quantum state distribution that Alice is dealer. % which may make the protocol require bespoke hardware.

Both the protocols from Hillery \cite{Hillery1999} and Karlsson \cite{Karlsson1999} assume perfect state creation and noiseless and lossless quantum channels. This is an unrealistic assumption and one which must be relaxed before entanglement-based QSS can be implemented securely. 

Chen \emph{et. al.} \cite{Chen2005a} modify the Hillery's protocol \cite{Hillery1999} to allow for an imperfect distribution of entangled state. By proposing a method for entanglement distillation on a multipartite state, which can be used before a cryptographic protocol, Chen effectively reduces the extreme resource requirement of protocols like Ref.~\cite{Hillery1999}. The resource state does not even need to violate a Bell inequality.

An important generalization of the Hillery's scheme allows for analysis of the optimal entangled states required to share a secret between more than three players. While one option would be to simply replace the resource state with the N-partite GHZ state

\begin{equation}
\ket{N-GHZ} = \frac{1}{\sqrt{2}}\left(\ket{000\dots0} + \ket{111\dots1} \right)
\end{equation}

\noindent another option is to generalize to graph states \cite{Markham2008a, Keet2010} or in the continuous-variable regime Refs.~\cite{Lau2013, Wu2016} under which the tasks qCSS, QSS, QStS and entanglement-based QKD may be united and described within the same framework. %A graph state is a special type of multipartite entangled state.
One advantage of using such a state is that it can allow for QSS to be completed without collaboration from all recipients, which may help practical QSS to be robust and prevent against denial-of-service attacks from a dishonest internal player\footnote{Though we note that even QKD is susceptible to denial-of-service attack where Eve simply destroys the quantum (or classical) channels between Alice and Bob.}.

There have been several attempts to prove security of entanglement-based QSS. As we have seen, security proofs based on highly-entangled GHZ states or graph states become insecure once realistic channel parameters are considered, even though they offer unconditional security in the ideal limit. One way to tackle this is to borrow tools from entanglement-based QKD. Kogias \emph{et. al.} use similar analysis to so-called one-sided device-independent ($1$sDI) QKD \cite{Walk2016, Armstrong2015} in order to prove QSS security while modelling channel effects on their CV resource state.

Key to Kogias' protocol is the assumption that neither the measurement device of Bob, nor of Charlie, should be trusted. Rather, each player is assumed to possess a black-box which can output one of two measurement outcomes, corresponding in the honest case to homodyne measurement in either $x$ or $p$ quadrature. Protocol security is based on monogamy of entanglement and employs an entropic uncertainty relation which makes no assumption about the action of a dishonest player.. To our knowledge Ref.~\cite{Kogias2017} marked the first full security proof of QSS, against all forms of dishonesty and all types of attack over realistic channels. It was later shown that the resource required for entanglement-based QSS is two-way steering of the shared state \cite{Xiang2017, Xiang2018}, where the optimal Gaussian resource states for a given energy were also considered. 

The links between QSS and $1$sDI QKD explored in Ref.~\cite{Kogias2017} hint at an interesting direction for exploration: what is the relationship between QSS and other quantum communication protocols? It was already shown in Ref.~\cite{Markham2008a} that qCSS, QSS and QStS may be united under the same framework using graph states, while even in Hillery's oroiginal work \cite{Hillery1999} the links between qCSS and QSS were acknowledged. Additionally it can be shown \cite{Hillery1999} that a QStS protocol may be readily constructed from a teleportation protocol plus QSS (or qCSS or QKD) scheme if Alice teleports a quantum state to Bob, but sends the classical information required for state reconstruction to Charlie.

There are strong links between QSS and quantum conferencing \cite{Wu2016, Ottaviani2017b} which is a natural multipartite generalization of QKD in which $N$ players receive identical keys. Indeed, as shown in Refs.~\cite{Wu2016, Ottaviani2017b} the same resource states and network configurations may be readily used for both QSS and quantum conferencing. It is an open question however whether these additional tasks have the same optimal requirements \cite{Kogias2017, Xiang2017} on the resource state as QSS. %, or whether the optimal resource state for one protocol remains optimal for another protocol. 


%\MT{Add some more detail to this section. Add some examples of states and the transformations on them, and how they are used for QSS. Add some pictures too.}

%\MT{Add some chat about experimental implementations of EBQSS.}

%\MT{Still got some papers I need to talk about.}

\subsection{Sequential QSS}
The above protocols which implement QSS using entangled resource states offer an advanced level of security and neatly demonstrate the important role of entanglement in quantum communication. However, it is hard to see how they will be preferable to qCSS which can offer equivalent levels of security for the same task, but without the problems associated with generation and distribution of large entangled states. An entanglement-based scheme may even be fine if the number of players is small -- for example the scheme \cite{Karlsson1999} relyies only on Bell-pairs -- they cannot be easily scaled to many parties. We note that qCSS scales much more favourably as the number of required quantum channels is linear in the total number of players.

It should still be explored whether there are any QSS protocols which outperform qCSS. One promising direction is that of sequential\footnote{Sometimes referred to as entanglement-free QSS.} QSS in which the QSS task is fulfilled by sharing of a single quantum system between multiple players.

In the first sequential QSS protocol \cite{Zhang2005}, Zhang \emph{et. al.} propose a system in which Bob prepares a single photon state with his choice of polarization and sends it to Charlie. Charlie performs a unitary operation, either the identity, a Hadamard gate or a bit-flip, on the photon and sends it to Alice who stores the photon in a quantum memory. This process is repeated many times. Later, Alice will sample some of her stored photons for errors by asking Bob and Charlie to declare which state was sent and which operation was performed. She then performs the claimed operation, and measures the claimed basis, in order to check for errors.

On the remaining photons Alice performs her unitaries (either the identity or a bit-flip) to encode her secret. She then sends the photons back to Charlie. %\MT{how does the rest of the protocol run?}
If Bob and Charlie collaborate they can deduce the correct basis in which to measure Alice's photon, and so recover her information.



Sequential protocols have the obvious advantage that large entangled states are not required. Even though Ref.~\cite{Zhang2005} proposes to use a quantum memory it is ultimately not necessary for the protocol, and the work by Schmid \emph{et. al.} demonstrates this in a sequential QSS experiment \cite{Schmid2005}. Their experiment, in which players perform operations on heralded single photons, allows for a secret to be shared among six players in a setup which is much more readily scalable to more players than the earlier QSS schemes requiring entanglement.

\MT{Describe and appraise Schmid's protocol.}

Schmid's scheme relies on sequential interactions with a qubit state encoded into the polarization of a heralded single photon. Each player imposes a randomly chosen phase onto the state
\begin{equation}
\frac{\ket{0} + \ket{1}}{\sqrt{2}} \rightarrow \frac{\ket{0} + e^{i \phi_k} \ket{1}}{\sqrt{2}}
\end{equation}
and so at the end of the distribution the state is
\begin{equation}
\frac{\ket{0} + \exp\left(i \sum_{k}^N \phi_k\right)\ket{1}}{\sqrt{2}}.
\end{equation}
The final player measures in the $\ket{0} \pm \ket{1}$ basis. Collaboration of the first $N-1$ players allows them to infer, with certainty, the $N^{\text{th}}$ player's outcome.


%Just as prepare-and-measure QKD allows Alice and Bob to mimic the measurement outcomes of a shared entangled state  under \MT{criterion} the scheme \cite{Zhang2005} allows players to receive the same measurement outcomes they would if they had shared a GHZ state. Secret sharing then may proceed in the usual way. \MT{talk about this.}

The sequential scheme, while secure against an Eve external to the protocol, is difficult to secure against dishonesty from one of the internal players \cite{Deng2005, Qin2006, He2007}. For example, Ref.~\cite{He2007} points out that the order in which recipients declare their information is of utmost importance, and this is adopted into the sequential protocol in Ref.~\cite{Schmid2007}. Unfortunately, the protocol remains insecure against a so-called Trojan Horse attack \cite{Deng2005}, in which an internal player to the protocol adds one mode of an entangled state to the single photon as it is being distributed. This entangled mode will undergo the same subsequent gates as the signal photon, and so the dishonest player gains additional information. 

The Trojan Horse attack is guarded against in the recent work from Grice \emph{et. al.} \cite{Grice2019}. In their protocol for sequential QSS, each player creates a coherent state which is chosen from a Gaussian modulation. These states are added to the initial coherent state as it travels, and the final state is heterodyned by the dealer, Alice. With combined knowledge of their injected states, the players are able to estimate Alice's measurement outcome. This protocol has the advantage of high tolerable losses, especially when compared to entanglement-based QSS. Crucially, the scheme is immune to Trojan Horse attacks since once a coherent state has been added to the total state, it only interacts with Alice's measurement apparatus and does not pass through the equipment of any other player. Additionally a dishonest player cannot access other players' devices.

Owing to its simplicity of implementation QSS has been performed in many experiments \cite{Schmid2005, Hai-Qiang2013a} including those explicitly using telecom fiber networks \cite{Bogdanski2009}. This latter work, Ref.~\cite{Bogdanski2009}, demonstrates QSS in two experiments between three players and four players using phase encoding of single qubits. Their implementation uses a Sagnac interferometer, with light travelling in two directions around a loop. The light is at standard telecom wavelengths $1550$~nm and channel lengths are between $50$ and $70$~km, rendering secure QSS eminently practical.


\subsection{Summary}
Quantum Secret Sharing has been an intense field of active research for the quantum communications community for the last two decades. Entanglement-based QSS boasts a high level of provable security against both internal and external dishonesty. While there have been some proof-of-principle demonstrations of these QSS schemes \cite{Gartner2007, Bell2014, Tittel2001, Chen2005b} the style of protocol is still far off routine and practical implementation.

In contrast, sequential QSS involving sharing of a single quantum system is much more practical for implementation, and has been demonstrated in realistic settings with many players \cite{Schmid2005, Bogdanski2009, Hai-Qiang2013a}. However, these protocols face difficulty against a dishonest player internal to the protocol, which is precisely the context which secret sharing should guard against. Moreover, even though these schemes do not require generation or distribution of entangled states, they still require a dedicated hardware setup in order to distribute the quantum state and perform sequential measurements. To our knowledge, the most plausible protocol in terms of both its security and practicality, that of Grice \emph{et. al.} \cite{Grice2019}, is yet to be implemented. 

It is therefore yet unclear whether these quantum secret sharing protocols will give an advantage over the quantum-mediated qCSS protocols which we have discussed in Sec.\ref{sec:qss_qcss}. The underlying quantum encryption algorithm, QKD, boasts advanced security proofs and intensely researched hardware, and any proposed QSS scheme must be benchmarked against a QKD-based classical protocol which performs the same task.


