\chapter{PhoG: Numerical methods}\label{appendix:phog_numerical_methods}

In this appendix we will briefly overview the numerical methods which are used in Part $2$ of this Thesis. The first two, Secs.~\ref{appendix:direct_integration},~\ref{appendix:monte_carlo} are standard methods for handling the Lindblad master equation with much written about them elsewhere \MT{cite}, and so we will discuss them only in passing. The final two methods, mean-field and linearization, Secs.~\ref{appendix:mean-field},~\ref{appendix:single_mode_linear},~\ref{appendix:multi_mode_linear}, are discussed at length in the main body of the Thesis, Sec.~\ref{sec:phog_multi_mode}, and so we will just reproduce the final systems of equations in this Appendix.

\section{Direct integration}\label{appendix:direct_integration}

The dynamics of a quantum system coupled to a reservoir is governed by the Lindblad equation
\begin{equation}\label{eqn:appendix_lindblad}
\ddt \rho = - i \left[ \hat{H}, \rho \right] + \sum_n \left(\hat{C}_n \rho \hat{C}_n^\dagger - \frac{1}{2} \hat{C}_n^\dagger \hat{C}_n \rho - \frac{1}{2} \rho \hat{C}_n^\dagger \hat{C}_n \right)
\end{equation}
where $\hat{H}$ acts only on $\rho$ and $\hat{C}_n$ are the collapse operators governing decay into the reservoir. Here we take $\hat{C}_n = \sqrt{\gamma_n} \hat{A}_n$ where $\hat{A}_n$ is an operator acting on $\rho$, and is the operator through which $\rho$ couples to the reservoir in the original system-reservoir Schr{\"o}dinger equation. The derivation of this Lindblad equation including requisite approximations is discussed extensively in many classic texts such as Refs.~\cite{Breuer2002, Carmichael1999}. 

There are many routes which one can take to solve Eq.~\ref{eqn:appendix_lindblad}. One such approach is to interpret $\rho$, $\hat{H}$ and $\hat{C}_n$ as matrices. Let our underlying Hilbert-space be denoted $\mathcal{H}$ and have dimension $\dims$. Then $\rho$, $\hat{H}$ and $\hat{C}_n$ each have dimension $\dims$ and may be interpreted as a matrices in $M_{\dims \times \dims}\left(\mathbb{C}\right)$. The Lindblad equation~\ref{eqn:appendix_lindblad} may be interpreted as a coupled system of $\dims^2$ first-order ODEs. This system can then be solved via an appropriate numerical method \MT{cite something - advanced engineering mathematics? Or numerical methods book?}, the efficiency and power of which will depend strongly on the choices of $\hat{H}$, $\hat{C}_n$ and initial condition $\rho\left(0\right)$.

In this Thesis we use the open-source QuTiP package\footnote{QuTiP version $4.4.1$; Numpy version $1.16.4$; Scipy version $1.3.1$; Cython version $0.29.13$; Matplotlib version $3.1.0$; Python version $3.7.4$} \cite{qutip2} in Python to perform such numerical solutions. Direct integration of the ODE system is performed using \code{qutip.mesolve} command, with the actual integration performed by \code{scipy.integrate.ode} \MT{cite}. 

On a standard home-use laptop\footnote{Intel(R) Core(TM) i$5-3230$M CPU $@ 2.60$~GHz; $8.00$~GB RAM} a single-mode system with $\dims = 35$, $\rho\left(0\right)$ a coherent state with $\alpha = 3.0$, decay rate $\gamma = 8.0$, collapse operator $\hat{a}$ and free Hamiltonian $\hat{H} = \omega \hat{a}^\dagger \hat{a}$ with $\omega = 1.0$ can be solved in $58.3$~ms $\pm 5.65$~ms\footnote{Timed via iPython \code{\%timeit} magic command.}. Setting $\dims = 50$ takes $195 \pm 9.8$~ms, $\dims = 200$ takes $2.76$~s $\pm 129$~ms.

However, using collapse operator $\ncl$ increases the computational power required, and $\dims = 10$ takes $341 \pm 77.2$~ms, $\dims = 35$ takes over $100$~s, and $\dims=100$ takes $18$~minutes. Larger $\dims$ are intractable. We see then that this approach is highly dependent on both the form of $\hat{H}, \hat{C}_n$ and the Hilbert-space size. Indeed, a Hilbert space size $\dims$ yields $\dims^2$ coupled ODEs to solve. 

The coherent state must be defined in QuTiP specifying the ``analytic" option to \code{qutip.coherent\_dm}, which ensures that $\dyad{\alpha}$ uses the expression Eq.~\ref{eqn:intro_coherent_state}. The default option ``operator" instead finds the eigenstate of collapse operator $\hat{a}$. As $\dims \rightarrow \infty$ the two forms of coherent state become equivalent, but for small $\dims$ they can differ significantly. We have found that the analytic form gives much more accurate behaviour in the parameter ranges considered. Since the coherent state lives on an infinite-dimensional Hilbert space, care must be taken to choose $\dims$ sufficiently large that no ill effects are introduced from the truncation to a finite-sized one. A good rule-of-thumb is to pick
\begin{equation*}
\dims \ge \lceil 2 \alpha^2 \rceil
\end{equation*}
which ensures that the coherent state Eq.~\ref{eqn:appendix_coherent_state_truncated} is correctly normalized and a good approximation of the full Eq.~\ref{eqn:intro_coherent_state}. \MT{add truncated coherent state}. In any case, in all simulations we include built-in checks of state normalization which indicate when $\dims$ must be increased.

\MT{mention about scaling with number of modes}

\MT{integrate this section into the thesis body. Which graphs and quantities did I calculate via direct integration?}

\section{Quantum Monte-Carlo}\label{appendix:monte_carlo}
We have seen that direct integration of Eq.~\ref{eqn:appendix_lindblad} requires a system of $\dims^2$ coupled ODEs to be simultaneously solved. This is doable in the limit of small $\dims$, but quickly becomes difficult as $\dims$ increases. An alternative approach does not solve a matrix differential equation, rather a vector one, and so instead scales with $\dims$.

We will briefly outline the quantum monte-carlo approach and then discuss its implementation and use in this Thesis.



\section{Linearized single-mode model}\label{appendix:single_mode_linear}
By applying the linearization approximations derived in Sec.~\ref{sec:linearization} to the system of coupled ODEs derived from single-mode Lindblad equation~\ref{eqn:phog_single_mode_deriv} we arrive at the following closed system of ODEs:


\begin{align}\label{eqn:expectations_linear_first}
%
%a
%
\partial_t\langle s_- \rangle &= c_1 \langle s_- \rangle + c_2 \left(\langle s_-^\dagger \rangle \langle s_-^2 \rangle + 2 \langle s_- \rangle \langle n_-\rangle - 2 \langle s_-^\dagger \rangle \langle s_- \rangle^2\right) \notag \\
%
&-\frac{\gamma_3}{2}\left(
6 \langle s_-^\dagger \rangle \langle n_- \rangle \langle s_-^2\rangle + 3 \langle s_- \rangle \langle s_-^{\dagger 2} \rangle \langle s_-^2 \rangle \right. \notag \\
%
&+ 6 \langle s_- \rangle \langle n_- \rangle^2 - 2 \langle s_-^{\dagger 2} \rangle \langle s_- \rangle^3 - 12 \langle n_- \rangle \langle s_-^\dagger \rangle \langle s_- \rangle^2 \notag \\
%
&\left. - 6 \langle s_-^2 \rangle \langle s_-^\dagger \rangle^2 \langle s_- \rangle + 6 \langle s_-^\dagger \rangle^2 \langle s_- \rangle^3\right)
\end{align}
\begin{align}
%
% ad
%
\partial_t\langle s_-^\dagger \rangle &= c_1^* \langle s_-^\dagger \rangle + c_2^* \left(\langle s_- \rangle \langle s_-^{\dagger 2} \rangle + 2 \langle s_-^\dagger \rangle \langle n_- \rangle - 2 \langle s_-^\dagger \rangle^2\langle s_- \rangle \right) \notag \\
%
&-\frac{\gamma_3}{2}\left(6 \langle s_- \rangle \langle n_- \rangle \langle s_-^{\dagger 2}\rangle + 3 \langle s_-^\dagger \rangle \langle s_-^2 \rangle \langle s_-^{\dagger 2}\rangle \notag \right. \notag \\
%
&+ 6 \langle s_-^\dagger \rangle \langle n_-\rangle^2 - 2 \langle s_-^2 \rangle \langle n_-^3\rangle - 12 \langle n_- \rangle \langle s_- \rangle \langle s_- \rangle^2 \notag \\
%
&\left. - 6 \langle s_-^{\dagger 2} \rangle \langle s_- \rangle^2 \langle s_-^\dagger \rangle + 6 \langle s_-^{\dagger 3} \rangle \langle s_- \rangle^2 \right)
\end{align}
\begin{align}
%
% aa
%
\partial_t\langle s_-^2 \rangle &= c_3 \langle s_- s_- \rangle + c_4 \left( 3 \langle n_- \rangle \langle s_-^2 \rangle  - 2 \langle s_-^\dagger \rangle \langle s_- \rangle^3 \right) \notag \\
%
&- \gamma_3 \left( 3 \langle s_-^{\dagger 2}\rangle \langle s_-^2\rangle^2 + 12 \langle n_- \rangle^2 \langle s_-^2\rangle - 2 \langle s_-^{\dagger 2}\rangle \langle s_- \rangle^4 \right. \notag \\
%
& - 12 \langle s_-^2 \rangle \langle s_-^{\dagger}\rangle^2 \langle s_- \rangle^2 - 16 \langle n_- \rangle \langle s_-^\dagger \rangle \langle s_- \rangle^3 \notag \\
%
&\left. + 16 \langle s_-^\dagger \rangle^2 \langle s_- \rangle^4 \right)
\end{align}
\begin{align}
%
% adad
%
\partial_t\langle s_-^{\dagger^2} \rangle &= c_3^* \langle s_-^{\dagger 2}\rangle + c_4^* \left(3 \langle s_-^{\dagger 2} \rangle \langle n_- \rangle - 2 \langle s_-^\dagger \rangle^3 \langle s_- \rangle \right) \notag \\
%
&- \gamma_3 \left(3 \langle s_-^{\dagger 2} \rangle^2 \langle s_-^2 \rangle + 12 \langle s_-^{\dagger 2} \rangle \langle n_- \rangle^2 - 2\langle s_-^2 \rangle \langle s_-^\dagger \rangle^4 \right. \notag \\
%
&\left. - 21 \langle s_-^{\dagger 2}\rangle \langle s_-^\dagger\rangle^2 \langle s_- \rangle^2 - 16 \langle n_- \rangle \langle s_-^\dagger \rangle^3 \langle s_-\rangle \right. \notag \\
%
& \left.+ 16 \langle s_-^\dagger\rangle^4 \langle s_- \rangle^2 \right)
\end{align}
\begin{align}\label{eqn:expectations_linear_last}
%
% ada
%
\partial_t \langle n_- \rangle &= - \gamma_1 \langle n_- \rangle + c_5 \left( \langle s_-^{\dagger 2} \rangle \langle s_-^2 \rangle + 2 \langle n_- \rangle^2 - 2 \langle s_-^\dagger \rangle^2 \langle s_- \rangle^2 \right) \notag \\
%
&- \gamma_3 \left( 9 \langle s_-^{\dagger 2} \rangle \langle n_-  \rangle \langle s_-^2\rangle + 6 \langle n_-\rangle^3  - 6 \langle s_-^{\dagger 2}\rangle \langle s_-^\dagger \rangle \langle s_- \rangle^3 \right. \notag \\
%
& - 18 \langle n_- \rangle \langle s_-^\dagger \rangle^2 \langle s_- \rangle^2 - 6 \langle s_-^2 \rangle \langle s_-^\dagger \rangle^3 \langle s_- \rangle \notag \\
%
& \left. + 16 \langle s_-^\dagger \rangle^3 \langle s_-\rangle^3 \right)
\end{align}
with $n_- = s_-^\dagger s_-$.

This system is solved numerically for $\langle s_- \rangle$, $\langle s_-^\dagger \rangle$, $\langle s_-^2\rangle$, $\langle s_-^{\dagger 2}\rangle$, $\langle n_-\rangle$ and the results are shown as dashed lines in Fig.~\ref{fig:phog_single_mode_linearization}.


\section{Linearized multi-mode model}\label{appendix:multi_mode_linear}
We will derive a linearized and closed system of coupled differential equations capable of modelling the multi-mode PhoG device. In fact, our equations will be capable of modelling any collection of coupled modes with on-site Kerr nonlinearity and Markovian reservoirs. Our starting Lindblad equation is Eq.~\ref{eqn:phog_multi_mode}

\begin{equation}
\ddt \rho = - i \left[\hat{H}, \rho\right] + \gamma_1 \left[\mathcal{L}\left(\hat{a}_1\right) + \mathcal{L}\left(\hat{a}_2\right) + \sum_{j=3}^{N+2} \mathcal{L}\left(\hat{a}_j\right) \right] \rho
\end{equation}
where we here take $\hat{H} = \hat{H}^{\text{Coupling}} + \hat{H}^{\text{Kerr}}$. In this Appendix we label all modes as $\hat{a}_k$, with the subscript denoting which mode in the PhoG device is meant. In particular, $a_1 \leftrightarrow a$, $a_2 \leftrightarrow b$, and $a_{k \ge 3} \leftrightarrow c_{k-3}$ in the main body. The Hamtiltonian is
\begin{align}
\hat{H}^{\text{Kerr}} = \frac{U}{2} \sum_{x} \hat{x}^\dagger \hat{x}^\dagger \hat{x} \hat{x} \qq{} x \in \left\{a, b, c_j\right\}\; 0 \le j \le N
%
\hat{H}^{\text{Coupling}} = \sum_{k, l} \mathcal{G}_{k, l}\left(\hat{a}_k^\dagger \hat{a}_l + \hat{a}_l^\dagger \hat{a}_k \right)
\end{align}
where we have introduced a "coupling matrix" $\mathcal{G}$ which contains all information relating to the linear coupling between modes of our system. Coupling matrix element $\mathcal{G}_{j, p}$ denotes the coupling strength between modes $j$ and $p$, and $\mathcal{G}_{j, p} = 0$ if the modes are not coupled to each other, which is the case for most pairs $\left(j, p\right)$. We display some example coupling matrices below.
\subsection{Example coupling matrices $\mathcal{G}$}
\begin{figure}[htp]
\begin{minipage}{.4\textwidth}
\[
\mathcal{G} = 
\begin{bmatrix}
0 & \highlight{g} & 0 & 0 & \dots & 0 & 0 & 0 \\
\highlight{g} & 0 & \highlight{g} & 0 & \dots & 0 & 0 & 0 \\
0 & \highlight{g} & 0 & \highlight{g} & \dots & 0 & 0 & 0  \\
0 & \vdots & \vdots & \vdots & \ddots & \vdots & \vdots & 0 \\
0 & 0 & 0 & 0 & \dots & 0 & \highlight{g} & 0 \\
0 & 0 & 0 & 0 & \dots & \highlight{g} & 0 & \highlight{g} \\
0 & 0 & 0 & 0 & \dots & 0 & \highlight{g} & 0
\end{bmatrix}
\]
\end{minipage}
\begin{minipage}{.5\textwidth}
\centering
\includegraphics[width=0.3\textwidth, draft=false]{phog/bosonic_line_vertical}
\end{minipage}
\caption{\label{fig:coupline}Line of modes}
\end{figure}



\begin{figure}[htp]
\begin{minipage}{.4\textwidth}
\[
\mathcal{G} = 
\begin{bmatrix}
0 & 0 & \highlight{g1} & 0 & 0 & 0 & \dots & 0 & 0 & 0 \\
0 & 0 & \highlight{g2} & 0 & 0 & 0 & \dots & 0 & 0 & 0 \\
\highlight{g1} & \highlight{g2} & 0 & \highlight{g3} & 0 & 0 & \dots & 0 & 0 & 0 \\
0 & 0 & \highlight{g3} & 0 & \highlight{g3} & 0 & \dots & 0 & 0 & 0 \\
0 & 0 & 0 & \highlight{g3} & 0 & \highlight{g3} & \dots & 0 & 0 & 0 \\
0 & 0 & 0 & 0 & \highlight{g3} & 0 & \dots & 0 & 0 & 0  \\
0 & \vdots & \vdots & \vdots & \vdots & \vdots & \ddots & \vdots & \vdots & 0 \\
0 & 0 & 0 & 0 & 0 & 0 & \dots & 0 & \highlight{g3} & 0 \\
0 & 0 & 0 & 0 & 0 & 0 & \dots & \highlight{g3} & 0 & \highlight{g3} \\
0 & 0 & 0 & 0 & 0 & 0 & \dots & 0 & \highlight{g3} & 0 
\end{bmatrix}
\]
\end{minipage}
\begin{subfigure}{.4\textwidth}
\centering
\includegraphics[width=0.3\textwidth, draft=false]{phog/bosonic_chain_phog_vertical}
\end{subfigure}
\caption{\label{fig:coupPhoG}PhoG system \MT{TODO: fix alignment}}
\end{figure}
\clearpage
\subsection{Linearized equations}
Letting $n, m \in \left[1, N+2\right], n \ne m$, we derive a closed system of coupled differential equations for first- and second-order expectations

\begin{align}\label{eqn:linearsystem}
% an
\partial_t\langle\hat{a}_n\rangle &= \left(- i \omega_n - \frac{\Gamma_n}{2}\right)\langle\hat{a}_n\rangle - 2 i U \langle\hat{a}_n\rangle\langle\hat{a}_n^\dagger\hat{a}_n\rangle - i U \langle\hat{a}_n^\dagger\rangle\langle\hat{a}_n\hat{a}_n\rangle + 2 i U \langle\hat{a}_n^\dagger\rangle \langle\hat{a}_n\rangle\langle\hat{a}_n\rangle - \sum_{j=1}^N i \mathcal{G}_{n, j} \langle\hat{a}_j\rangle\\
% adn
\partial_t\langle\hat{a}_n^\dagger\rangle &= \left(+ i \omega_n - \frac{\Gamma_n}{2} \right) \langle\hat{a}_n^\dagger\rangle + 2 i U \langle\hat{a}_n^\dagger\rangle\langle\hat{a}_n^\dagger\hat{a}_n\rangle + i U \langle\hat{a}_n\rangle\langle\hat{a}_n^\dagger\hat{a}_n^\dagger\rangle - 2 i U\langle\hat{a}_n^\dagger\rangle\langle\hat{a}_n^\dagger\rangle\langle\hat{a}_n\rangle + \sum_{j=1}^N i \mathcal{G}_{n,j}\langle\hat{a}_j^\dagger\rangle \\
% anan
\partial_t\langle\hat{a}_n\hat{a}_n\rangle &= \left( - 2 i \omega_n - \Gamma_n\right) \langle\hat{a}_n\hat{a}_n\rangle - 5 i U \langle\hat{a}_n^\dagger \hat{a}_n\rangle \langle\hat{a}_n\hat{a}_n\rangle + 4 i U \langle\hat{a}_n^\dagger\rangle \langle\hat{a}_n\rangle \langle\hat{a}_n\rangle\langle\hat{a}_n\rangle - i U \langle \hat{a}_1\hat{a}_1\rangle\langle\hat{a}_1\hat{a}_1^\dagger\rangle \notag \\
&- \sum_{j=1}^N 2 i \mathcal{G}_{n, j}\langle\hat{a}_n\hat{a}_j\rangle \\
% adnan
\partial_t \langle\hat{a}_n^\dagger\hat{a}_n\rangle &= - \Gamma_n\langle\hat{a}_n^\dagger\hat{a}_n\rangle + \Gamma_n \bar{n}_{th}^{\left(n\right)} + \sum_{j=1}^N i \mathcal{G}_{n, j}\left(\langle\hat{a}_j^\dagger\hat{a}_n\rangle - \langle\hat{a}_n^\dagger\hat{a}_j\rangle\right) \\
%anadn
\partial_t \langle\hat{a}_n \hat{a}_n^\dagger\rangle &= - \Gamma_n \langle\hat{a}_1\hat{a}_1^\dagger\rangle + \Gamma_n \bar{n}_{th}^{\left(n\right)} + \sum_{j=1}^N i \mathcal{G}_{n,j}\left(\langle\hat{a}_j^\dagger\hat{a}_n\rangle - \langle\hat{a}_n^\dagger\hat{a}_j\rangle \right)
\\
% adnadn
\partial_t\langle\hat{a}_n^\dagger\hat{a}_n^\dagger\rangle &= \left(2 i \omega_n - \Gamma_n  \right) \langle\hat{a}_n^\dagger\hat{a}_n^\dagger\rangle + 5 i U \langle\hat{a}_n^\dagger\hat{a}_n^\dagger\rangle\langle\hat{a}_n^\dagger \hat{a}_n\rangle - 4 i U \langle\hat{a}_n^\dagger\rangle\langle\hat{a}_n^\dagger\rangle\langle\hat{a}_n^\dagger\rangle\langle\hat{a}_n\rangle + i U \langle\hat{a}_1\hat{a}^\dagger_1\rangle \langle\hat{a}_1^\dagger\hat{a}_1^\dagger\rangle \notag \\
&+ \sum_{j=1}^N 2 i \mathcal{G}_{n, j} \langle\hat{a}_n^\dagger\hat{a}_j^\dagger\rangle \\
% anam
\partial_t\langle\hat{a}_n\hat{a}_m\rangle &= \left(i\left(\omega_n + \omega_m\right) - \frac{\Gamma_n + \Gamma_m}{2}\right)\langle\hat{a}_n\hat{a}_m\rangle - 2 i U \langle\hat{a}_n^\dagger\hat{a}_n\rangle \langle\hat{a}_n\hat{a}_m\rangle - i U \langle\hat{a}_n^\dagger\hat{a}_m\rangle \langle\hat{a}_n\hat{a}_n\rangle \notag \\
& - 2 i U \langle\hat{a}_m^\dagger\hat{a}_m\rangle\langle\hat{a}_n\hat{a}_m\rangle - i U \langle\hat{a}_m^\dagger\hat{a}_n\rangle\langle\hat{a}_m\hat{a}_m\rangle + 2 i U \langle\hat{a}_n^\dagger\rangle \langle\hat{a}_n\rangle \langle\hat{a}_n\rangle \langle\hat{a}_m\rangle + 2 i U \langle\hat{a}_n\rangle \langle\hat{a}_m^\dagger\rangle \langle\hat{a}_m\rangle\langle\hat{a}_m\rangle \notag \\
& - \sum_{j=1}^N i \mathcal{G}_{n, j}\langle\hat{a}_j\hat{a}_m\rangle - \sum_{q=1}^N i \mathcal{G}_{m, q} \langle\hat{a}_n\hat{a}_q\rangle \\
% adnam
\partial_t\langle\hat{a}_n^\dagger\hat{a}_m\rangle &= \left(i\left(\omega_n - \omega_m\right) - \frac{\Gamma_n + \Gamma_m}{2}\right)\langle\hat{a}_n^\dagger\hat{a}_m\rangle + 2 i U \langle\hat{a}_n^\dagger\hat{a}_m \rangle \langle\hat{a}_n^\dagger \hat{a}_n\rangle + i U \langle \hat{a}_n^\dagger \hat{a}_n^\dagger\rangle\langle\hat{a}_n\hat{a}_m\rangle \notag \\
& - 2 i U \langle\hat{a}_n^\dagger \hat{a}_m \rangle \langle \hat{a}_m^\dagger \hat{a}_m\rangle - i U \langle\hat{a}_n^\dagger \hat{a}_m^\dagger\rangle \langle\hat{a}_m\hat{a}_m\rangle - 2 i U \langle\hat{a}_n^\dagger\rangle \langle\hat{a}_n^\dagger\rangle \langle\hat{a}_n\rangle \langle\hat{a}_m\rangle + 2 i U \langle\hat{a}_n^\dagger \rangle \langle \hat{a}_m^\dagger \rangle \langle \hat{a}_m\rangle \langle\hat{a}_m\rangle \notag \\
& + \sum_{j=1}^N i \mathcal{G}_{n, j}\langle\hat{a}_j^\dagger\hat{a}_m\rangle - \sum_{q=1}^N i \mathcal{G}_{m, q}\langle \hat{a}_n^\dagger \hat{a}_q\rangle \\
% adnadm
\partial_t\langle\hat{a}_n^\dagger\hat{a}_m^\dagger\rangle &= \left(i \left(\omega_n + \omega_m\right) - \frac{\Gamma_n + \Gamma_m}{2}\right) \langle \hat{a}_n^\dagger \hat{a}_m^\dagger\rangle + 2 i U \langle\hat{a}_n^\dagger \hat{a}_m^\dagger\rangle \langle \hat{a}_n^\dagger \hat{a}_n\rangle + i U \langle \hat{a}_n^\dagger \hat{a}_n^\dagger \rangle \langle \hat{a}_m^\dagger \hat{a}_n\rangle \notag \\
&+ 2 i U \langle\hat{a}_n^\dagger \hat{a}_m^\dagger \rangle \langle \hat{a}_m^\dagger \hat{a}_m\rangle + i U \langle \hat{a}_n^\dagger \hat{a}_m\rangle \langle \hat{a}_m^\dagger \hat{a}_m^\dagger \rangle - 2 i U \langle\hat{a}_n^\dagger \rangle \langle \hat{a}_n^\dagger \rangle \langle \hat{a}_n \rangle \langle \hat{a}_m^\dagger\rangle - 2 i U \langle \hat{a}_n^\dagger \rangle \langle \hat{a}_m^\dagger \rangle \langle \hat{a}_m^\dagger \rangle \langle \hat{a}_m\rangle \notag \\
& + \sum_{j=1}^N i \mathcal{G}_{n, j} \langle\hat{a}_j^\dagger \hat{a}_m^\dagger\rangle + \sum_{q=1}^N i \mathcal{G}_{m, q} \langle\hat{a}_n^\dagger \hat{a}_q^\dagger\rangle.
\end{align} 
\MT{TODO: get rid of antinormal ordered guy. TODO: fix appearance}
\iffalse
\fi