\chapter{Quantum secret sharing}
Goal of chapter: introduce our QSS protocol and prove its security in different contexts.

Key results which I want to present:
\begin{itemize}
\item our QSS protocol
\item security proof (what does it do? what does it not do?)
\item analysis of security in various settings - heterodyne, $\mathcal{A}_4$, BS0, BS1, BS2, EC, varying $\alpha$, $T$, $\xi$, $g$, $h$ of both channels
\item show how the protocol performs
\end{itemize}

\MT{Short introduction to chapter. It can function a bit like an abstract}

%\section{How to share a secret}\label{sec:qss_lit_review}
\MT{introductory remarks}
Such schemes are regularly employed in \MT{context}, the canonical example being that of a bank. The head of the bank, Alice, wishes to distribute keys to its vault between several deputies, any unknown subset of whom may be dishonest. If the deputies work together and use their keys simultaneously they are able to access the vault, but any nefarious deputies working along should not be able to gain access.

\MT{Make sure to cite the paper "how to share a secret" and talk about its title}.


\subsection*{Classical secret sharing}
Although many existing secret-sharing schemes are already information-theoretically secure while relying only on classical resources \MT{cite}, they may encounter problems when distribution shares of the secret across insecure channels. \MT{talk in more detail about an information-theoretically secure classical scheme} This is analogous to the classical unconditionally secure signature schemes discussed in Sec.~\MT{X}. Secure implementation of a scheme such as \MT{X} requires shared secret keys which, in reality, requires QKD. Thus we may ask whether it is more or less resource-efficient to first run pairwise QKD between players, or to run a "direct"-QSS scheme without first distilling pairwise secret keys. We should expect interesting parallels between QSS and QKD, since intuitively they are very similar, and rely on encryption of classical messages.

Let us consider some examples. Alice wishes to share a secret $m$ between $n$ players, such that any $k$ of them can access $m$. The general framework for this is an $\left(n, k\right)$-threshold scheme, where there are $n$ players in total and any subset of $k$ players can reconstruct the secret. An example of an information-theoretically secure threshold sharing scheme was designed by Shamir in Ref.~\MT{X}. Shamir's scheme relies on polynomial equations and is provably secure even against an adversary with infinite computing power. 

For example, Alice wishes to distribute a secret $m$ between four players, such that any three of them can access $m$. Alice generates a prime number $p$, and the polynomial \MT{what does it mean to add $m$ to a polynomial? TODO: read Shamir's original paper. why do we need to take it modulo p?}
\begin{equation}
\left( a x^2 + b x + m \right) \text{modulo} p.
\end{equation}
Prime $p$ should be chosen larger than any of the coefficients $a, b$ or $m$. Alice then evaluates this polynomial at $4$ different points $x$, and sends the outcomes to each player. These points will be referred to as ``shares".

The polynomial has three unknown coefficients, $a$, $b$ and $m$, and so any three players can combine their shares to create three equations, which may be solved for each unknown. Any fewer points will yield an underdefined system which cannot be solved. An attempt to guess the final share will show that any such message $m$ can be the secret and so such a guessing attempt is useless.

Another analogous threshold secret sharing scheme was built on similar principles by Blakley \MT{cite}. In Blakley's scheme, the message $m$ is defined as a point in a large $k$ dimensional space. Each share is then a hyperplane in a $k-1$~dimensional space, which includes the point $m$. It therefore requires the intersection of all $k$ hyperplanes to reveal $m$. For example, if Alice again wishes to share a secret between four players, such that three of them are able to access $m$, then each share is a two-dimensional plane. The intersection of any two planes is a one-dimensional line containing $m$, and the third plane is required to reduce this line to the point $m$.

While both of these schemes are information-theoretically secure once the shares have been distributed (assuming that each share is securely stored and cannot be stolen), the main issue arises when considering how the shares can be distributed in the first place. If a malevolent party can access the shares while Alice distributes them, then they can reconstruct the secret. In implementation, Shamir's and Blakley's schemes are therefore only as secure as the underlying encryption used to share the shares.

%\subsection*{Early quantum secret sharing}
%\MT{Gottesman1999, Karlsson1999, Hillery1999, Cleve1999}
%
%\MT{distinguish between secret sharing and state sharing}

\subsection{Quantum secret sharing}
One might therefore wish to investigate whether the task of secret sharing can be made secure using quantum resources. It is important to notice that the translation from classical secret sharing to quantum secret sharing is not straightforward. There are at least three directions which one can pursue in order to perform a secret sharing task using quantum mechanics:

\begin{itemize}
\item quantum-assisted classical secret sharing (qCSS): encrypt a classical secret sharing protocol (e.g. Shamir \MT{cite}) using quantum resources. For example, perform pairwise QKD between Alice and each recipient, then encrypt the shares of the classical secret sharing protocol
\item quantum secret sharing (QSS): use quantum states to securely distribute shares of a classical secret
\item quantum state sharing (QStS): securely distribute shares of a quantum state
\end{itemize}

Quantum state sharing is an important and exciting research direction in its own right and helps to establish the close links between quantum secret sharing, QKD and quantum teleportation \MT{cite some stuff}. Despite the fact that both QSS and QStS are natural extensions of classical secret sharing to the quantum realm, , and despite the fact that early work \MT{cite} proposes related protocols for each task, it should be understood that the two are distinct quantum tasks with different goals and hardware requirements, so for the rest of this Thesis we will restrict ourselves to QSS. In what follows we will only refer to the first two options as quantum secret sharing, while the third option we shall refer to as quantum state sharing.

\subsection{Entanglement-based QSS}

%\MT{I can add a fuller explanation of HBB (and the other protocols), since I go into quite some depth in QDS lit review.}
All three directions are discussed at length in the pioneering work by Hillery \emph{et. al.} \cite{Hillery1999}. They propose the use of a GHZ resource state Eq.~\MT{X} shared between three players, which can be used to distribute shares of a classical secret such that collaborating recipients can recover the secret while a dishonest subset of players cannot. Alternatively, the GHZ resource state may be used to distribute shares of a quantum state, such that collaborating players may reconstruct the original quantum state while a dishonest subset of players can gain no information.

Each player chooses independently and at random to measure their state in either the $x$ or $y$ basis \MT{define these}. If for example \MT{give an example of the type of calculation that lets players recover the state. Just copy it from the HBB paper}.  Crucially, knowledge of the outcomes of two players allows one to infer the outcome of the third player.  \MT{talk about how this actually helps to distribute a secret.}

Despite its high resource requirement, and despite the fact that $50\%$ of the resource states are wasted \MT{why?} the HBB protocol has influenced the direction of all subsequent QSS protocols, and the paper was instrumental in demonstrating that multipartite entanglement may be utilized as an important resource for quantum communication protocols. 
\MT{Should I talk somewhere about qCSS in HBB paper?}

Multipartite entanglement is difficult to create and manipulate, and will degrade quickly as it is distributed over a quantum channel exposed to realistic loss or noise levels. Just as QKD has an equivalence between entanglement-based and prepare-and-measure versions \MT{Talk about ekert and BB92?}, it should be expected that the requirement of large multipartite state in Ref.~\cite{Hillery1999} can likewise be reduced \cite{Karlsson1999, Tittel2001, Zhang2005b, Williams2019}. Karlsson \emph{et. al.} \cite{Karlsson1999} propose an entanglement-based QSS scheme which, rather than relying on creation and distribution of the GHZ state, relies on distribution of \emph{pairs} of entangled qubits in a Bell state. 

This configuration allows for correlations between players to be established identically to HBB with more readily accessible resources. Recipients Bob and Charlie can determine with certainty which Bell state Alice sent, which allows Alice to establish a key with Bob/Charlie, which may subsequently be used to encrypt a message. \MT{demonstrate that it can give the same measurement outcomes as HBB with GHZ.}

This protocol drastically reduces the requirements of practical QSS, but the resulting protocol is still non-trivial to implement. The protocol requires Bell states and superpositions of Bell states which will all be degraded over a realistic channel. 

These protocols also introduces a fundamental asymmetry into QSS at the quantum level. While in the HBB protocol any of the three players can be chosen as dealer \MT{check this}, for Ref.~\cite{Karlsson1999} it is established at the time of quantum state distribution that Alice is dealer, which may make the protocol require bespoke hardware.

Both of these protocols \cite{Hillery1999, Karlsson1999} assume both perfect state creation and noiseless and lossless quantum channels. This is an unrealistic assumption and one which must be relaxed before entanglement-based QSS can be implemented securely. Chen \emph{et. al.} \cite{Chen2005a} modify the HBB protocol to the case when the resource state is a noisy-GHZ state. By proposing a method for distillation of multipartite entangled states which may be used before a communication protocol--such as QSS or quantum conferencing--requiring multiparite entanglement as a resource, their entire protocol allows for successful QSS even when the resource state does not violate a Bell inequality.


An important generalization of the HBB scheme allows for analysis of some of the optimal entangled states required to share a secret between more than three players. While one option would be to simply replace the resource state with the N-partite GHZ state

\MT{insert equation}
\MT{problems with npartite GHZ state}

\noindent another option is to generalize to graph states \cite{Markham2008a, Keet2010} or continuous-variable cluster states \cite{Lau2013, Wu2016} under which the tasks qCSS, QSS, QStS and entanglement-based QKD may be united and described within the same framework. A graph state is \MT{add a description}. One advantage of using such a state is that it can allow fro QSS to be completed without collaboration from all recipients, which may help practical QSS to be robust and prevent against denial-of-service attacks from a dishonest internal player\footnote{Though we note that even QKD is susceptible to denial-of-service attack where Eve simply destroys the quantum (or classical) channels between Alice and Bob.}

There have been several attempts to prove security of entanglement-based QSS. As we have seen, security proofs based on highly-entangled GHZ states or graph states become insecure once realistic channel parameters are considered, even though they offer unconditional security in the ideal limit. One way to tackle this is to borrow tools from entanglement-based QKD. Kogias \emph{et. al.} use similar analysis to so-called one-sided device-independent ($1$sDI) QKD \cite{Armstrong2015} \MT{check it is this paper} in order to prove QSS security while modelling channel effects on their CV resource state.

Key to Kogias' protocol is the assumption that neither the measurement device of Bob nor of Charlie should be trusted. Rather, each player is assumed to possess a black-box which can output one of two measurement outcomes, corresponding in the honest case to homodyne measurement in either $x$ or $p$ quadrature. Protocol security is based on monogamy of entanglement \MT{cite} and employs an entropic uncertainty relation which makes no assumption about the action of a dishonest player, by analogy with $1$sDI QKD. To our knowledge Ref.~\cite{Kogias2017} marked the first full security proof of QSS. It was later shown that the resource required for entanglement-based QSS is two-way steering of the shared state \cite{Xiang2017, Xiang2018}, and the optimal Gaussian resource states for a given energy were also considered. 

The links between QSS and $1$sDI QKD explored in Ref.~\cite{Kogias2017} hint at an interesting direction for exploration: what is the relationship between QSS and other quantum communication protocols? It was already shown in Ref.~\cite{Markham2008a} that qCSS, QSS and QStS may be united under the same framework using graph states, while even in the original HBB work \cite{Hillery1999} the links between qCSS (classical secret sharing + QKD) and QSS were acknowledged. Additionally it can be shown \MT{cite} that a QStS protocol may be readily constructed from a teleportation protocol plus QSS (or qCSS or QKD) scheme if Alice teleports a quantum state to Bob, but sends the classical information required for state reconstruction to Charlie.

There are strong links between QSS and quantum conferencing \cite{Wu2016, Ottaviani2017b} which is a natural multipartite generalization of QKD in which $N$ players receive identical keys. Indeed, as shown in Refs.~\cite{Wu2016, Ottaviani2017b} the same resource states and network configurations may be readily used for both QSS and quantum conferencing. It is an open question however whether these additional tasks have the same requirements \cite{Kogias2017, Xiang2017} on the resource state as QSS, or whether the optimal resource state for one protocol remains optimal for another protocol. 


%\MT{Add some more detail to this section. Add some examples of states and the transformations on them, and how they are used for QSS. Add some pictures too.}

%\MT{Add some chat about experimental implementations of EBQSS.}

%\MT{Still got some papers I need to talk about.}

\subsection{Sequential QSS}
Although the above protocols which implement QSS using entangled resource states offer an advanced level of security and neatly demonstrate the important role of entanglement in quantum communication, it is hard to see how they will be preferable to qCSS which can offer equivalent levels of security but without the problems associated with generation and distribution of large entangled states. An entanglement-based scheme may even be fine if the number of players is small--for example the schemes \cite{Karlsson1999} \MT{and others} only relying on Bell-pairs--they cannot be easily scaled to many parties. We note that qCSS scales much more favourably as the number of required quantum channels is linear in the total number of players.

It should still be explored whether there are any QSS protocols which outperform qCSS. One promising direction is that of sequential QSS\footnote{This is sometimes referred to as entanglement-free QSS} in which the QSS task is fulfilled by sharing of a single quantum system between multiple players.

In the first sequential QSS protocol \cite{Zhang2005}, Zhang \emph{et. al.} propose a system in which Bob prepares a single photon state with his choice of polarization, and sends it to Charlie. Charlie performs a unitary operation \MT{what does he do?} on the photon and sends it to Alice, who stores the photon in a quantum memory. This process is repeated many times. Later, Alice will sample some of her stored photons for errors \MT{how?}, and on the remaining photons she performs her unitaries \MT{what are they?} and sends them back to Charlie. \MT{how does the rest of the protocol run?}

Just as prepare-and-measure QKD allows Alice and Bob to mimic the measurement outcomes of a shared entangled state \MT{is it only when they measure in the same basis?} under \MT{criterion} the scheme \cite{Zhang2005} allows players to receive the same measurement outcomes they would if they had shared a GHZ state. Secret sharing then may proceed in the usual way. \MT{talk about this.}

Sequential protocols have the obvious advantage that large entangled states are not required. Even though Ref.~\cite{Zhang2005} proposes to use a quantum memory it is ultimately not necessary for the protocol \MT{talk some more about this.}

\MT{Add more stuff about sequential QSS. Make sure I have cited everything.}


%
%\subsection*{Recent quantum secret sharing}
%\MT{(I should think up better titles for these)}
%
%\subsubsection*{Using entangled states}
%\subsubsection*{Using sequential measurements}
%\subsubsection*{Using a QKD-like setup}
% ah, I forgot, the lit review should go in the lit review chapter

\section{Our QSS protocol}

Our quantum secret-sharing scheme allows for a dealer, Alice, to distribute a classical secret between two recipients, Bob and Charlie. Bob and Charlie should be able to exactly reconstruct the secret when they behave honestly, while a dishonest and unauthorised conspiracy of players--including those outside the protocol--should gain no information. Crucially, the scheme should allow for dishonesty among the recipients, but a dishonest player should be forced to collaborate with an honest one.

We propose a QSS protocol which will perform the task of quantum secret sharing without requiring the distribution of highly entangled states between players (in contrast to Ref.~\cite{Kogias2017}) and without requiring a dedicated hardware or network setup (in contrast to Ref.~\MT{cite}). Instead, we rely on the QPSK alphabet \MT{cite} and heterodyne detection \MT{cite} 
\MT{make sure that this description mirrors what I have put in the QDS chapter}

\MT{I don't want to say too much about compatibility with the QDS protocol or QKD here, since this should wait until the agility chapter}

In our protocol, Bob and Charlie are chosen as the senders of the quantum states. This has the advantage of \MT{something}, and allows us to fully trust Alice's heterodyne detection. We note that allowing Bob or Charlie to perform heterodyne detection places implicit trust in their heterodyning beamsplitter \MT{cite 1sDI paper}. Instead, in our configuration we may trust and characterise Alice's device, and exploit the fact that \emph{at least one} recipient is honest, even though Alice does not know which recipient it is.

\begin{figure}[htp]
\centering
\includegraphics{qss_setup.png}
\caption{\label{fig:qss_setup}}
\end{figure}

Since Alice is the dealer who will decide on the eventual shared key our protocol is analogous to a reverse-reconciliation (RR) QKD system, and so we may similarly expect the performance benefits of RR QKD at high loss and noise. We note that having potentially untrusted players as the senders may open the protocol up to new classes of attack, e.g. if they are permitted to send a state which is outside the QPSK alphabet, and such attacks should be addressed in future work.

\MT{talk somewhere about the types of attack we allow}

%OOur QSS protocol runs as follows.
Our QSS protocol runs in three stages, a Distribution stage, an Encryption stage and, finally, a Decryption stage. The Distribution stage involves distribution and measurement of quantum coherent states chosen from QPSK alphabet. At the end of Distribution Alice will hold classical information which is correlated with Bob and Charlie. In the Encryption stage, Alice will combine her classical information and use it to encode her sensitive classical secret. The encoded secret is distributed to Bob and Charlie. The secret is decoded by Bob and Charlie during Decryption. Our protocol setup is described in Fig.~\ref{fig:qss_setup}, and we describe it in detail below.

\MT{make sure that the following is in the same style as my qds protocol description}

\subsubsection*{Distribution stage}
\noindent \underline{Step $1$.} Alice wishes to encrypt a classical secret, $\sigma$. Bob forms a classical sequence $X_B = \left\{x_B^j\right\}_j$ where the $x_B^j$ are complex phases independently chosen from $\mathcal{A}_4$. The $x_B^j$ are assumed to be chosen uniformly at random, but we relax this assumption slightly in Ch.~\MT{X}. Charlie likewise forms $X_C$.

\noindent \underline{Step $2$.} Bob and Charlie form sequences of coherent states
\begin{equation}
\rho\left[X_{\left(B, C\right)}\right] := \otimes_j \rho\left[x_{\left(B, C\right)}^j\right]
\end{equation}
where $\rho\left[x_{\left(B, C\right)^j}\right]$ denotes a coherent state with phase $x_{\left(B, C\right)}^j$. These sequences of states are sent to Alice through quantum channels. \MT{Talk later about the types of channels which these are.}. Alice performs heterodyne detection on each of her received states and records her complex outcomes. We denote strings of Alice's measurement outcomes as $A_B, A_C \in \mathbb{C}$, where the subscript denotes which player sent the corresponding quantum state. The $A_B$ and $A_C$ are kept separate at this stage, and Bob and Charlie should retain their information $X_{\left(B, C\right)}$.

\subsubsection*{Encryption stage}

\noindent \underline{Step $3$.} Alice creates a new complex variable
\begin{equation}
X_A = F\left(A_B, A_C\right)
\end{equation}
from her measurement outcomes. The function $F$ is chosen by Alice and should be freely chosen to optimize security. \MT{Where shall I talk about function $F$?} \MT{I can create some nice graphs of different functions $F$, even those requiring a lot of parameters to be optimized over. But when it comes to actually analysing security I should pick simple ones.} 

\noindent \underline{Step $4$.} Alice now holds string $X_A$ which depends on both Bob and Charlie's choices of states to send. She then uses $X_A$ to encode $\sigma$
\begin{equation}
\tilde{\sigma} = \text{Enc}\left(\sigma, X_A\right)
\end{equation}
\MT{Talk briefly about how this might be done, e.g. with reference to QKD}
and distributes $\tilde{\sigma}$ to Bob and Charlie. Because Bob and Charlie do not have $X_A$ they cannot gain $\sigma$.

\subsubsection*{Decryption stage}

\noindent \underline{Step $5$.} Later, when Alice desires to allow Bob and Charlie access to $\sigma$, she broadcasts which function $F$ she used, along with enough classical information to perform a reconciliation procedure between $X_A$ and $F\left(X_B, X_C\right)$. This stage is similar to regular CV QKD. Bob and Charlie, by working together to form and reconcile $F\left(X_B, X_C\right)$ gain a copy of Alice's key, and are now able to access her secret $\sigma$.

Critical to the protocol is the fact that Alice forms a secret key based on a degree of freedom which is shared between Bob and Charlie. This forces them to collaborate. If one of Bob or Charlie is dishonest, they are forced to work with an honest player and so the scheme as succeeded.

\MT{TODO: convert the strings $X_{A, B, C}$ into random variables describing the classical information, rather than classical strings. It'll make notation in the security proof easier.}

\MT{some more remarks about the running of the protocol}


\section{Security against Eve}
\MT{talk here about security against an external eavesdropper.}

The QSS protocol presented above must be secure against both the actions of an external eavesdropper and the actions of a dishonest Bob or Charlie, who may collaborate with Eve. We first consider just an external Eve to illustrate key steps from the security analysis, an a dishonest Bob or Charlie is considered in the next section.

The starting point for our security analysis is the following Devetak-Winter key-rate bound \MT{cite} \MT{should I motivate why this bound is helpful for us?}

\begin{equation}\label{eqn:qss_dw_eve}
\kappa_{Eve} \ge \text{I}\left(X_A : X_B, X_C\right) - \chi\left(X_A : \mathbb{E}\right)
\end{equation}

\noindent which describes the balance between the mutual information $\text{I}$ shared between Alice and a Bob-Charlie collaboration, and the Holevo information between Eve's quantum system $\mathbb{E}$ and Alice. It is unsurprising that Eq.~\ref{eqn:qss_dw_eve} should be our starting point given the noted similarities between QSS and QKD. The $X_A = F\left(X_B, X_C\right)$ is Alice's variable.

We will consider each term in Eq.~\ref{eqn:qss_dw_eve} in turn.

\subsubsection{Mutual information}

Using Eq.~\MT{X} the mutual information may be written as 
\begin{equation}\label{eqn:qss_deriv_1}
\text{I}\left(X_A : X_B, X_C\right) = \text{H}\left(X_B, X_C\right) - \text{H} \left(X_B, X_C \given X_A\right)
\end{equation}
where the first term on the right hand side is the joint Shannon entropy of $X_B$ and $X_C$, and the second term is the conditional Shannon entropy of $X_B, X_C$ given $X_A$. Intuitively this second term encodes the uncertainty one has about which $X_B, X_C$ were chosen, given a particular choice for $X_A$. 

The joint Shannon entropy may be written
\begin{equation}\label{eqn:qss_deriv_2}
\text{H}\left(X_B, X_C\right) = \sum_{X_B=b, X_C=c} - \text{P}\left(b, c\right) \log \text{P}\left(b, c\right)
\end{equation}
where $b, c$ are individual instances of variables $X_B, X_C$. The $b, c \in \mathcal{A}_4$. Since $b, c$ are taken to be independently chosen and uniformly random we see that the joint probability
\begin{equation}\label{eqn:qss_deriv_3}
\text{P}\left(b, c\right) = \text{P}\left(b\right)\times \text{P}\left(c\right) = \frac{1}{16}
\end{equation}
since each of the $b, c$ are chosen with probability $1/4$. 

Expanding the conditional probability in the prior \MT{is this the right term?} variable $X_A$ we reach

\begin{equation}\label{eqn:qss_deriv_4}
\text{H}\left(X_B, X_C \given X_A\right) = \int\limits_{a \in \mathbb{C}} \Diff2 a \; \text{P}\left(X_A = a\right) \text{H}\left(X_B, X_C \given X_A = a\right).
\end{equation}
Each term in Eq.~\ref{eqn:qss_deriv_4} can be calculated theoretically once function $F$ is known. 

We have no requirement that $F$ should be injective. In particular, this implies that \MT{X}. To be concrete, in what follows we assume that $F$ is linear
\begin{equation}\label{eqn:qss_F_linear}
F\left(x, y\right) := g x + h y \qq{with} g, h \in \mathbb{R}\setminus \left\{0\right\}
\end{equation} 
Although we make no claims about the optimality of this choice of $F$, we are free to optimize over $g, h$, and we will make it clear when we have done so. The conditional entropy in the integrand of Eq.~\ref{eqn:qss_deriv_4} expands as \MT{make equation look nice}

\begin{align}
\text{H}\left(X_B, X_C \given X_A\right) = \sum_{b, c \in \mathcal{A}_4} - &\text{P}\left(X_B=b, X_C=c \given X_A=a\right) \times \notag \\
%
&\log \text{P}\left(X_B=b, X_C=c \given X_A=a\right)
\end{align}

\noindent and so all that remains to calculate are the probabilities \begin{equation}
\text{P}\left(X_A=a\right) \qq{and} \text{P}\left(X_B=b, X_C=c \given X_A=a\right).
\end{equation}

\noindent Applying Bayes' formula to the second of these, we see that
\begin{equation}
\text{P}\left(X_B=b, X_C=c \given X_A=a\right) = \text{P}\left(X_A=a \given X_B=b, X_C=c\right) \frac{\text{P}\left(X_B=b, X_C=c\right)}{\text{P}\left(X_A=a\right)}.
\end{equation}
We can access $\text{P}\left(X_A=a \given X_B=b, X_C=c\right)$ by modelling the effects of the channel on quantum states distributed by Bob and Charlie, which we now do. We take
\begin{equation}\label{eqn:qss_deriv_5}
X_A = F\left(A_B, A_C\right) = g A_B + h A_C
\end{equation}
and thus
\begin{equation}
A_C = \frac{X_A - g A_B}{h}.
\end{equation}

\noindent Since our $F$ is not injective (there are multiple $A_B, A_C$ which will give the same $X_A$) we must average over all of the possible ways to reach a given $X_A$. Therefore, once $X_A$ is fixed, the choice of $A_B, A_C$ reduces to a one-variable problem and so

\begin{equation}
\text{P}\left(X_A \given X_B=b, X_C=c\right) = \int\limits_{A_B \in \mathbb{C}} \Diff2 A_B \; \text{P}\left(A_B , \frac{X_A - g A_B}{h} \given X_B=b, X_C=c\right)
\end{equation}
which may be calculated once we know how the channel acts on input states. A similar equation would be reached by rearranging Eq.~\ref{eqn:qss_deriv_5} as $A_B = \left(X_A - h A_C\right)/g$ but it will make no difference to the resulting quantities.

Assuming that the channel Charlie-Alice is independent from the channel Bob-Charlie\footnote{We shall see later what this physically corresponds to} allows us to write
\begin{equation}\label{eqn:qss_deriv_6}
\text{P}\left(A_B, A_C \given X_b=b, X_C=c\right) = \text{P}\left(A_B \given X_B=b\right) \times \text{P}\left(A_C \given X_C=c\right)
\end{equation}
for Alice's heterodyne measurement outcomes $A_B, A_C$. Let us assume for now that each channel is noiseless but lossy. The probability that Alice measures a particular heterodyne outcome $a \in \mathbb{C}$ when a coherent state of complex amplitude $\beta$ is sent through a lossy channel, transmittivity $T$, is 
\begin{equation}\label{eqn:qss_channel_classical_prob}
\frac{1}{\pi}\exp\left( - \left| a - \sqrt{T}\beta \right|^2\right)
\end{equation}
which we have used previously in Ch.~\MT{X}. The required changes to include thermal noise of the channel can be readily made.

The integral in Eq.~\ref{eqn:qss_deriv_6} may be calculated analytically to reach \MT{TODO: check this against my MMA file}
\begin{align}\label{eqn:qss_deriv_7}
\text{P}\left(X_A \given X_B=b, X_C=c\right) = \frac{1}{\pi} \frac{1}{g^2 + h^2} &\exp \left( - \frac{\left[b^R g \sqrt{T_B} + c^R h \sqrt{T_C} - X_A^R \right]^2}{g^2 + h^2}\right) \notag \\
%
&\exp \left( - \frac{\left[b^I g \sqrt{T_B} + c^I h \sqrt{T_C} - X_A^I \right]^2}{g^2 + h^2} \right)
\end{align}
where $b, c$ are Bob and Charlie's coherent state amplitudes, $X_A$ is Alice's final variable after applying $F$ to her heterodyne outcomes, $T_B, T_C$ are the transmittivities of the Bob-Alice channel and Charlie-Alice channel, respectively, and where a superscript $R\left(I\right)$ denotes the real (imaginary) part of the corresponding quantity. \MT{I probably don't need to say much about how this integration is actually done, since it should be obvious.} The probability $\text{P}\left(X_A=a\right)$ may be readily found by summing Eq.~\ref{eqn:qss_deriv_7} over $b, c \in \mathcal{A}_4$. 

Finally, the mutual information Eq.~\ref{eqn:qss_deriv_1} may be calculated. \MT{Q: do I explicitly perform the integral over $X_A$? If so then I should show it. Otherwise I should mention that it is performed numerically.}

Let us now explore how the mutual information behaves. \MT{Now let's make some graphs and really have fun exploring how $I$ behaves.}

\subsubsection{Holevo information}

We will now detail how the Holevo information term in Eq.~\ref{eqn:qss_dw_eve} may be calculated. In doing so we will point to areas where future work might strengthen the security analysis to wider classes of attack, which should help to illuminate the contexts to which our security proof may be applied. In this section we consider a dishonest Eve performing attack BS$0$, as detailed above in Sec.~\MT{X}, and more general attacks will be considered later.

Bob and Charlie prepare a state from QPSK alphabet with equal probability. The coherent states should be independently and randomly chosen. Before the channel they hold the joint state
\begin{equation}
\rho_{\text{before}} = \rho_B \otimes \rho_C
\end{equation}
with
\begin{equation}
\rho_B = \frac{1}{4} \sum_{k=0}^3 \dyad{\beta_k}_B \qq{and} \rho_C = \frac{1}{4} \sum_{k^\prime = 0}^3 \dyad{\gamma_{k^\prime}}_C
\end{equation}
where $\beta, \gamma$ are the amplitudes of Bob's and Charlie's coherent state alphabets.\footnote{Complex amplitudes $\beta, \gamma$ were denoted $b, c$ in the previous section.}

We assume that the channel acts separately on each mode, and that modes $\rho_B$, $\rho_C$ undergo independent evolution. In other words, we assume that the channel has tensor-product structure
\begin{equation}
\Phi\left[\rho\right] = \Phi_B\left[\rho\right] \otimes \Phi_C\left[\rho\right]
\end{equation}
where \MT{check this equation}
\begin{equation}
\Phi_B\left[\rho\right] = \phi_B\left(\text{Tr}_B \rho\right) \otimes \mathbb{1}\left(\rho\right).
\end{equation}

\noindent Eve performs separate beamsplitter attacks on each channed and retains two output modes $\mathbb{E}_{B, C}$. The total state after the channel becomes
\begin{equation}
\rho_{\text{after}} = \rho_{\mathbb{A}_B, \mathbb{E}_B} \otimes \rho_{\mathbb{A}_C, \mathbb{E}_C}
\end{equation}
with $\mathbb{A}_{B, C}$ denoting Alice's two modes and where
\begin{equation}
\rho_{\mathbb{A}_B, \mathbb{E}_B} = \frac{1}{4} \sum_{k=0}^3 \dyad{\sqrt{T_B} \beta_k}_{\mathbb{A}_B} \otimes \dyad{\sqrt{1-T_B} \beta_k}_{\mathbb{E}_B}
\end{equation}
and similarly for $\rho_{\mathbb{A}_C, \mathbb{E}_C}$. Now, Alice heterodynes and measures $A_B \in \mathbb{C}$ from $\rho_{\mathbb{A}_B, \mathbb{E}_B}$ and $A_C \in \mathbb{C}$ from $\rho_{\mathbb{A}_C, \mathbb{E}_C}$. Eve's total state conditioned on these outcomes becomes 
\begin{equation}
\rho_{\left.\mathbb{E} \given A\right.} = \rho_{\left.\mathbb{E}_B \given A_B\right.} \otimes \rho_{\left. \mathbb{E}_C \given A_C\right.}
\end{equation}
with
\begin{equation}
\rho_{\left.\mathbb{E}_B \given A_B\right.} = \frac{1}{4 \pi} \sum_{k=0}^3 \text{P}_B\left(A_B \given \beta_k, T_B\right) \dyad{\sqrt{1-T_B} \beta_k}_{\mathbb{E}_B}
\end{equation}
and similarly for $\rho_{\left.\mathbb{E}_C \given A_C\right.}$. The probability $\text{P}_B\left(A_B \given \beta_k, T_B\right)$ is calculated analogously to Eq.~\ref{eqn:qss_channel_classical_prob}, and similarly for $A_C$.

Take $X_A = g A_B + h A_C$ as usual, with $g, h$ fixed, and write $A_C = \left(X_A - g A_B\right)/h$. Therefore we have

\begin{align}
\rho_{\left. \mathbb{E} \given X_A, A_B\right.} &= \frac{1}{16 \pi^2} \sum_{k, k^\prime = 0}^3 \text{P}_B\left(A_B \given \beta_k, T_B\right) \text{P}_C\left(\frac{X_A - g A_B}{h} \given \gamma_{k^\prime}, T_C\right) \notag \\
%
&\dyad{\sqrt{1-T_B} \beta_k}_{\mathbb{E}_B} \otimes \dyad{\sqrt{1-T_C}\gamma_{k^\prime}}_{\mathbb{E}_C}
\end{align}

\noindent Once again since Alice's function $F$ is in general not injective, we must mix over outcomes $A_B, A_C$ in order to find Eve's state $\rho_{\left.\mathbb{E} \given X_A\right.}$

\begin{equation}\label{eqn:qss_aposteriori_state}
\rho_{\left.\mathbb{E} \given X_A\right.} = \int\limits_{A_B \in \mathbb{C}} \Diff2 A_B \; \text{P}\left(A_B\right) \rho_{\left.\mathbb{E} \given X_A, A_B\right.}
\end{equation}

\noindent and mixing over $X_A$ we finally reach

\begin{equation}\label{eqn:qss_apriori_state}
\rho_{\mathbb{E}} = \int\limits_{X_A \in \mathbb{C}} \Diff2 X_A \; \text{P}\left(X_A\right) \rho_{\left.\mathbb{E}\given X_A\right.}.
\end{equation}

\noindent We may identity Eq.~\ref{eqn:qss_aposteriori_state} as Eve's \emph{a prosteriori} state and Eq.~\ref{eqn:qss_apriori_state} as Eve's \emph{a priori} state and so Eve's Holevo information is given by the usual formula

\begin{equation}\label{eqn:qss_holevo}
\chi = \text{S}\left(\rho_\mathbb{E}\right) - \int\limits_{X_A \in \mathbb{C}} \Diff2 X_A \; \text{P}\left(X_A\right) \text{S}\left(\rho_{\left.\mathbb{E} \given X_A\right.}\right).
\end{equation}

\noindent Let us explore the behaviour of Eve's Holevo information Eq.~\ref{eqn:qss_holevo}.

\MT{TODO: make some nice graphs of $\chi$ in different scenarios and under different attacks BS1, BS2, EC.}

\section{Security against a dishonest player}
\MT{Talk about guarding against a dishonest Bob/Charlie.}
Including a dishonest player, Bob or Charlie, in the above security proof requires us to re-calculate several quantities from the above section. For concreteness we will first assume that Bob is dishonest and Charlie is honest, and we allow Bob to collaborate with Eve. Later we will discuss how to account for the fact that we do not know \emph{which} player is dishonest.

The effect of including a dishonest player is that Bob knows precisely which coherent states he sent to Alice, and so he should have reduced uncertainty (increased Holevo information) about $X_A$. Intuitively it is also possible that Bob could wait and see which coherent states Charlie sent before choosing his own, in order to preference a certain outcome $X_A$. \MT{Make a comment about this. Would it be taken care of by Alice's optimization over $g, h$?} We will also assume that Bob sends states from his QPSK alphabet, though this could be relaxed in future work.

Since Bob knows which coherent states he sent we must re-calculate several expressions from the previous section. \MT{talk about which ones should be changed.}





























\section{Outlook}
\MT{this section should be somewhere else, perhaps in an "outlook" section?}
The classical post-processing of the above protocol is inherently very similar to Ref.~\cite{Kogias2017}, in which a secret key is generated between Alice and a shared Bob-Charlie degree of freedom via incompatible homodyne measurements on a tripartite entangled state. We expect that our protocol will be secure against a more restricted set of attacks, but over a wider range of channel parameters, for several reasons. 

Ref.~\cite{Kogias2017} has potentially dishonest players Bob and Charlie performing homodyne measurements on incompatible observables (i.e. switching between $q$ and $p$ quadratures). No assumptions are made about the measurement devices used and they are each treated as a "black-box". Security comes inherently because of a Heisenberg-type relation between incompatible observables, and the security proof relies on an Entropic Uncertainty Relation (EUR). These EURs have had success elsewhere in quantum cryptography \MT{cite}. However, since we desire to use heterodyne detection we are forced to adopt a different approach and explicitly model the states' evolution and measurement during the protocol. We note that this matches the current state-of-the-art of QPSK-based QKD, but can be improved in future work.

We have assumed that a dishonest Bob or Charlie still sends a state from the QPSK alphabet. It is yet unclear whether they could gain an advantage by sending something exotic and potentially highly entangled, perhaps in order to force Alice to reach a certain key $X_A$. This should be explored and potentially relaxed in future work. We anticipate that applying methods from quantum bit commitment might prove fruitful here. 

Finally, we note that our assumption that the channel between Alice and Bob-Charlie takes a tensor-product structure is perhaps a strong one and should be relaxed. A potential strategy of a dishonest player could be to exploit properties of a general channel which maps a two-mode input state to a two-mode output state at Alice, though potentially allowing a dishonest player many output ancilla modes correlated with Alice. Such a strategy will be restricted by the conditions that the reduced state of an honest player should be a coherent state. Similarly it will require that Alice's measurement outcomes don't look errant, though this should be quantified.








