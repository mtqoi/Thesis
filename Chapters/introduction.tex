\chapter{Introduction}

\subsection{Things to include}
\begin{itemize}
\item Beamsplitter properties and relations
\item Definitions and interpretations of various entropic quantities
\item Holevo information
\item Phase space
\item Background quantum mechanics
\item POVMs
\item Covariance matrices
\item Master equations
\item Coherent states
\item Homodyning and heterodyning
\item Mixed states, entangled states
\item Thermal state
\item TMSV state
\item binary entropy
\item Fock basis
\end{itemize}

\subsection{Hoeffding's inequalities}
Let $\mathcal{X} = X_1, X_2, \dots, X_n$ be $n$ independent binary random variables. Let $\bar{\mathcal{X}}$ be their empirical mean \MT{define} and let $\mathbb{E}\left(\bar{\mathcal{X}}\right)$ be its expected value. Then $\forall \epsilon \ge 0$ we may bound the probability that the empirical mean $\bar{\mathcal{X}}$ differs from its expectation $\mathbb{E}\left(\bar{\mathcal{X}}\right)$ by the following inequalities

\begin{align}
\label{eqn:hoeffding1}
\text{P}\left(\bar{\mathcal{X}} - \mathbb{E}\left(\bar{\mathcal{X}}\right) \ge \epsilon\right) &\le \text{exp}\left(- 2 \epsilon^2 n\right) \\
\label{eqn:hoeffding2}
\text{P}\left(\mathbb{E}\left(\bar{\mathcal{X}}\right) - \bar{\mathcal{X}} \ge \epsilon\right) &\le \text{exp}\left(- 2 \epsilon^2 n\right).
\end{align}





\noindent These inequalities are known as Hoeffding's inequalities \MT{cite} and will provide a necessary tool for analysis of our Quantum Digital Signatures protocol.


\subsection{Coherent state alphabets}

\begin{figure}[htp]
\centering
\begin{subfigure}[b]{0.4\linewidth}
\includegraphics[draft=false, width=\linewidth]{introduction/qpsk_1}
\caption{}
\end{subfigure}
\begin{subfigure}[b]{0.4\linewidth}
\includegraphics[draft=false, width=\linewidth]{introduction/qpsk_2}
\caption{}
\end{subfigure}
\caption{\label{fig:qpsk} The Wigner function for a mixture over $\mathcal{A}_4$ is a sum of individual Wigner functions for each of the coherent states. QPSK alphabet with (a) $\alpha=1.5$; (b) $\alpha=0.8$}
\end{figure}

% In a quantum communications protocol, the alphabet states should be chosen with significant overlap.

\subsection{Thermal state}

\begin{figure}[htp]
\centering
\includegraphics[draft=false, width=0.4\linewidth]{introduction/thermal_state}
\caption{\label{fig:thermal_state} The thermal state Wigner function is Gaussian, with variance greater than the vacuum state vacuum, depicted in orange.}
\end{figure}

\subsection{TMSV state}
\begin{figure}[htp]
\centering
\includegraphics[draft=false, width=0.8\linewidth]{introduction/tmsv_wigner}
\caption{\label{fig:tmsv_wigner} Plots of the Wigner function for each reduced mode of the TMSV state. Locally the modes look like thermal states, see Fig.~\ref{fig:thermal_state}. The vacuum variance is depicted in orange.}
\end{figure}

\begin{figure}[htp]
\centering
\includegraphics[draft=false, width=0.8\linewidth]{introduction/tmsv_histogram}
\caption{\label{fig:tmsv_histogram} \MT{caption for TMSV histogram}}
\end{figure}

\begin{figure}
\centering
\includegraphics{binary_entropy.png}
\caption{\label{fig:binary_entropy}}
\end{figure}
