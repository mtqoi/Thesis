\chapter{Introduction}


\section{Introduction to Thesis}
\MT{Statements about historical background of quantum optics/quantum information/quantum cryptography}

This Thesis consists of two parts. In Part One, we consider several related cryptographic protocols which securely perform the tasks of Quantum Digital Signatures (QDS) and Quantum Secret Sharing (QSS). Our goals here are, firstly: to remove an assumption in continuous-variable (CV) QDS of secure quantum channels, and provide a security proof when an eavesdropping attack is permitted, and secondly: to demonstrate that several CV quantum cryptographic protocols may be run over identical hardware setups while the hardware at the quantum level is agnostic to the protocol being implemented. This so-called ``agile" system illustrates our translation of cryptographic agility from classical (conventional) cryptography to quantum cryptography, and allows for a move towards secure and practical quantum cryptosystems which can perform multiple tasks.

In Part Two we consider the task of quantum state generation. We design and analyse a system which is capable to produce highly non-classical states at output, deterministically from a quasi-classical coherent state input. Using methods from nonlinear pulse propagation in fibers and from the field of open quantum systems, we analytically and numerically analyse a large multimode system and reduce it, step-by-step, to a single-mode system which is much more mathematically tangible. The system, which we denote PhoG (\underline{Pho}ton \underline{G}un), when implemented will provide a practical and cheap source of nonclassicality for quantum enhanced imaging and metrology, and may even improve the performance of quantum cryptographic systems.

This Thesis is structured as follows. In the remainder of this chapter we will introduce and outline several of the theoretical and analytical tools which we make extensive use of in the rest of the Thesis. We will show how the electromagnetic field may be quantized and then discuss several common and useful quantum states of the field, and examine one method for their description and visualization. In Chapter~\ref{chapter:crypto_intro} we outline some of the developments in conventional cryptography which underpin our modern communications infrastructure, and discuss how two cryptographic tasks--digital signatures and secret sharing--may be translated to the quantum realm. In particular we will look at several recent and historic attempts to build such quantum protocols.

\paragraph{Part One:} In Chapter~\ref{chapter:qds} we introduce our own QDS protocol, and prove its security against several classes of attack. We show that ours is the first CV QDS protocol to allow for an eavesdropper on the channels, and we show that despite this we attain very short signature lengths over practical distances. In Chapter~\ref{chapter:qss} we introduce our QSS protocol, prove its security, and analyse its performance. Crucially, unlike several recent QSS protocols, our protocol does not require generation and distribution of large-scale entangled states, nor does it require dedicated hardware for a sequential ``round-robin" style of approach. In the last chapter of Part One, Chapter~\ref{chapter:aqc} we introduce and discuss the concept of quantum cryptographic agility, and demonstrate how it may apply to our protocols discussed in earlier chapters. We additionally introduce a new QDS protocol which runs in a modified configuration, and discuss how a quantum key distribution (QKD) protocol which already exists in the literature may be adopted into our agile system. We finish with a figure-of-merit graph which demonstrates that our QDS scheme, in addition to being practical and compatible with commercial telecommunications hardware, is also the fastest QDS protocol over comparable distances. 

\paragraph{Part Two:} In Chapter~\ref{chapter:phog} we motivate and introduce the PhoG device which will be the focus of the second half of this Thesis. We demonstrate that dissipation, far from being a hinderance to the desired evolution of our quantum system, is actually an asset and the main driver towards target nonclassicality. We introduce an exotic form of dissipation, called Nonlinear Coherent Loss, and show that in the ideal limit it will deterministically lead to single photons at the output. We then introduce progressively more complex models involving more bosonic system modes, and demonstrate that a realistic full multi-mode model of the system can be used to effectively simulate the Nonlinear Coherent Loss decay channel. Finally, we end the chapter by demonstrating that in addition to generating quasi-single-photon states, a slight modification to the PhoG device will lead to generation of entanglement at the output.

\paragraph{Part Three:} We finish the Thesis with the inclusion of several appendices containing results which are used at multiple points throughout the Thesis. Appendix~\ref{appendix:cryptography_numerical_methods} contains the forms of quantum states which are used under various channel attacks in Chapter~\ref{chapter:qds}, and a description of how the states may be numerically modelled. Appendix~\ref{appendix:qds_larger_alphabets} contains a generalization to the QDS protocol from Chapter~\ref{chapter:qds} to allow for larger alphabets of coherent states. Appendix~\ref{appendix:qds_noisy_channel} contains results describing the output state of a channel in which an initial coherent state is mixed with thermal noise. This result is used multiple times throughout Part One. Appendix~\ref{appendix:adiabatic_elimination} provides a tool which is used multiple times in Chapter~\ref{chapter:phog} to reduce the complexity of a model by effectively ignoring a mode which reaches its steady state much quicker than the typical decay time of the system. Finally, in Appendix~\ref{appendix:phog_numerical_methods} we describe several numerical methods which may be applied to model the PhoG device. We compare several of them for speed, memory usage and accuracy, and then explicitly display several systems of coupled differential equations which approximate the PhoG device. Appendix~\ref{appendix:bibliography} contains the Bibliograpy.

\section{Introduction to Quantum Optics}


\FloatBarrier
\subsection{Quantization of the electromagnetic field}


\FloatBarrier
\subsection{Single-mode operators}
We have seen that the bosonic operators $\hat{a}, \hat{a}^\dagger$ create and destroy a quantum of energy in the light mode. These operators obey the bosonic commutation relation
\begin{equation}
\left[ \hat{a}, \hat{a}^\dagger \right] = 1
\end{equation}
and are crucial for modelling the quantum properties of our light. 

We define the number operator $\hat{n}$ as 
\begin{equation}
\hat{n} = \hat{a}^\dagger \hat{a},
\end{equation}
and we will show in the next section that $\hat{n}$ counts the number of photons in our mode.

We construct the following two operators

\begin{equation}
\hat{q} = \frac{\hat{a}^\dagger + \hat{a}}{\sqrt{2}} \qq{and} \hat{p} = i \frac{\hat{a}^\dagger - \hat{a}}{\sqrt{2}}
\end{equation}
which can be shown to obey the commutator
\begin{equation}\label{eqn:intro_quadrature_commutator}
\left[\hat{q}, \hat{p}\right] = i.
\end{equation}

\noindent The quadrature operators may be used to write the well known Heisenberg uncertainty principle 
\begin{equation}
\text{Var}\left(\hat{q}\right)\text{Var}\left(\hat{p}\right) \ge 1/4, %note that this is 1/4 not 1/2, due to using variances (not standard deviations) on the LHS. c.f. Ulf eq 3.80 p54
\end{equation} 
where the variances in a general quantum state $\ket{\psi}$ are defined as
\begin{align}
\text{Var}\left(\hat{q}\right) &= \mel{\psi}{\hat{q}^2}{\psi} - \mel{\psi}{\hat{q}}{\psi} \notag \\
\text{Var}\left(\hat{p}\right) &= \mel{\psi}{\hat{p}^2}{\psi} - \mel{\psi}{\hat{p}}{\psi}. 
\end{align}

\noindent The commutator Eq.~\ref{eqn:intro_quadrature_commutator} is equivalent to the commutator describing position and momentum of a quantum particle, and so we denote $\hat{x}$ as the position operator and $\hat{p}$ as the momentum operator. Indeed, it is easy to show that
\begin{equation}
\hat{E} = \hat{n} + \frac{1}{2} = \frac{\hat{x}^2}{2} + \frac{\hat{p}^2}{2},
\end{equation}
the right-hand side of which describes the energy of a simple harmonic oscillator. The position and momentum operators $\hat{x}, \hat{p}$ describe real and imaginary components of the complex phase $\hat{a}$ of the electric field.



\FloatBarrier
\subsection{State vectors and density matrices}\label{sec:intro_state_vectors}
An ideal quantum state is denoted in the so-called Dirac notation by $\ket{\psi}$. This state has been perfectly prepared with no noise, loss or additional uncertainty due to the preparation. The $\ket{\psi}$ is a vector living in a vector space denoted $\mathcal{H}$, which we call a Hilbert space. The space has dimension $\dims$ which we will typically take to be countably infinite, but there will be several points in this Thesis where we consider a finite $\dims$.

Since $\ket{\psi}$ is a vector it can be written in terms of a set of basis vectors $\left\{\ket{e_j}\right\}_j$
\begin{equation}
\ket{\psi} = \sum_j c_j \ket{e_j}
\end{equation}
where the number of basis vectors typically equals $\dims$. 

The state $\ket{\psi}$ should be normalized, that is
\begin{equation}
\dyad{\psi} = 1,
\end{equation}
where the notation $\dyad{\psi}$ denotes 
\begin{equation}
\sum_j \left|c_j\right|^2.
\end{equation}

\noindent We define an eigenstate of an arbitrary quantum operator $\hat{r}$ to be the state $\ket{r}$ such that
\begin{equation}
\hat{r} \ket{r} = r \ket{r}
\end{equation}
with $r \in \mathbb{C}$.

The basis we choose for vector $\ket{\psi}$ is not unique and we may similarly have expanded $\ket{\psi}$ in terms of eigenstates of $\hat{r}$,
\begin{equation}
\ket{\psi} = \sum_j \dyad{r}{\psi} \ket{r}
\end{equation}
where the $\dyad{r}{\psi} \in \mathbb{C}$ is such that its square modulus $\left|\dyad{r}{\psi}\right|^2$ is the probability that a measurement of $r$ on $\ket{\psi}$ will give outcome $r$.

For convenience any basis vectors we use are chosen to be orthonormal to each other, that is
\begin{equation}
\dyad{\hat{e}_j}{\hat{e}_k} = \delta_{j, k}
\end{equation}
where $\delta_{j, k}$ is the Kronecker $\delta$ function.

That a state $\ket{\psi}$ may be written as a sum of basis vectors corresponding to different measurement outcomes is a curious one, and a key feature of quantum mechanics known as the \emph{superposition principle}. A superposition state is one of the form
\begin{equation}\label{eqn:intro_superposition}
\ket{\psi} = \frac{\ket{0} + \ket{1}}{\sqrt{2}}
\end{equation}
where the constant factor $\mathcal{N}$ ensures normalization. We have chosen an abstract orthonormal basis set $\left\{ \ket{0}, \ket{1}\right\}$, and here $\dims=2$. The superposition state possesses a fundamental uncertainty about its measurement outcomes, that is to say, if many identical copies of $\ket{\psi}$ are created, and on each copy a measurement which distinguishes between $\ket{0}$ and $\ket{1}$ is performed, then the measurement will output $0$ half of the time, and $1$ the other half of the time. Such uncertainty is intrinsic to quantum mechanics and is unavoidable. However, it should be noted that unlike a classical uncertainty which one desires to reduce, this quantum uncertainty is highly desirable and forms the basis for many useful applications of quantum states.

The density operator corresponding to a quantum state is
\begin{equation}
\hat{\rho} = \sum_{i, j} \rho_{i, j} \dyad{i}{j}
\end{equation}
where $\bra{i} \in \mathcal{H}^*$ is dual to $\ket{i}$. The matrix $\rho_{i, j}$ is referred to as the density matrix, though we shall often use the terms \emph{density operator} and \emph{density matrix} interchangably. As such, we will often neglect the hat above $\hat{\rho}$, and simply write $\rho$.

The density operator description of a quantum state allows for a wider range of states to be described than the state vector description. Any state vector $\ket{\psi}$ can be described as a density operator. For example, the state $\ket{\psi}$ in Eq.~\ref{eqn:intro_superposition} may equivalently be described as 
\begin{equation}
\rho_\psi = \dyad{\psi}{\psi} = \frac{1}{2} \left(\dyad{0}{0} + \dyad{0}{1} + \dyad{1}{0} + \dyad{1}{1} \right).
\end{equation}
However, there are density operators which do not correspond to a state vector. For example, there exists no vector $\ket{\phi}$ such that
\begin{equation}
\rho_{\varphi} := \frac{\dyad{0}{0} + \dyad{1}{1}}{2} = \dyad{\phi}.
\end{equation}
The density operator formalism may be interpreted as encoding two distinct forms of uncertainty: quantum and classical. The quantum uncertainty arises from states which may be written as superposition state vector, while the classical uncertainty arises from states which cannot. We refer to the first type of summation as superposition, and the second type of summation as classical mixing. The classical mixing represents classical uncertainty about which state the system is in, and may in principle be removed simply by keeping more accurate records, or by building better equipment.

In the $\left\{\ket{0}, \ket{1}\right\}$ basis, the states $\rho_\phi$ and $\rho_\varphi$ may be written as
\begin{equation}
\rho_\phi = \frac{1}{2}\pmqty{1 & 1 \\ 1 & 1} \qq{and} \rho_\varphi = \frac{1}{2}\pmqty{1 & 0 \\ 0  & 1}.
\end{equation}

\noindent The quantum nature of a state $\rho$ is therefore intimately connected to the off-diagonal elements of its density matrix. These off-diagonal elements may be referred to as \emph{coherences}. 

The density matrix $\rho$ will be our primary tool for describing a quantum state in this Thesis.


\FloatBarrier
\subsection{Wigner function}
An equivalent description of the quantum state is to use a (\emph{quasi})-probability distribution known as the Wigner function. The Wigner function, $W\left(q, p\right)$ allows operator expectation values to be calculated and so can be thought of as an accurate description of the quantum state. We desire $W\left(q, p\right)$ to behave analogously to a joint probability distribution, and by analogy with a classical phase space picture in which the system is described in terms of position and canonical momentum, we introduce a quantum phase space. A phase space picture is not strictly necessary for the work in this Thesis, but the Wigner function provides a useful visualisation tool.

One immediate observation is that the quantum phase space must behave qualitatively differently to the classical one. Because of Heisenberg's principle we cannot accurately define a joint probability distribution of position $q$ and momentum $p$, and this leads to interesting behaviours of the phase space distribution, such as becoming negative (for Wigner functions) or highly singular (for ``$P$-functions''). Indeed, one may even measure the ``quantumness'' of a given state by checking how pathological is the chosen quasi-probability distribution.

We define the Wigner function corresponding to density operator $\hat{\rho}$ as \cite{Leonhardt2010}

\begin{equation}
W\left(q, p\right) = \frac{1}{2 \pi} \int\limits_{-\infty}^{\infty} \mathrm{d}x \; \exp\left(i p x\right) \mel{q - \frac{x}{2}}{\hat{\rho}}{q + \frac{x}{2}}.
\end{equation}

\MT{Look at q and p marginals.}

\noindent A useful feature of the Wigner function is that traces over operators may be calculated as

\begin{equation}\label{eqn:wigner_overlap}
\tr\left[\hat{O}_1 \hat{O}_2\right] = 2\pi \int\limits_{-\infty}^\infty \int\limits_{-\infty}^{\infty} \mathrm{d}q \; \mathrm{d}p \; W_1\left(q, p\right) W_2\left(q, p\right)
\end{equation}
where $W_i\left(q, p\right)$ is the Wigner function corresponding to operator $\hat{O}_i$. Operator expectation values with respect a given state $\hat{\rho}$ may be calculated using Eq.~\ref{eqn:wigner_overlap}. 

The well known Heisenberg uncertainty relation expresses itself in the Wigner function picture by imposing a minimum area which a quantum state must occupy. In the following sections we will see some examples of quantum states and their corresponding Wigner functions. 



Finally, we must note that the Wigner function is not the only quasi-probability distribution which one could define on the phase space. It is a purely non-classical feature that there are multiple ways to consistently define a phase space description. Other common quasi-probability distributions are: the Hussimi $Q$ function \MT{cite} which is intimately related to Heterodyne measurement; Glauber-Sudarshan \MT{spelling} $P$ function which is often used to describe mixtures of coherent states and becomes highly singular in all other cases; and the positive-$P$ function which is a generalization to the $P$ function defined as non-diagonal in the coherent state basis, and which allows quantum effects such as squeezing to be described. One may also use these quasi-probability distributions to predict dynamics of the system using a Fokker-Planck equation \MT{cite}.


\FloatBarrier
\subsection{Fock states}
Fock states (also called photon-number states) are defined as eigenstates of the photon-number operator $\hat{n} = \hat{a}^\dagger \hat{a}$:
\begin{equation}
\hat{n} \ket{n} = n \ket{n}.
\end{equation}
In other words, $\hat{n}$ measures the number of photons in $\ket{n}$. The states have perfectly defined photon-number and find many applications in quantum information processing. The creation and annihilation operators $\hat{a}^\dagger, \hat{a}$ act on $\ket{n}$ as 
\begin{align*}
\hat{a}\ket{n} &= \sqrt{n}\ket{n-1} \\
\hat{a}^\dagger \ket{n} &= \sqrt{n+1} \ket{n=1}.
\end{align*}
The Fock states form an orthogonal basis for $\mathcal{H}$
\begin{equation}
\ip{m}{n} = \delta_{n, m}
\end{equation}
and so we will often seek an expansion of other states in the Fock-state basis. We will regularly choose to write our density matrix in a Fock state expansion as
\begin{equation}
\rho_{m, n} = \mel{m}{\hat{\rho}}{n}.
\end{equation}

\noindent Wigner functions corresponding to some example Fock states are displayed in Fig.~\ref{fig:intro_fock_wigner}.


\begin{figure}[htp]
\centering
\includegraphics{intro/fock_wigner.png}
\caption{\label{fig:intro_fock_wigner}}
\end{figure}



\FloatBarrier
\subsection{Quadrature states}
We define the eigenstates of quadrature operators $\hat{q}$ and $\hat{p}$ as
\begin{equation}
\hat{q}\ket{q} = q \ket{q} \qq{and} \hat{p}\ket{p} = p \ket{p}.
\end{equation}
These $\ket{q}, \ket{p}$ are known as quadrature states. Although they are not normalizable, the quadrature states are useful for describing ideal homodyne detection, Sec.~\ref{sec:intro_homodyne}. %We display approximations to Wigner functions of 



\FloatBarrier
\subsection{Coherent states}
Coherent states\footnote{Also known as Glauber coherent states} are among the most important quantum states which we discuss in this Thesis. The coherent state $\ket{\alpha}$ is defined as the eigenstate of annihilation operator $\hat{a}$:
\begin{equation}
\hat{a}\ket{\alpha} = \alpha \ket{\alpha}.
\end{equation}
The coherent state has amplitude $\alpha \in \mathbb{C}$, and we display several examples of coherent state Wigner functions in Fig.~\ref{fig:intro_coherent_wigner}. It can additionally be shown that the area occupied by the coherent state is the smallest allowable area of phase space for a Wigner function to cover. In other words, the coherent state saturates the Heisenberg bound and is a minimum uncertainty state.

\begin{figure}[htp]
\centering
\includegraphics{introduction/coherent_wigner.png}
\caption{\label{fig:intro_coherent_wigner}}
\end{figure}

The coherent states are non-orthogonal
\begin{equation}
\ip{\alpha}{\beta} = \exp\left(- \frac{\left|\alpha\right|^2}{2} - \frac{\left|\beta\right|^2}{2} + \alpha^* \beta \right),
\end{equation}
where $\alpha^*$ denotes the complex conjugate of $\alpha$, and so we will regularly make use of an expansion of $\ket{\alpha}$ in the orthogonal Fock basis:
\begin{equation}\label{eqn:intro_coherent_fock}
\ket{\alpha} = e^{-\frac{\left|\alpha\right|^2}{2}} \sum_{n=0} \frac{\alpha^n}{\sqrt{n!}} \ket{n}.
\end{equation}

\noindent In the first part of this Thesis we will consider several quantum cryptographic protocols involving distribution of coherent states. We will reglarly use an alphabet of coherent states known as the QPSK (Quadrature Phase-Shift Keying) alphabet, 

\begin{equation}
\qq*{QPSK alphabet:} \left\{\ket{\alpha}, \ket{i \alpha}, \ket{- \alpha}, \ket{- i \alpha}\right\}
\end{equation}

and we display a mixture over the QPSK alphabet in Fig.~\ref{fig:qpsk}. We will not explicitly distinguish between whether we refer to the set of quantum states $\left\{\ket{\alpha}, \ket{i \alpha}, \ket{- \alpha}, \ket{- i \alpha}\right\}$ or the corresponding set of phases $\left\{ \alpha, i \alpha, -\alpha, - i\alpha\right\}$ since it should always be obvious which is meant.

Finally, we consider a special case of the coherent states, that with eigenvalue $0$
\begin{equation}
\hat{a} \ket{0} = 0 \ket{0}.
\end{equation}
\noindent This state is known as the ``vacuum'' state, and is a special example since $\ket{0}$ is also an eigenstate\footnote{It can also be viewed as a thermal state with $\bar{n}=0$.} of $\hat{n}$, and is thus a Fock state with a photon number $0$.


\begin{figure}[htp]
\captionsetup{width=0.8\linewidth}
\centering
\begin{subfigure}[b]{0.4\linewidth}
\includegraphics[draft=false, width=\linewidth]{introduction/qpsk_1}
\caption{}
\end{subfigure}
\begin{subfigure}[b]{0.4\linewidth}
\includegraphics[draft=false, width=\linewidth]{introduction/qpsk_2}
\caption{}
\end{subfigure}
\caption{\label{fig:qpsk} The Wigner function for a mixture over $\mathcal{A}_4$ is a sum of individual Wigner functions for each of the coherent states. QPSK alphabet with (a) $\alpha=1.5$; (b) $\alpha=0.8$ \MT{Beef up caption.}}
\end{figure}

Finally, a generalization to QPSK is the so-called $N$PSK alphabet, in which $N$ coherent states are chosen. The states are again equally distributed around the origin of phase space, and we note that QPSK is the special case $N=4$. We display some examples in Fig.~\ref{fig:intro_npsk}.

\begin{figure}[htp]
\captionsetup{width=0.8\linewidth}
\centering
\includegraphics{intro/npsk.png}
\caption{\label{fig:intro_npsk}}
\end{figure}

\FloatBarrier
\subsection{Thermal states}

The thermal state is defined as
\begin{equation}
\rho_{\text{thermal}} = \left( 1 - e^{-\beta} \right) \sum_{n=0}^\infty e^{- n \beta} \dyad{n}
\end{equation}
with $\beta = \left( \hbar \omega \right)/\left(k_B T\right)$, for reduced planck's constant $\hbar$, angular frequency $\omega$, Boltzmann's constant $k_B$ and thermal equilibrium temperature $T$. The thermal state is a classical mixture of Fock states. Typically we will parametrise the thermal state using the thermal photon number $\bar{n}$ which is defined as 
\begin{equation}
\bar{n} = \frac{1}{e^\beta - 1}.
\end{equation}
We display the Wigner function of a thermal state in Fig.~\ref{fig:thermal_state}, where we have also displayed the vacuum state variance, for comparison.


\begin{figure}[htp]
\captionsetup{width=0.8\linewidth}
\centering
\includegraphics[draft=false, width=0.4\linewidth]{introduction/thermal_state}
\caption{\label{fig:thermal_state} The thermal state Wigner function is Gaussian, with variance greater than the vacuum state vacuum, depicted in orange.}
\end{figure}
\subsection{Squeezed states (quadrature)}
\MT{introduce heisenberg before now. Talk about it in terms of phase space area. Show that coherent states are minimum uncertainty states. Introduce vacuum state in the coherent state section.}
We have already encountered the Heisenberg relation which specifies the minimum phase space area which a state can occupy. The coherent state was a minimum uncertainty state and occupied the minimum possible area, while possessing symmetry in $\Delta q = \Delta p$. Of course, it is possible to satisfy the Heisenberg relation while also taking $\Delta q \ne \Delta p$, and this is precisely what (quadrature) squeezed states do. We display several examples of quadrature squeezed states in Fig.~\ref{fig:intro_quadrature_squeezed_wigner}. In the limit of infinite squeezing one obtains a state analogous to quadrature states $\ket{q}$ and $\ket{p}$.

\begin{figure}[htp]
\captionsetup{width=0.8\linewidth}
\centering
\includegraphics{intro/quadrature_squeezed_wigner.png}
\caption{\label{fig:intro_quadrature_squeezed_wigner}}
\end{figure}

\FloatBarrier
\subsection{Squeezed states (photon-number)}
We have seen that the coherent state $\ket{\alpha}$ may be expanded in Fock basis as Eq.~\ref{eqn:intro_coherent_fock}. From this equation it may be shown that the photon-number distribution of $\ket{\alpha}$ is 
\begin{equation}
\mathcal{P}_n\left[\ket{\alpha}\right] = \frac{\left|\alpha\right|^{2n}}{n!} e^{-\left|\alpha\right|^2}
\end{equation}
which obeys Poissonian statistics, i.e. its mean is equal to its variance. The thermal state can be shown to possess super-Poissonian photon statistics, with variance larger than its  mean. Conversely, a sub-Poissonian state has reduced photon-number variance, in particular a variance smaller than its mean. The Fock state, with zero photon-number variance, is the limiting example of a sub-Poissonian state.

In addition to the quadrature squeezing discussed above, in which the variance in one quadrature was reduced at the expense of the other, we may think of a photon-number squeezed state in which the photon-number variance is reduced, at the expense of an increase in the phase variance. We display the Wigner functions of states with increasing levels of photon-number squeezing in Fig.~\ref{fig:intro_photon_number_squeezed_wigner}.

\begin{figure}[htp]
\centering
\captionsetup{width=0.8\linewidth}
\includegraphics{intro/photon_number_squeezed_wigner.png}
\caption{\label{fig:intro_photon_number_squeezed_wigner}}
\end{figure}



%\FloatBarrier
\subsection{Mixing, purity, and entanglement}
%I need this section in order to introduce multiple modes, and trace operation.

\subsubsection{Purity}
We have already seen in Sec.~\ref{sec:intro_state_vectors} that the density matrix $\rho$ can encode two types of uncertainty, quantum and classical. The quantum uncertainty is related to superpositions of basis states and is insurmountable, while the classical uncertainty represents ignorance of which quantum state was prepared, and can in principle always be reduced. A state possessing only the first type of ignorance may in general be written as
\begin{equation}\label{eqn:intro_rho_pure}
\rho = \dyad{\psi} \qq{with} \ket{\psi} = \sum_n c_n \ket{n}
\end{equation}
where without loss of generality we have used the Fock basis, and the $c_n$ are complex coefficients. We call the the state Eq.~\ref{eqn:intro_rho_pure} a ``pure'' state. Any state which is not pure is ``mixed''. A state with only the second type of ignorance may in general be written as
\begin{equation}\label{eqn:intro_rho_fully_mixed}
\rho = \sum_n d_n \dyad{n}
\end{equation}
where without loss of generality we have used the Fock basis, and $d_n$ are real coefficients. The state Eq.~\ref{eqn:intro_rho_fully_mixed} is fully mixed.

Given a density matrix $\rho$ it is time consuming and difficult to check by hand which of these forms it takes, and in most situations it will not fit neatly into either form. We therefore desire a function which will measure which type of ignorance the state possesses, and which may be easily computed on $\rho$. To do this, we first introduce the trace of $\rho$ as
\begin{equation}
\tr\left[\rho\right] := \sum_m \mel{m}{\rho}{m},
\end{equation}
where without loss of generality we have used the Fock basis, but we note that we may instead sum over any orthonormal basis for $\mathcal{H}$. It can easily be shown that the trace of a normalized quantum state must be $1$. We introduce the notion of \emph{purity} of a quantum state as a measure of how close to Eq.~\ref{eqn:intro_rho_pure} our state is. This may be quantified and measured by
\begin{equation}\label{eqn:intro_purity}
\tr\left[ \rho^2 \right]
\end{equation}
which we will simply call the purity. It can easily be shown that state Eq.~\ref{eqn:intro_rho_pure} as purity $1$ while state Eq.~\ref{eqn:intro_rho_fully_mixed} has purity $0$.

\subsubsection{Entanglement}

Let us now turn to consider two-mode quantum states. We have already seen that a single-mode quantum state exists as a vector on Hilbert space $\mathcal{H}$. For two modes, we introduce an additional Hilbert space and write $\mathcal{H}_{tot} = \mathcal{H}_1 \otimes \mathcal{H}_2$, where $\mathcal{H}_{1,2}$ are Hilbert spaces of the individual modes, $\otimes$ represents the tensor-product, and $\mathcal{H}_{tot}$ is the total Hilbert space. We may tensor product any two single-mode quantum states together to form a state on $\mathcal{H}_{tot}$, for example
\begin{equation}\label{eqn:intro_separable_1}
\ket{\alpha}_1 \otimes \ket{n}_2
\end{equation}
represents a state on $\mathcal{H}_{tot}$, consisting of a coherent state with amplitude $\alpha$ on $\mathcal{H}_1$, and an $n$ photon Fock state on $\mathcal{H}_2$. For convenience we will often write $\ket{\alpha, n}$ instead of $\ket{\alpha}_1 \otimes \ket{n}_2$.

Now, there are many states which we can write in the form Eq.~\ref{eqn:intro_separable_1}, as a tensor product between two single-mode kets. The general form of this type of ``product-state'' is
\begin{equation}
\rho_{\text{product}} = \dyad{\Psi} \qq{with} \ket{\Psi} = \ket{\psi}_1 \otimes \ket{\phi}_2.
\end{equation}
A general ``separable state'' may be written
\begin{equation}\label{eqn:intro_separable_2}
\rho_{\text{separable}} = \sum_{i, j} c_{i, j} \dyad{\psi_i}_1 \otimes \dyad{\phi_j}_2,
\end{equation}
so-called because the total two-mode state can be separated out into distinct single-mode density operators of modes $1$ and $2$ individually. The separable state is a classical mixture of product states.

Any state which cannot be written in the form Eq.~\ref{eqn:intro_separable_2} is known as a ``non-separable'' or ``entangled'' state. Entangled states cannot be written as a classical mixture over single-mode density operators, and so even full information about each individual mode is not sufficient to fully describe the total system.

\subsubsection{Partial trace}
Let $\rho = \sum_{i, j, k, l} c_{i, j, k, l} \ket{i, j}\bra{k, l}$ be a general two-mode density operator. We define the partial trace over mode $1$ as
\begin{align*}
\tr_1\left[\rho\right] = \sum_n \bra{n}\rho\ket{n} = \sum_n \bra{n} \left[\sum_{i, j, k, l} \left(\ket{i}\bra{j}\right) \otimes \left(\ket{j}\bra{l}\right) \right] \ket{n}& \\
= \sum_n \sum_{i, j, k, l} c_{i, j, k, l} \left(\ip{n}{l} \ip{k}{n} \right) \ket{j}\bra{l}&,
\end{align*}
and the partial trace over mode $2$ is defined analogously.

Defining $\rho_2 = \tr_1\left[\rho\right]$, it can be shown that if $\rho$ is separable, then $\rho_2$ must be pure, i.e. $\tr\left[\rho_2^2\right]=1$. Conversely, if $\tr\left[\rho_2^2\right] < 1$ then the total state $\rho$ must be entangled.







\FloatBarrier
\subsection{Two-mode squeezed vacuum (TMSV)}
Each of the states we have seen so far are single-mode states. Here we meet our first two-mode state, the two-mode squeezed vacuum, written in a Fock-basis expansion as
\begin{equation}
\ket{TMSV} = \frac{1}{\cosh \zeta} \sum_n \left( \tanh \zeta \right)^n \ket{n, n}
\end{equation}
where $\ket{n, n} = \ket{n}_1 \otimes \ket{n}_2$, i.e. a tensor product between Fock states on modes $1$ and $2$. The parameter $\zeta$ controls the level of two-mode squeezing of the state, and thus parametrises both its energy and its level of entanglement. The reduced states of $\ket{TMSV}$ are thermal states with thermal photon number $\bar{n} = \sinh^2 \zeta$, which we display in Fig.\ref{fig:tmsv_wigner}. Remarkably, the state has strong quadrature correlations between each mode
\begin{align*}
q_1 &\sim q_2 \\
p_1 &\sim - p_2
\end{align*}
where the position quadratures are correlated and the momentum quadratures are anticorrelated. We display histograms of quadrature measurements on $\ket{TMSV}$ in Fig.~\ref{fig:tmsv_histogram}. The correlations between modes make the two-mode squeezed vacuum one of the most useful resource states for quantum information processing, and we shall use it extensively in the first part of this Thesis since possession of one of its modes allows information about its second mode to be gained.


\begin{figure}[htp]
\captionsetup{width=0.8\linewidth}
\centering
\includegraphics[draft=false, width=0.8\linewidth]{introduction/tmsv_wigner}
\caption{\label{fig:tmsv_wigner} Plots of the Wigner function for each reduced mode of the TMSV state. Locally the modes look like thermal states, see Fig.~\ref{fig:thermal_state}. The vacuum variance is depicted in orange.}
\end{figure}

\begin{figure}[htp]
\captionsetup{width=0.8\linewidth}
\centering
\includegraphics[draft=false, width=0.8\linewidth]{introduction/tmsv_histogram}
\caption{\label{fig:tmsv_histogram} \MT{caption for TMSV histogram}}
\end{figure}


\FloatBarrier
\section{Quantum measurement}


\FloatBarrier
\subsection{POVMs}


\FloatBarrier
\subsection{Homodyne measurement}


\FloatBarrier
\subsection{Heterodyne measurement}


\FloatBarrier
\section{Modelling the quantum state}


\FloatBarrier
\subsection{Covariance matrix}


\FloatBarrier
\subsection{Beamsplitter relations}


\FloatBarrier
\subsection{Master equation}



\FloatBarrier
\section{Entropy and probability}


\FloatBarrier
\subsection{Hoeffding's inequalities}
Let $\mathcal{X} = X_1, X_2, \dots, X_n$ be $n$ independent binary random variables. Let $\bar{\mathcal{X}}$ be their empirical mean \MT{define} and let $\mathbb{E}\left(\bar{\mathcal{X}}\right)$ be its expected value. Then $\forall \epsilon \ge 0$ we may bound the probability that the empirical mean $\bar{\mathcal{X}}$ differs from its expectation $\mathbb{E}\left(\bar{\mathcal{X}}\right)$ by the following inequalities

\begin{align}
\label{eqn:hoeffding1}
\text{P}\left(\bar{\mathcal{X}} - \mathbb{E}\left(\bar{\mathcal{X}}\right) \ge \epsilon\right) &\le \text{exp}\left(- 2 \epsilon^2 n\right) \\
\label{eqn:hoeffding2}
\text{P}\left(\mathbb{E}\left(\bar{\mathcal{X}}\right) - \bar{\mathcal{X}} \ge \epsilon\right) &\le \text{exp}\left(- 2 \epsilon^2 n\right).
\end{align}





\noindent These inequalities are known as Hoeffding's inequalities \MT{cite} and will provide a necessary tool for analysis of our Quantum Digital Signatures protocol.



\FloatBarrier
\subsection{Shannon entropy}

\FloatBarrier
\subsection{Binary entropy}
\begin{figure}
\centering
\includegraphics{binary_entropy.png}
\caption{\label{fig:binary_entropy}}
\end{figure}


\FloatBarrier
\subsection{Mutual information}


\FloatBarrier
\subsection{Von Neumann entropy}

\FloatBarrier
\subsection{Holevo information}






\iffalse
%
% NOTE: I can stick this in my intro chapter.
%
each received $\rho\left[\phi_{j, m}^{\left(B, C\right)}\right]$ and receives complex phase outcome $x_{B,C}\in\mathbb{C}$. In other words, they perform the POVM
\begin{equation}
E\left[x\right] := \otimes_{j=1}^L E_j\left[x\right] \qq{with} E_j\left[x\right] := \frac{1}{\sqrt{\pi}} \ket{x}_j\bra{x}_j
\end{equation}
with $x \in \mathbb{C}$. 
\fi







