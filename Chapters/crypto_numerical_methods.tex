\chapter{Cryptography: numerical methods}\label{appendix:crpto_numerical_methods}

In this Appendix we display the full form of the quantum states which are used in Chapter.~\ref{chapter:qds} to analyse different eavesdropping attacks. Each state is calculated as described in Sec.~\ref{sec:qds_attack_analysis}, and is conditioned on the honest player possessing outcome $c \in \mathbb{C}$.

We denote each state as $\tilde{\rho}_{\left. \mathbb{B} \given c\right.}$, where $\mathbb{B}$ denotes that the state belongs to the quantum system of dishonest Bob, while $\left. \mathbb{B} \given c \right.$ makes explicit that this state depends on $c$. Since each state $\tilde{\rho}_{\left. \mathbb{B} \given c\right.}$ is a conditional quantum state, they have norm $\tr{\tilde{\rho}_{\left. \mathbb{B} \given c\right.}}\le 1$, since $\tr{\tilde{\rho}_{\left. \mathbb{B} \given c\right.}}$ is the probability, $\text{P}\left(c\right)$, that Charlie measures $c$. The tilde denotes that the state is sub-normalized, and we will define the normalized state as

\begin{equation}
\rho_{\left. \mathbb{B} \given c \right.} := \frac{\tilde{\rho}_{\left. \mathbb{B} \given c\right.}}{\text{P}\left(c\right)}.
\end{equation}

\noindent Unless otherwise specified, it is the normalized forms $\rho_{\left. \mathbb{B} \given c \right.}$ which are used in the main body of this Thesis.

Each of the states required to calculate Holevo information $\chi$ involves sums of the form
\begin{equation}
\sum_{n = 0}^\infty ,
\end{equation}
since each state lives in a Hilbert space $\mathcal{H}$ with a countably infinite dimensionality $\dims$. It is impossible to exactly encode such a state, and so we must resort to a truncation of the Hilbert space size to some large but finite $\dims = N$, i.e.

\begin{equation}
\sum_{n=0}^\infty \rightarrow \sum_{n=0}^N.
\end{equation}
It is still an open question as to whether this will afford an eavesdropper additional powers. In lieu of an answer to this, whenever we numerically encode any of the following states we will choose $\dims$ large enough such that the state living on the truncated space is normalized. Additionally, in each of the calculations in the main body of this Thesis, we gradually increased the Hilbert space size $\dims$ until we converged to identical output $\chi$. For a typical coherent state amplitude $\alpha \le 2$ we chose $\dims \lesssim 10$, though we note that often much smaller $\dims$ were often possible. We note that this strategy is the same one as was adopted in the recent state-of-the-art work Ref.~\cite{Lin2019}.

Finally, we note that each of the following expressions for output states $\tilde{\rho}_{\left. \mathbb{B} \given c\right.}$ was encoded in the displayed Fock basis in a custom script\footnote{Making extensive use of the \code{SparseArray[]} function} in Mathematica~$11.3$. The overall state is then a matrix in $\mathcal{M}_{\dims \times \dims}\left(\mathbb{C}\right)$. The Von Neumann entropy $\text{S}\left(\centerdot\right)$ of the state is then found by taking eigenvalues and using Eq.~\ref{eqn:intro_von_neumann}.

\section{Attack BS$0$}
Beamsplitter attack BS$0$ is described in Sec.~\ref{sec:qds_bs0}. The total input state into the channel is

\begin{equation}
\rho_{\text{input}} = \frac{1}{4}\sum_k \dyad{\alpha_k}_A \otimes \dyad{0}_B
\end{equation}
where
\begin{equation}
\ket{\alpha_k}_A = e^{- \frac{\left| \alpha_k \right|^2}{2}} \sum_{n=0}^\infty \frac{\alpha_k^n}{\sqrt{n!}} \ket{n}_A
\end{equation}
is Alice's input coherent state, with $\alpha_k$ chosen from the QPSK alphabet. 

Enacting beamsplitter relation Eq.~\ref{eqn:intro_beamsplitter} on $\rho_{\text{input}}$ and performing heterodyne detection on Charlie's mode, we arrive at \MT{TODO: make the equation look nice.}

\begin{equation}
\tilde{\rho}_{\left. \mathbb{B} \given c \right.} = \sum_k \frac{1}{\pi} e^{-\left|\alpha_k\right|^2}  e^{-\left|c\right|^2} \sum_{n, m = 0}^\infty \sum_{k, l = 0}^{n, m} \frac{\alpha^n \alpha^{* m}}{\sqrt{k! \left(n - k\right)!}} \frac{c^k c^{* l}}{\sqrt{k! l!}} \frac{\left(\sqrt{T}\right)^{k+l} \left(\sqrt{1-T}\right)^{n + m - k - l}}{\sqrt{l! \left(m-l\right)!}} \dyad{n-l}{m-l}.
\end{equation}

\noindent Rearranging\footnote{See e.g. page \MT{X} of Ref.~\cite{GerryandKnight}} the summation indices we arrive at Eq.~\MT{X} from the main body.

\section{Attack BS$1$}\label{appendix:bs1}
Beamsplitter attack BS$1$ is described in Sec.~\ref{sec:qds_bs1}. The total input state into the channel is 

\begin{equation}\label{eqn:qds_bs1_input_state}
\rho_{\text{input}} = \frac{1}{4}\sum_k \dyad{\alpha_k}_A \otimes \rho_{\text{thermal}}.
\end{equation}
with 

\begin{align}
&\ket{\alpha_k}_A = e^{- \frac{\left| \alpha_k \right|^2}{2}} \sum_{n=0}^\infty \frac{\alpha_k^n}{\sqrt{n!}} \ket{n}_A \qq{and}\notag \\
&\rho_{\text{thermal}} = \left(1 - e^{-\tilde{\beta}}\right) \sum_{p=0}^\infty e^{- p \tilde{\beta}}\dyad{p}_B \qq{with} \tilde{\beta} = \log_e\left(\frac{1}{\bar{n}} +1\right)
\end{align}



\noindent Enacting beamsplitter relation Eq.~\ref{eqn:intro_beampslitter} on $\rho_{\text{input}}$ and heterodyning on Charlie's mode we arrive at \MT{TODO: make the equation look nice. TODO: compare this to my MMA file for consistency}

\begin{align}\label{eqn:appendix_bs1_state}
%
&\tilde{\rho}_{\left. \mathbb{B} \given c\right.} = \sum_k \frac{e^{- \left|\alpha_k\right|^2}}{\pi} e^{-\left|c\right|^2} \left( 1 - e^{- \tilde{\beta}}\right) \sum_{n, m, p = 0}^\infty \frac{\alpha^n \alpha^{* m}}{\sqrt{n! m!}} e^{- p \tilde{\beta}}  \notag \\
%
&\sum_{k_1, k_2, l_1, l_2=0}^{n, p, m, p} c^{k_1 + k_2} \left(c^*\right)^{l_1 + l_2} \pmqty{n \\ k_1} \pmqty{ p \\ k_2} \pmqty{ m \\ l_1 } \pmqty{p \\ l_2} \left( \sqrt{T} \right)^{k_1 + l_1} \notag \\
%
&\left(\sqrt{1-T}\right)^{n + m - k_1 - l_1} \left( - \sqrt{1-T}\right)^{k_2 + l_2} \left( \sqrt{T} \right)^{2 p - k_2 - l_2} \notag \\
%
&\sqrt{\left(n + p - k_1 - k_2\right)!}\sqrt{\left(m + p - l_1 - l_2 \right)!} \dyad{n + p - k_1 - k_2}{m + p - l_1 - l_2}.
%
\end{align}

\section{Attack EC}\label{appendix:ec_state}
Entangling cloner attack EC is described in Sec.~\MT{X}. The total input state into the channel is

\begin{equation}
\rho_{\text{input}} = \frac{1}{4}\sum_k \dyad{\alpha_k}_A \otimes \dyad{\text{TMSV}}_B,
\end{equation}

\noindent with Alice's coherent state
\begin{equation}
\ket{\alpha_k}_A = e^{- \frac{\left| \alpha_k \right|^2}{2}} \sum_{n=0}^\infty \frac{\alpha_k^n}{\sqrt{n!}} \ket{n}_A
\end{equation}
and Bob's two-mode squeezed vacuum state
\begin{equation}
\ket{\text{TMSV}} = \frac{1}{\cosh{\zeta}}\sum_{n=0}^\infty \left(\tanh{\zeta}\right)^n \ket{n, n}_B.
\end{equation}

\noindent Enacting beamsplitter relation Eq.~\ref{eqn:intro_beamsplitter} on $\rho_{\text{input}}$ and heterodyning on Charlie's mode, we arrive at \MT{TODO: make equation look pretty.}

\begin{align}\label{eqn:appendix_ec_state}
&\tilde{\rho}_{\left. \mathbb{B} \given c\right.} = \sum_k\frac{e^{- \left|\alpha_k\right|^2}}{\cosh^2 \zeta}\frac{e^{-\left|c\right|^2}}{\pi} \sum_{n_1, m_1, n_2, m_2=0}^\infty \sum_{k_1, k_2 = 0}^{n_1, n_2} \sum_{l_1, l_2 = 0}^{m_1, m_2} \alpha^{n_1} \alpha^{* m_1} \left(\tanh{\zeta}\right)^{n_2 + m_2} \notag \\
%
&c^{k_1 + k_2} \left(c^*\right)^{l_1 + l_2} \sqrt{n_2! m_2!} \left(\sqrt{T}\right)^{k_1 + l_1} \left(\sqrt{1-T}\right)^{n_1 + m_1 - k_1 - l_1} \left(- \sqrt{1-T}\right)^{k_2 + l_2} \notag \\ %
&\left( \sqrt{T}\right)^{n_2 + m_2 - k_2 - l_2}
\frac{\sqrt{\left(n_1 + n_2 - k_1 - k_2\right)!} \sqrt{\left(m_1 + m_2 - l_1 - l_2\right)!}}{k_1! k_2! l_1! l_2! \left(n_1 - k_1\right)! \left(n_2 - k_2\right)! \left(m_1 - l_1\right)! \left(m_2 - l_2\right)!} \notag \\
%
& \dyad{n_1 + n_2 - k_1 - k_2}{m_1 + m_2 - l_1 - l_2}\otimes \dyad{n_2}{m_2}.
\end{align}

To illustrate the power of the EC attack we examine several histograms of $\rho_{\left. \mathbb{B} \given c\right.}$ and show how the correlations between Bob's two modes give him additional information. \MT{TODO: make some nice histograms.}





%
%
%\subsubsection{BS$2$: $\xi > 0$}\label{sec:qds_bs2}
%We now introduce and consider a second modification to attack BS$0$, which also allows the excess noise on the channel to be modelled. This attack can be viewed as an unphysical combination of attacks BS$0$ and BS$1$ which advantages Bob and disadvantages honest parties.
%
%As we shall see below in Fig.~\ref{fig:qds_holevo_comparisons}, Bob performs worse under BS$1$ than BS$0$. This is a consequence of the fact that the input noise also affects his own measurements and so reduces his information about Charlie's output. We here modify the attack by imposing that the channel excess noise should \emph{only} affect honest players. That is, the presence of $\xi$ causes $\perr$ to increase, but Holevo information and therefore $\pe$ should both be unaffected. 
%
%While strictly this attack is physically inconsistent, and therefore impossible for Bob to perform, it is more pessimistic than either BS$0$ or BS$1$ and so the security bounds it gives are also safe bounds on both of those attacks. Indeed, since an analytic expression for Bob's states under attack BS$0$ is readily attainable without resort to the numerical methods required for BS$1$ (Appendix~\ref{appendix:crypto_numerical_methods}), in some circumstances it may be computationally preferable to assume attack BS$2$.
%
%The Holevo information is given by using Eqs.~\ref{eqn:qds_aposterioristate} and \ref{eqn:qds_aprioristate} in the usual way, while $\perr$ is given by Eq.~\ref{fig:qds_perr_noisy}. 
%
%



