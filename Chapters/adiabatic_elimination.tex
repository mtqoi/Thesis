%
% Q: do I need to reference Shchesnovich2011? Or other places?
%
%


\chapter{Adiabatic elimination}\label{appendix:adiabatic_elimination}

In this appendix we will derive Eq.~\MT{X} which is used in Sec.~\ref{sec:phog_three_mode_model} to simplify the analytical description of a multi-mode bosonic system when one of the modes takes its steady-state. 

Consider a system involving a highly lossy bosonic mode, $c$. We assume that $c$ decays into a Markovian reservoir with decay rate $\gamma$, and that the system contains additional modes with possible losses, couplings and nonlinearities. 

After $t \gg 1/\gamma$ the mode $c$ is empty, and it is assumed that this timescale is much faster than all other timescales in the system. Then, for $t \gg 1/\gamma$ the mode $c$ can be assumed to be in its steady-state and so, for example, $\ddt \hat{c} = 0$, which will allow us to simplify the description of our original system.

The Lindblad master equation for our entire system is

\begin{equation}
\ddt \rho = - i \left[ \hat{H}, \rho \right] + \gamma \mathcal{L}\left(\hat{c}\right) \rho + \Gamma_j\mathcal{L}\left(f\left(\hat{a}_j\right)\right) \rho
\end{equation}
where $\rho$ is the density matrix describing the entire system and $\hat{H}$ is the system's Hamiltonian, on which we will later derive some restrictions. The term in $\gamma$ is strong loss of mode $c$ into the reservoir, and the term in $\Gamma$ denotes potential other sources of loss which do not affect $c$.

 \MT{Talk to Pablo about this transformation}
It is easiest to proceed in Heisenberg picture, and so we transform to the adjoint master equation \cite{Breuer2002} describing the evolution of an arbitrary operator $\hat{A}$

\begin{equation}\label{eqn:appendix_phog_adjoint_start}
\ddt \hat{A} = i \left[ \hat{H}, \hat{A} \right] + \gamma \mathcal{L}\left(\hat{c}^\dagger \right) \hat{A} + \Gamma_j \mathcal{L} \left( f\left(\hat{a}_j\right)^\dagger\right) \hat{A}
\end{equation}
and expand $\hat{H}$ and $\hat{A}$ in terms of normal-ordered powers of creation and annihilation operators of the lossy mode
\begin{align}
\hat{H} = \sum_{p, q = 0}^{\infty} \hat{H}^{\left(p, q\right)} \left(\hat{c}^\dagger\right)^p \hat{c}^q \label{eqn:appendix_phog_H_expand}  \\
%
\hat{A} = \sum_{p, q = 0}^{\infty} \hat{A}^{\left(p, q\right)} \left(\hat{c}^\dagger \right)^p \hat{c}^q \label{eqn:appendix_phog_A_expand}
\end{align}
where operators $\hat{H}^{\left(p, q\right)}, \hat{A}^{\left(p, q\right)}$ are, in general, operators acting on the remaining modes of the system, and $\hat{A}^{\left(p, q\right)}$ is time-dependent.

Let us take Eqs.~\ref{eqn:appendix_phog_H_expand},~\ref{eqn:appendix_phog_A_expand} and substitute into Eq.~\ref{eqn:appendix_phog_adjoint_start}. Grouping terms in $\hat{c}^\dagger, \hat{c}$ we see that \MT{TODO: change indices on LHS}

\begin{align}
\ddt A^{\left(p, q\right)} c^{\dagger p} c^q = i \sum_{p, q, k, l} \left(H^{\left(k, l\right)} A^{\left(p, q\right)} c^{\dagger k} c^{l} c^{\dagger p} c^q - A^{\left(p, q\right)} H^{\left(k, l\right)} c^{\dagger p} c^q c^{\dagger k} c^l\right)& \notag \\
%
+ \qq{terms in $\gamma$}&
\end{align}
where we have absorbed terms in $\Gamma_j$ into the definition of $H^{\left(p, q\right)}$ \MT{can I do this?}. We have also dropped hats from operators for ease of notation.

Relabelling dummy indices,
\begin{equation}\label{eqn:appendix_phog_Apq}
\ddt A^{\left(p, q\right)} c^{\dagger p} c^q = i \sum_{p, q, k, l} F\left(p, q, k, l\right) c^{\dagger p} c^{q} c^{\dagger k} c^l + \qq{terms in $\gamma$}
\end{equation}
where
\begin{equation}
F\left(p, q, k, l\right) = H^{\left(k, l\right)} A^{\left(p, q\right)} - A^{\left(p, q\right)} H^{\left(k, l\right)}. \notag
\end{equation}

The differential equation for $A^{\left(0, 0\right)}$ is such that there are no operators in mode $c$ remaining on the right hand side of Eq.~\ref{eqn:appendix_phog_Apq}. Clearly this occurs when $p=q=k=l=0$. It can also occur for other values of $p, q, k, l$, owing to constant terms appearing the commutators of $\hat{c}$ operators. Additionally, since the lindblad operator $\mathcal{L}$ is second-order in $\hat{c}$ we can see immediately that there will be no terms proportional to $\gamma$ in the equation for $A^{\left(0, 0\right)}$. 

Writing Eq.~\ref{eqn:appendix_phog_Apq} in normal order using the commutator,
\begin{equation}
\ddt A^{\left(p, q\right)} c^{p, \dagger} c^q = i \sum_{p, q, k, l}\left( F\left( p, q, k, l\right) c^{\dagger p} c^{\dagger k} c^q c^l + F\left(p, q, k, l\right)  \left[ c^{q}, c^{\dagger k}\right] c^{\dagger p} c^l \right)
\end{equation}
we observe that additional contributions to $A^{\left(0, 0\right)}$ are possible when $p=0, l=0$ but $q, k \ne 0$ provided that the commutator $\left[ c^q, c^{\dagger k}\right]$ contains a constant term. We observe

\begin{align}\label{eqn:phog_appendix_commutator}
&\left[ c, c^\dagger \right] = 1 \notag \\
%
&\left[ c^2, c^{\dagger 2} \right] = 4 c^\dagger c + 2 \notag \\
%
&\left[ c^3, c^{\dagger 3} \right] = 9 c^\dagger c^\dagger c c + 18 c^\dagger c + 6 \notag
\end{align}
while commutators with $q \ne k$ cannot give a constant term. 

For ease, and because the largest terms considered in Ch.~\ref{chapter:phog} are of the form $c^\dagger c^\dagger c c$ we will restrict ourselves to $0 \le q, k \le 2$ and so

\begin{equation}
\ddt A^{\left(0, 0\right)} = i \left( \left[ H^{\left(0, 0\right)}, A^{\left(0, 0\right)}\right] + F\left(0, 1, 1, 0\right) + 2 F\left(0, 2, 2, 0\right)\right) \notag
\end{equation}
and so 
\begin{align}
\ddt A^{\left(0, 0\right)} = i  &\left(\left[ H^{\left(0, 0\right)}, A^{\left(0, 0\right)}\right] + H^{\left(0, 1\right)} A^{\left(1, 0\right)} - A^{\left(0, 1\right)} H^{\left(1, 0\right)} \right. \notag \\
%
&+\left. 2 H^{\left(0, 2\right)} A^{\left(2, 0\right)} - 2 A^{\left(0, 2\right)} H^{\left(2, 0\right)} \right).
\end{align}






\MT{----------------}

Finally, we arrive at 
\begin{equation}
\ddt A^{\left(0, 0\right)} = i \left[ H^{\left(0, 0\right)}, A^{\left(0, 0\right)}\right] + \frac{4}{\gamma} \sum_{p = {1, 2}} \mathcal{L}\left( H^{\left(0, p\right)} \right) A^{\left(0, 0\right)} + \Gamma_j \mathcal{L}\left(a_j\right) \rho %Can I just stick the extra loss term here? Is it affected by the adiabatic elimination at all?
\end{equation}
and so transforming back to our master equation in Lindblad form
\begin{equation}\label{eqn:adiabatic_elimination}
\ddt \rho = - i \left[ H^{\left(0, 0\right)}, \rho\right] + \frac{4}{\gamma} \sum_{p = 1, 2} \mathcal{L} \left(H^{\left(0, p\right)\dagger}\right) \rho.
\end{equation}
This Eq.~\ref{eqn:adiabatic_elimination} is our key equation for performing the adiabatic elimination of the highly lossy mode $c$. The recipe to apply it to a general system is to identify the $H^{\left(0, 0\right)}$ and $H^{\left(0, p\right)}$ terms, which can take arbitrary form. The only requirement for the use of Eq.~\ref{eqn:adiabatic_elimination} is that $\hat{H}$ must have its largest term in lossy mode $c$ of the form $\hat{c}^\dagger \hat{c}^\dagger \hat{c} \hat{c}$, i.e. two creation and two annihilation operators. More general forms of Eq.~\ref{eqn:adiabatic_elimination} may be considered by continuing to higher-order commutators $\left[ \hat{c}^n, \hat{c}^{\dagger n}\right]$ which has constant term $n!$, allowing for different maximum combinations of $\hat{c}$ operators 

