\documentclass{article}

\usepackage[noBBpl]{mathpazo}
\usepackage{color}
\usepackage{amsmath}
\usepackage{physics}
\usepackage{geometry} \geometry{ a4paper, total={170mm,257mm}, left=20mm, top=20mm,}
\def\MT #1{\textcolor{magenta}{#1}}
\title{Title of my thesis \\ \vspace{0.2in}  \Large Thesis plan}
\author{Matthew Thornton}
\date{\today}
\begin{document}
\maketitle 


\section*{Changelog}
\begin{itemize}
\item $2020/02/17$ added part~$2$ plan
\item $2020/02/17$ updated part~$1$ plan to what I currently have written. I have used section titles from the table of contents
\item $2020/02/17$ updated part~$1$ section name
\item $2020/02/17$ changed title to make clear it is draft
\item $2020/01/09$ merged chapter ``Cryptography literature review" with chapter ``Quantum digital signatures"
\end{itemize}

\subsection*{1 Introduction and background material}
Goal of chapter: Introduce equations, frameworks and states which I can continually refer back to in the rest of the thesis.

\section*{Part one: Agile cryptography: signatures and secrets}
Short introduction (/abstract) to section, briefly mentioning what my contributions are. This whole section is flowing towards agility.

\subsection*{2 Introduction to cryptography}
Goal of chapter: consistent literature review which covers chapters $3$ (QDS) and $4$ (QSS), with sufficient background to motivate quantum cryptography in general.

\subsection*{3 Quantum digital signatures}
Goal of chapter: introduce our QDS protocol and prove its security in different contexts using several methods.

\begin{enumerate}
\item Our QDS protocol
\item Security against repudiation
\item Robustness
\item Security against forgery
\item Bounding $\text{p}_{\text{e}}$
\item Attack analysis
\item Signature length $L$
\item Postselection
\item Protocol performance
\item Outlook
\end{enumerate}

\subsection*{4 Quantum secret sharing}
Goal of chapter: introduce our QSS protocol and prove its security in different contexts

\begin{enumerate}
\item Our QSS protocol
\item Security against Eve
\item Security against a dishonest player
\item Protocol performance
\item Outlook
\end{enumerate}


\subsection*{5 Agile quantum cryptography}
Goal of chapter: discuss and motivate concept of agility as it applies to quantum cryptography

\begin{enumerate}
\item Introduction
\item CV agile quantum system
\item Agile system QDS-$b$-QSS-$b$-CV-QPSK
\item Agile system QDS-$f$-QKD-$f$-CV-QPSK
\item Experimental implementation
\item Data analysis
\item Outlook
\end{enumerate}

\section*{Part two: coherent signal transfer in quantum networks}
\subsection*{\MT{Chapter title}}
Goal of chapter: analyse and model the PhoG device and its efficacy for producing, from a classical input, (i) bright sub-Poissonian state; (ii) entangled state

Planned structure for chapter:
\subsubsection*{Introduction}
\begin{itemize}
\item introduce dissipation as a means for state engineering
\item intoduce our goal to produce single-photons (or close to single-photons)
\item introduce this chapter, include a chapter outline, and provide motivation for why we will look at different models
\end{itemize}

\subsubsection*{Single-mode model}
\begin{itemize}
\item Single mode model - introduce Lindblad equation with a general reservoir operator
\item Let's consider some steady-states: $\hat{a}, \hat{a}^2, \hat{a}^3, \hat{a}\left(\hat{n}\right), \hat{a}\left(\hat{n} - 1\right), \hat{a}\left(\hat{n}-2\right)$
\item Have some graphs of $\langle \hat{n}\rangle, Var\left(\hat{x}\right), Var\left(\hat{p}\right)$ and fielity to a target state (e.g. $\ket{0}, \ket{1}$, phase state). All graphs wrt time.
\item (Stills from my animation of Wigner function evolution)
\end{itemize}


\subsubsection*{Including loss}
\begin{itemize}
\item Begin taking combinations of the above operators
\item Loss spoils our evolution and prevents us from reaching a nice output state (reference e.g. outrun linear loss paper 2013)
\item Two options. (i) Stop evolution somewhere nice. (ii) Use bright input state. We will do both of these. (\MT{make sure to note why using a bright input state is helpful $\rightarrow$ because for large $n$, $n^3 > n^1$.})
\item Analyse single-mode model, varying loss parameters and input amplitude. 
\item Observe ``signature behaviours''.  \MT{I'll need to introduce MandelQ in this section}
\end{itemize}

\subsubsection*{Two-mode model}
\begin{itemize}
\item Demonstrate how it reduces to single-mode model
\item Analyse it for signature behaviours, and compare it to single-mode model with equivalent parameters
\end{itemize}

\subsubsection*{Three-mode model}
\begin{itemize}
\item Demonstrate how it reduces to two-mode model
\item Analyse it for signature behaviours, and compare it to single-mode and two-mode model with equivalent parameters
\end{itemize}
Note: over the last three sections it should be clear that $\left|\mathcal{H}\right|$ is decreasing as I increase the number of modes. This will provide a nice segue into the next section.

\subsubsection*{Numerical methods and linearization}
\begin{itemize}
\item Direct integration
\item Monte-carlo
\end{itemize}
Neither of these will work for many modes, so we need something better.
\begin{itemize}
\item Meanfield
\item Linearization (and how do we do it?)
\item Comparison of single-mode, two-mode and three-mode models under (i) meanfield and (ii) linearization, with equivalent parameters between models
\end{itemize}

\subsubsection*{Multi-mode model}
\begin{itemize}
\item Motivate model
\item Show how we approach it in MF and Linearization
\item Show signature behaviours
\item Show parity dependence 
\end{itemize}

\subsubsection*{Entanglement}

\subsubsection*{Conclusion and outlook}




\end{document}